%%17:33:53 21/10/2016 -VieTeX creates E:\tex\book-mau\mau-dethi30\cauhoi02-tracnghiem.tex
\baitracnghiem{dtracnghiem:b01}{%
Tập xác định của hàm số $y=\dfrac{\sqrt{x^2-5x+6}}{x+2}$ là:
}{
\datcot[2]
\bonpa
{\sai{$R\setminus \{ 3;2;-2\}$}}
{\sai{$R\setminus [2;3]$}}
{\sai{$(-\infty ,2]\cup [3,+\infty)$}}
{\dung{$(-\infty ,2]\cup [3,+\infty)\setminus \{-2\}$}}
}
%-[câu hỏi 2]%%%%%%%%%%
\baitracnghiem{dtracnghiem:b02}{%
Đạo hàm tại $x=-1$ của hàm số $y=x^3-3x-4$ là:
}{
\datcot
\bonpa
{\dung{$0$}}
{\sai{$6$}}
{\sai{$2$}}
{\sai{$3$}}
}
%-[câu hỏi 3]%%%%%%%%%%
\baitracnghiem{dtracnghiem:b03}{%
Đạo hàm của hàm số $y=\dfrac{x^2-3x+1}{x-2}$ tại $x\in R\setminus \{ 2\}$ là:
}{
\datcot[2]
\bonpa
{\sai{$y'=\dfrac{x^2-4x-7}{(x-2)^2}$}}
{\dung{$y'=\dfrac{x^2-4x+5}{(x-2)^2}$}}
{\sai{$y'=\dfrac{3x^2-10x+7}{(x-2)^2}$}}
{\sai{$y'=\dfrac{x^2+4x-5}{(x-2)^2}$}}
}
%-[câu hỏi 4]%%%%%%%%%%
\baitracnghiem{dtracnghiem:b04}{%
Phương trình tiếp tuyến với đồ thị $(C)$ : $y=x^3-3x-4$ đi qua điểm $(-1;-2)$ là:
}{
\datcot[4]
\bonpa
{\sai{$y=-2; x=-1$}}
{\sai{$y=-2$ và $y=-\dfrac{9}{4}x+\dfrac{17}{4}$; $y=\dfrac{9}{2}x+\dfrac{17}{2}$}}
{\dung{$y=-2$ ; $9x+4y+17=0$}}
{\sai{$y=-2$ ; $y=-\dfrac{9}{4}x+\dfrac{17}{4}$}}
}
%-[câu hỏi 5]%%%%%%%%%%
\baitracnghiem{dtracnghiem:b05}{%
Hệ số góc của tiếp tuyến với đồ thị $(C)$: $y=\dfrac{x^2-3x+1}{x-2} $ tại $M(1;1)$ là:
}{
\datcot
\bonpa
{\dung{$2$ }}
{\sai{$\dfrac{1}{2}$ }}
{\sai{$-2$ }}
{\sai{$\dfrac{9}{4}$ }}
}
%-[câu hỏi 6]%%%%%%%%%
\baitracnghiem{dtracnghiem:b06}{%
Hàm số $y=x^3-3x-4$ đồng biến trên miền nào dưới đây:
}{
\datcot[2]
\bonpa
{\sai{$(-\infty ,-1)\cup (1, +\infty)$}}
{\dung{$(-\infty ,-1)$ và $(1, +\infty)$}}
{\sai{$R\setminus [-1;1]$}}
{\sai{$R\setminus \{-1;1\}$}}
}
%-[câu hỏi 7]%%%%%%%%%%
\baitracnghiem{dtracnghiem:b07}{%
Cho hàm số $y=(m^2-1)\dfrac{x^3}{3}+(m+1)x^2+3x+5$; Để hàm số đồng biến trên $R$ thì giá trị của $m$ là:
}{
\datcot
\bonpa
{\sai{$m=\pm 1$}}
{\sai{$m\leq -1$}}
{\sai{$m\geq 2$}}
{\dung{$\left[	\begin{array}{l}	m\leq -1\\	m\geq 2	\end{array}\right.$}}
}
%-[câu hỏi 8]%%%%%%%%%%
\baitracnghiem{dtracnghiem:b08}{%
Cho hàm số $y=f(x)$ xác định và liên tục trên miền $K$. Điều kiện để hàm số có cực trị tại $x_0$ là: 
}{
\datcot[4]
\bonpa
{\sai{$x_0\in K, y'(x_0)=0$ và $y'$ đổi dấu khi qua $x_0$}}
{\sai{$y'(x_0)=0$ hoặc $y'(x_0)=0$ không xác định}}
{\dung{$x_0\in K, y'(x_0)=0$ hoặc $y'(x_0)=0$ không xác định, $y'$ đổi dấu khi qua $x_0$}}
{\sai{$x_0\in K, y'(x_0)$ không xác định, $y'$ đổi dấu khi qua $x_0$}}
}
%-[câu hỏi 9]%%%%%%%%%%
\baitracnghiem{dtracnghiem:b09}{%
Cho hàm số $y=x^3-3x-4$. Giá trị lớn nhất của hàm số trên đoạn $[-2;2]$ là:
}{
\datcot
\bonpa
{\dung{$-2$}}
{\sai{$-1$}}
{\sai{$2$}}
{\sai{$0$}}
}
%-[câu hỏi 10]%%%%%%%%%%
\baitracnghiem{dtracnghiem:b10}{%
Đồ thị hàm số $y=x^3-3x-4$ lồi trên miền: 
}{
\datcot
\bonpa
{\sai{$(0; +\infty)$}}
{\sai{$R$}}
{\dung{$(-\infty ;0)$}}
{\sai{$(-\infty ;0)\cup (0; +\infty)$}}
}
%-[câu hỏi 11]%%%%%%%%%%
\baitracnghiem{dtracnghiem:b11}{%
Đồ thị hàm số $y=\dfrac{x^2-3x+1}{x-2}$ có các tiệm cận sau:
}{
\datcot[2]
\bonpa
{\sai{$y=2$ và $y=x-1$}}
{\dung{$y=x-1$ và $x=2$}}
{\sai{$x=2$ và $y=x+1$}}
{\sai{$x=2$ và $y=-x+1$}}
}
%-[câu hỏi 12]%%%%%%%%%
\baitracnghiem{dtracnghiem:b12}{%
Cho hàm số $y=x^3-3x+2-m$. Đồ thị hàm số cắt trục hoành tại 3 điểm phân biệt khi:
}{
\datcot
\bonpa
{\dung{$0<m<4$}}
{\sai{$0\leq m\leq 4$}}
{\sai{$m>4$}}
{\sai{$m<0$}}
}
%-[câu hỏi 13]%%%%%%%%%%
\baitracnghiem{dtracnghiem:b13}{%
Cho đồ thị $(L)$: $y=\dfrac{x^2+mx-1}{x-1}$ và đường thẳng $(d): y=mx+2$,
$(L) $ cắt $(d)$ tại 2 điểm phân biệt khi:
}{
\datcot
\bonpa
{\sai{$\left[	\begin{array}{l}	m\leq 0\\	
m\geq 1	\end{array}\right.$}}
{\sai{$\left[	\begin{array}{l}	m<0\\	
m\geq 1	\end{array}\right.$}}
{\dung{$\left[	\begin{array}{l}	m<0\\	
m>1	\end{array}\right.$}}
{\sai{$\left[	\begin{array}{l}	m\leq 0\\	
m>1	\end{array}\right.$}}
}
%-[câu hỏi 14]%%%%%%%%%%
\baitracnghiem{dtracnghiem:b14}{%
Cho $C$ là hằng số tuỳ ý. Các nguyên hàm của hàm số $y=\dfrac{lnx}{x}, x>0$ có dạng:
}{
\datcot
\bonpa
{\sai{$\dfrac{ln^2x}{2}$}}
{\dung{$\dfrac{ln^2x}{2}+C$}}
{\sai{$2lnx+C$}}
{\sai{$\dfrac{ln^2x}{x^2}+C$}}
}
%]-%%%%%%%%%%%%%%%

%-[câu hỏi 15]%%%%%%%%%%
\baitracnghiem{dtracnghiem:b15}{%
Một nguyên hàm của hàm số $y=2\sin x\cos 3x+x$ là:
}{
\datcot[2]
\bonpa
{\sai{$\dfrac{1}{4}\cos 4x-\dfrac{1}{2}\cos 2x+\dfrac{x^2}{2}$}}
{\sai{$-\dfrac{1}{4}\sin 4x+\dfrac{1}{2}\sin 2x+\dfrac{x^2}{2}$}}
{\dung{ $-\dfrac{1}{4}\cos 4x+\dfrac{1}{2}\cos 2x+\dfrac{x^2}{2}+3$}}
{\sai{$-\dfrac{1}{4}\sin 4x+\dfrac{1}{2}\sin 2x+\dfrac{x^2}{2}+5$}}
}
%%%%%%%%%%%%%%%%%%%%%%%
\baitracnghiem{dtracnghiem:b16}{%
Kết quả của $I=\int\limits_{\frac{\pi }{6}}^{\frac{\pi }{3}}\sqrt{1-\sin 2x}dx$ là
}{
\datcot
\bonpa
{\dung{$2\sqrt 2-1-\sqrt 3$}}
{\sai{$2\sqrt 2+1+\sqrt 3$}}
{\sai{$0$}}
{\sai{$\dfrac{2\sqrt 2-1+\sqrt 3}{2}$}}
}
%%%%%%%%%%%%%%%%%%%%%%%
\baitracnghiem{dtracnghiem:b17}{%
Biểu thức phép tính tích phân của $I=\int\limits_{\frac{\pi }{6}}^{\frac{\pi }{3}}\sqrt{1-\sin 2x}dx$ khi lấy ra khỏi dấu tích phân là:
}{
\datcot[2]
\bonpa
{\sai{$(\cos x-\sin x)\Big |_{\frac{\pi }{6}}^{\frac{\pi }{3}}$}}
{\sai{$(\cos x+\sin x)\Big |_{\frac{\pi }{6}}^{\frac{\pi }{3}}$}}
{\dung{$(\cos x+\sin x)\Big |_{\frac{\pi }{6}}^{\frac{\pi }{4}}-(\cos x+\sin x)\Big |_{\frac{\pi }{4}}^{\frac{\pi }{3}}$}}
{\sai{$(\cos x-\sin x)\Big |_{\frac{\pi }{6}}^{\frac{\pi }{4}}-(\cos x-\sin x)\Big |_{\frac{\pi }{4}}^{\frac{\pi }{3}}$}}
}
%%%%%%%%%%%%%%%%%%%%%%%
\baitracnghiem{dtracnghiem:b18}{%
Để tính $I=\int\limits_{\frac{\pi }{6}}^{\frac{\pi }{3}}\sqrt{1-\sin 2x}dx$, một học sinh đã thực hiện các bước sau:\\
Bước 1: $I=\int\limits_{\frac{\pi }{6}}^{\frac{\pi }{3}}\sqrt{\sin^2x+cos^2x-2\sin x \cos x}dx$\\
Bước 2: $I=\int\limits_{\frac{\pi }{6}}^{\frac{\pi }{3}}\sqrt{(\sin x-\cos x)^2}dx$\\
Bước 3: $I=\int\limits_{\frac{\pi }{6}}^{\frac{\pi }{3}}(\sin x-\cos x)dx$\\
Bước 4: $I=\int\limits_{\frac{\pi }{6}}^{\frac{\pi }{3}}\sin xdx-\int\limits_{\frac{\pi}{6}}^{\frac{\pi }{3}}\cos xdx$\\
Bước 5: $I=\cos x\Big |_{\frac{\pi }{6}}^{\frac{\pi }{3}}+\sin x\Big |_{\frac{\pi }{6}}^{\frac{\pi }{3}}$\\
Các bước biến đổi sai so với bước ngay trên nó là:
}{
\datcot
\bonpa
{\sai{bước 3 và 4}}
{\sai{bước 2 và 3}}
{\dung{bước 2 và 4}}
{\sai{bước 3 và 5}}
}
%%%%%%%%%%%%%%%%%%%%%%%
\baitracnghiem{dtracnghiem:b19}{%
Trong trường có 8 đội bóng đá. Trường muốn cho các đội thi đấu giao hữu sao cho đội nào cũng được đấu một trận với đội còn lại. Số trận đấu phải tổ chức là:
}{
\datcot
\bonpa
{\sai{$14$}}
{\dung{$28$}}
{\sai{$56$}}
{\sai{$32$}}
}
%%%%%%%%%%%%%%%%%%%%%%
\baitracnghiem{dtracnghiem:b20}{%
Một tổ học sinh gồm 3 nam và 7 nữ, cần lập một nhóm học tập gồm 5 người, trong đó phải có ít nhất 1 nam. Số cách lập nhóm học tập là:
}{
\datcot
\bonpa
{\sai{$252$}}
{\dung{$231$}}
{\sai{$105$}}
{\sai{$30240$}}
}

