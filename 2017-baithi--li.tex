%Tệp mẫu làm đề thi trắc nghiệm phiên bản 3.0
%Tác giả Nguyễn Hữu Điển (ĐHKHTN, Hà Nội)
% Đề trắc nghiệm được thiết kế trên phông Unicode,
%Đã dùng lớp examdesign.cls có sửa đổi
%Cùng với gói lệnh dethi.sty tạo ra:
%Đề thi trắc nghiệm từ một bộ đề sinh ra các câu hởi được 
%sắp xếp ngẫu nhiên và các chi tiết của câu hỏi cũng được 
%xắp sếp ngẫu nhiên. Mỗi đề thi sinh ra đều có thể in ra đáp án riêng biệt.
%examdesign.cls đòi hỏi các gói lệnh enumerate, multicol, shortlst, keyval.
\documentclass[11pt]{article}
\usepackage{amsmath,amsxtra,latexsym, amssymb, amscd}
\usepackage[utf8]{vietnam}

\usepackage{color}
\usepackage{graphicx}
\usepackage{picinpar}
\usepackage{mathptmx} 
% \usepackage{mathpazo} 
\usepackage{enumerate}
\usepackage{multicol}
\usepackage{shortlst}
\usepackage[baithi]{dethi} %Gói lệnh cho đề thi Việt Nam
% \usepackage{fancybox}
% \cornersize*{3.6mm}
\Fullpages %Định dạng trang đề thi
\ContinuousNumbering %Đánh số liên tục các bài thi
\NumberOfVersions{3} %10 là số bài thi khác nhau được in ra
\SectionPrefix{\relax }%\bf Phần \Roman{sectionindex}. \space}
\tieudetracnghiem
\tieudedapan
%\tieudetren
\tieudeduoi
\daungoac{}{.}                  %Dấu quanh phương án trả lời: {(}{)};{}{.};{}{)}
%\chuphuongan{\alph}    %Ký tự cho các phương án
%\chuphuongan{\arabic} %\Roman%\roman%kể cả số cho các phương án
\chucauhoi{Câu}                %Chữ trước các số câu hỏi
\mauchu{red}                     %Mầu số câu hỏi và phương án
\setlength{\baselineskip}{12truept}
\def\v#1{\overrightarrow{#1}} %Làm vectơ
\graphicspath{{hinh-cauhoi/}} %Đường dẫn của nơi để hình
\khoanh{\cbox}         %Khoanh các phương án: \cbox, \fbox
\hovaten{Họ và tên}         %Nếu không muốn có dòng này không gõ lệnh
% \tenlop{Tên lớp}         %Nếu không muốn có dòng này không gõ lệnh
\sobaodanh{Số báo danh}  %Nếu không muốn có dòng này không gõ lệnh
%\ketqua{}          %In ra phần Kết quả
%\giamkhao{}     %In ra phần chữ ký giám khảo ở phiếu thi
%\NoRearrange  %Lệnh không trộn đề
% \motphieuthi      %In ra một phiếu thi, Mặc định là không hiện ra phiếu thi
%\nhieuphieuthi   %In ra mỗi đề một phiếu thi
%\coloigiai           %In ra đáp án có lời giải
\ShortKey             %Lệnh hiện ra đáp án mỗi đề thi
%\OneKey            %Lệnh chỉ in ra 1 bản đáp án
%\NoKey               %Lệnh không in ra phần đáp án
\soanthao %Lệnh dùng khi soạn cauu hỏi, khi dịch 1 bản và không đảo
\tentruong{BỘ GIÁO DỤC VÀ ĐÀO TẠO}
\tenkhoa{ĐỀ MINH HỌA}
\loaidethi{Đề gồm có 04 trang}%{ĐỀ THI LẠI}%%{ĐỀ CHÍNH THỨC}
\tenkythi{KÌ THI TRUNG HỌC PHỔ THÔNG QUỐC GIA NĂM 2017}
\tenmonhoc{Bài thi: Khoa học tự nhiên;  Môn: VẬT LÍ}
\madethi{100}
\thoigian{\underline{Thời gian làm bài: 50 phút, không kể thời gian phát đề}}
\begin{document}

\setlength{\baselineskip}{12truept}
 \begin{vnmultiplechoice}[ rearrange=yes, keycolumns=3]%

\begin{question} %%01
Một con lắc lò xo gồm một vật nhỏ khối lượng m và lò xo có độ cứng k. Con lắc dao
động điều hòa với tần số góc là
\datcot
\bonpa
{\sai{$2\pi\sqrt{\dfrac{m}{k}}$.}}
{\sai{$2\pi\sqrt{\dfrac{k}{m}}$.}}
{\sai {$\sqrt{\dfrac{m}{k}}$.}}
{\dung{$\sqrt{\dfrac{k}{m}}$.}}
\end{question}

\begin{question} %%02
Một chất điểm dao động điều hòa với phương trình $x=A\cos(\omega t+\varphi)$; trong đó $A, \omega$ là
các hằng số dương. Pha của dao động ở thời điểm $t$ là
\datcot
\bonpa
{\dung{$(\omega t+\varphi)$.}}
{\sai{$\omega$.}}
{\sai{$\varphi$.}}
{\sai {$\omega t$.}}
\end{question}


\begin{question} %%03
Hai dao động có phương trình lần lượt là: $x_1=5\cos(2\pi t+0,75\pi)$ (cm) và $x_2=10\cos(2\pi t+0,5\pi)$ (cm).
Độ lệch pha của hai dao động này có độ lớn bằng
\datcot
\bonpa
{\dung{$0,25\pi$.}}
{\sai{$1,25\pi$.}}
{\sai{$0,50\pi$.}}
{\sai {$0,75\pi$.}}
\end{question}

\begin{question} %%04
Một sóng cơ truyền dọc theo trục Ox với phương trình $u = 2cos(40\pi t - \pi x)$ (mm). Biên
độ của sóng này là
\datcot
\bonpa
{\dung{2 mm.}}
{\sai{4 mm.}}
{\sai{$\pi$ mm.}}
{\sai {$40\pi$ mm.}}
\end{question}

\begin{question} %%05
 Khi nói về sóng cơ, phát biểu nào sau đây \textbf{sai}?
\datcot[2]
\bonpa
{\dung{Sóng cơ lan truyền được trong chân không.}}
{\sai{Sóng cơ lan truyền được trong chất rắn.}}
{\sai{Sóng cơ lan truyền được trong chất khí.}}
{\sai {Sóng cơ lan truyền được trong chất lỏng.}}
\end{question}

\begin{question} %%06
Một sóng cơ truyền dọc theo trục Ox có phương trình $u = A\cos(20\pi t - \pi x)$, với t tính bằng s.
Tần số của sóng này bằng
\datcot
\bonpa
{\sai{10$\pi$Hz.}}
{\dung{10Hz.}}
{\sai{20Hz.}}
{\sai {20$\pi$Hz.}}
\end{question}

\begin{question} %%07
Suất điện động cảm ứng do máy phát điện xoay chiều một pha tạo ra có biểu thức
$e=220\sqrt2\cos(100\pi t+0,5\pi)$ (V). Giá trị hiệu dụng của suất điện động này là
\datcot
\bonpa
{\sai{$220\sqrt2$ V.}}
{\sai{$110\sqrt2$ V.}}
{\sai {100 V.}}
{\dung{220 V.}}
\end{question}

\begin{question} %%08
Đặt điện áp $u =U_0\cos\omega t$ (với $U_0$ không đổi,  $\omega$ thay đổi được) vào hai đầu đoạn mạch mắc
nối tiếp gồm điện trở R, cuộn cảm thuần có độ tự cảm L và tụ điện có điện dung C. Khi  $\omega =  \omega_0$ thì
trong mạch có cộng hưởng. Tần số góc  $\omega_0$ là
\datcot
\bonpa
{\sai{$2\sqrt{LC}$.}}
{\sai{$\dfrac{2}{\sqrt{LC}}$.}}
{\dung{$\dfrac{1}{\sqrt{LC}}$.}}
{\sai {$\sqrt{LC}$.}}
\end{question}

\begin{question} %%09
 Đặt điện áp $u=U_0\cos 100 \pi t$ ($t$ tính bằng $s$) vào hai đầu một tụ điện có điện dung
$\dfrac{10^{-4}}{\pi}$ (F).
Dung kháng của tụ điện là
\datcot
\bonpa
{\sai{150 $\Omega$.}}
{\sai{250 $\Omega$.}}
{\sai {50 $\Omega$.}}
{\dung{100 $\Omega$.}}
\end{question}



\begin{question} %%10
Sóng điện từ
\datcot[4]
\bonpa
{\sai{ là sóng dọc và truyền được trong chân không.}}
{\dung{ là sóng ngang và truyền được trong chân không.}}
{\sai{ là sóng dọc và không truyền được trong chân không.}}
{\sai {là sóng ngang và không truyền được trong chân không.}}
\end{question}



\begin{question} %%11
Để xem các chương trình truyền hình phát sóng qua vệ tinh, người ta dùng anten thu
sóng trực tiếp từ vệ tinh, qua bộ xử lí tín hiệu rồi đưa đến màn hình. Sóng điện từ mà anten thu
trực tiếp từ vệ tinh thuộc loại
\datcot
\bonpa
{\sai{ sóng trung.}}
{\sai{ sóng ngắn.}}
{\sai {sóng dài.}}
{\dung{sóng cực ngắn.}}
\end{question}

\begin{question} %%12
Một mạch dao động điện từ gồm cuộn cảm thuần có độ tự cảm $10^{-5}$ H và tụ điện có điện
dung $2,5.10^{-6} $ F. Lấy $\pi = 3,14$. Chu kì dao động riêng của mạch là
\datcot
\bonpa
{\sai{$1,57.10^{-5}$ s.}}
{\sai{$1,57.10^{-10}$ s.}}
{\sai {$6,28.10^{-10}$ s.}}
{\dung{$3,14.10^{-5}$ s.}}
\end{question}

\begin{question} %%13
 Tia X \textbf{không} có ứng dụng nào sau đây?
\datcot[2]
\bonpa
{\sai{Chữa bệnh ung thư.}}
{\sai{Tìm bọt khí bên trong các vật bằng kim loại.}}
{\sai {Chiếu điện, chụp điện.}}
{\dung{Sấy khô, sưởi ấm.}}
\end{question}

\begin{question} %%14
Trong máy quang phổ lăng kính, lăng kính có tác dụng
\datcot[2]
\bonpa
{\sai{nhiễu xạ ánh sáng.}}
{\dung{tán sắc ánh sáng.}}
{\sai{giao thoa ánh sáng.}}
{\sai {tăng cường độ chùm sáng.}}
\end{question}


\begin{question} %%15
Một bức xạ khi truyền trong chân không có bước sóng là 0,60 $\mu$m, khi truyền trong thủy
tinh có bước sóng là  $\lambda$ . Biết chiết suất của thủy tinh đối với bức xạ là 1,5. Giá trị của  $\lambda$ là
\datcot
\bonpa
{\sai{900 nm.}}
{\sai{380 nm.}}
{\dung{400 nm.}}
{\sai {600 nm.}}
\end{question}

\begin{question} %%16
Theo thuyết lượng tử ánh sáng, phát biểu nào sau đây đúng?
\datcot[4]
\bonpa
{\dung{Ánh sáng đơn sắc có tần số càng lớn thì phôtôn ứng với ánh sáng đó có năng lượng càng lớn..}}
{\sai{Năng lượng của phôtôn giảm dần khi phôtôn ra xa dần nguồn sáng.}}
{\sai{Phôtôn tồn tại trong cả trạng thái đứng yên và trạng thái chuyển động.}}
{\sai {Năng lượng của các loại phôtôn đều bằng nhau.}}
\end{question}

\begin{question} %%17
Quang điện trở có nguyên tắc hoạt động dựa trên hiện tượng
\datcot
\bonpa
{\sai{quang - phát quang.}}
{\sai{quang điện ngoài.}}
{\dung{quang điện trong.}}
{\sai {nhiệt điện.}}
\end{question}
\end{vnmultiplechoice}
\end{document}

\begin{question} %%18

\datcot
\bonpa
{\sai{}}
{\sai{}}
{\sai{}}
{\dung{}}
\end{question}

\begin{question} %%19

\datcot
\bonpa
{\sai{}}
{\sai{}}
{\sai{}}
{\dung{}}
\end{question}

\begin{question} %%20

\datcot
\bonpa
{\sai{}}
{\sai{}}
{\sai{}}
{\dung{}}
\end{question}
%%%%%%%%%%%%%%%%%%%%%%%%%
\begin{question} %%21

\datcot
\bonpa
{\sai{}}
{\sai{}}
{\sai{}}
{\dung{}}
\end{question}

\begin{question} %%22

\datcot
\bonpa
{\sai{}}
{\sai{}}
{\sai{}}
{\dung{}}
\end{question}

\begin{question} %%23

\datcot
\bonpa
{\sai{}}
{\sai{}}
{\sai{}}
{\dung{}}
\end{question}

\begin{question} %%24

\datcot
\bonpa
{\sai{}}
{\sai{}}
{\sai{}}
{\dung{}}
\end{question}

\begin{question} %%25

\datcot
\bonpa
{\sai{}}
{\sai{}}
{\sai{}}
{\dung{}}
\end{question}

\begin{question} %%26

\datcot
\bonpa
{\sai{}}
{\sai{}}
{\sai{}}
{\dung{}}
\end{question}

\begin{question} %%27

\datcot
\bonpa
{\sai{}}
{\sai{}}
{\sai{}}
{\dung{}}
\end{question}

\begin{question} %%28

\datcot
\bonpa
{\sai{}}
{\sai{}}
{\sai{}}
{\dung{}}
\end{question}

\begin{question} %%29

\datcot
\bonpa
{\sai{}}
{\sai{}}
{\sai{}}
{\dung{}}
\end{question}

\begin{question} %%30

\datcot
\bonpa
{\sai{}}
{\sai{}}
{\sai{}}
{\dung{}}
\end{question}
%%%%%%%%%%%%%%%%%%%%%%%%%%
\begin{question} %%31

\datcot
\bonpa
{\sai{}}
{\sai{}}
{\sai{}}
{\dung{}}
\end{question}

\begin{question} %%32

\datcot
\bonpa
{\sai{}}
{\sai{}}
{\sai{}}
{\dung{}}
\end{question}

\begin{question} %%33

\datcot
\bonpa
{\sai{}}
{\sai{}}
{\sai{}}
{\dung{}}
\end{question}

\begin{question} %%34

\datcot
\bonpa
{\sai{}}
{\sai{}}
{\sai{}}
{\dung{}}
\end{question}

\begin{question} %%35

\datcot
\bonpa
{\sai{}}
{\sai{}}
{\sai{}}
{\dung{}}
\end{question}

\begin{question} %%36

\datcot
\bonpa
{\sai{}}
{\sai{}}
{\sai{}}
{\dung{}}
\end{question}

\begin{question} %%37

\datcot
\bonpa
{\sai{}}
{\sai{}}
{\sai{}}
{\dung{}}
\end{question}

\begin{question} %%38

\datcot
\bonpa
{\sai{}}
{\sai{}}
{\sai{}}
{\dung{}}
\end{question}

\begin{question} %%39

\datcot
\bonpa
{\sai{}}
{\sai{}}
{\sai{}}
{\dung{}}
\end{question}

\begin{question} %%40

\datcot
\bonpa
{\sai{}}
{\sai{}}
{\sai{}}
{\dung{}}
\end{question}


\begin{examclosing}
\centerline{-- HẾT --}
\end{examclosing}
 \end{vnmultiplechoice}
\end{document}
[scribd id=327589426 key=key-oJpAkf8g7Gy0j9JNOeAs mode=scroll]
