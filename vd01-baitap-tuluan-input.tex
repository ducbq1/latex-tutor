% Tệp mẫu làm đề thi tự luận dựa vào gói lệnh lamdethi.sty
% Tác giả: Nguyên Hữu Điển
% Khoa Toán Cơ Tin học, ĐHKHTN HN, ĐHQGHN
% 334, Nguyễn Trãi, Thanh Xuân, Hà Nội
% huudien@vnu.edu.vn
% Ngày 26/12/2009
\documentclass[12pt]{article}
\usepackage{amsmath,amsxtra,amssymb,latexsym, amscd,amsthm}
\usepackage{graphicx}
\usepackage{picinpar}
\usepackage[utf8]{vietnam}
\usepackage{longtable}%
\usepackage{multicol}%
\usepackage{color}
\usepackage{fancybox}
\usepackage{lastpage}
%%%%%%%%%%%%%%%%%%%%%%%
\usepackage{enumerate}
 \usepackage{shortlst}
\usepackage{tikz}
\usetikzlibrary{arrows}
\usepackage{tkz-tab}
%%%%%%%%%%%%%%%%%%%%%
\usepackage{mathptmx} 
% \usepackage{mathpazo} 
\voffset=-3cm
% \hoffset=-2cm
\textheight 24truecm 
\textwidth 17truecm 
\usepackage[baitap]{dethi}
\tentruong{ĐẠI HỌC KHOA HỌC TỰ NHIÊN}
\tenkhoa{Khoa Toán - Cơ -Tin học}
\loaidethi{Đề gồm có \pageref{LastPage} trang}%{ĐỀ THI LẠI}%%{ĐỀ CHÍNH THỨC}
\tenkythi{ĐỀ THI GIỮA KỲ NĂM HỌC 2016-2017}
\tenmonhoc{Môn: Toán học tính toán}
\madethi{100}
\thoigian{\underline{Thời gian làm bài: 90 phút, không kể thời gian phát đề}}   
\hovaten{Họ và tên}         %Nếu không muốn có dòng này không gõ lệnh
\tenlop{Tên lớp}         %Nếu không muốn có dòng này không gõ lệnh
\sobaodanh{Số báo danh}  %Nếu không muốn có dòng này không gõ lệnh
% \cornersize*{3.6mm}
%\daungoac{\Ovalbox}{}
%\daungoac{(}{)}%%{[}{]}%Dấu quanh phương án trả lời: {(}{)};{}{.};{}{)}
% \chuphuongan{\small\bfseries\Alph}
\mauchu{blue}
\PSNrandseed{\time}
% \coloigiai
\renewcommand{\solutionname}{\textbf{Lời giải. }}

\PSNrandseed{\time}
\usepackage{centerpage}
\setlength{\baselineskip}{16truept}
\begin{document}
\soanthao
\thispagestyle{empty}
\lamtieude
\vspace*{1cm}
% \loadrandomproblems[dttuluan]{5}{vd01-cauhoi-tuluan}

\indebailoigiai

%MẪU LÀM CÂU HỎI TỰ LUẬN
%Dùng với gói lệnh lamdethi.sty
%Dùng Vietex 2.8. với phông mã Unicode
%Người soạn : Nguyễn Hữu Điển, ĐHKHTN, ĐHQG HN
%Mail: huudien@vnu.edu.vn, CQ: (84 - 4) 557 2869
%NR: (84 - 4) 641 8848, DĐ: 0989061951
%Ngày 26/12/2009
%%%%%%%%%%%%%%%%%%%%%%%%%

\baituluan{logic:1}{%Câu hỏi 1
a) Cho $P_1, P_2$ và $Q$ là những mệnh đề. Hãy chỉ ra sự tương đương sau đây
$$(P_1\vee P_2)\rightarrow Q \equiv (P_1\rightarrow Q)\wedge (P_1\rightarrow Q).$$

b) Sử dụng sự tương đương trên chứng minh mệnh đề sau đây: " Nếu $n$ không chia hết cho 3 thì $n^2$ không chia hết cho 3".
}{%Trả lời
a) Chứng minh bằng lập bảng giá trị.

b) Gọi $P$ là mệnh đề "$n$ không chia hết cho 3" và $Q$ là mệnh đề "$n^2$
không chia hết cho 3". Khi đó, $P$ tương đương với $P_1 \vee P_2$. Trong đó:

$P_1 =$ " $n$ mod $3 =1$"

$P_2 =$ " $n$ mod $3 =2$"

Vậy, để chứng minh $P \rightarrow  Q$ là đúng, có thể chứng minh rằng:
$(P_1 \vee P_2) \rightarrow  Q$ hay là $(P_1 \rightarrow  Q ) \wedge ( P_2\rightarrow  Q)$
Giả sử $P_1$ là đúng. Ta có, $n$ mod $3 = 1$. Đặt $n = 3k + 1$
( $k$ là số nguyên nào đó).

Suy ra
$n^2 = ( 3k+1)^2 = 9k^2 + 6k + 1 = 3(3k^2 + 2k) + 1$ không chia chẳn cho 3.

Do đó, $P_1 \rightarrow  Q$ là đúng.

Tương tự, giả sử $P_2$ là đúng. Ta có, $n$ mod $3 = 2$. Đặt $n = 3k + 2$ ( $k$ là số
nguyên nào đó).

Suy ra $n^2 = ( 3k+2)^2 = 9k^2 + 12k + 4 = 3(3k^2 + 4k + 1) + 1$ không chia chẳn
cho 3.

Do đó, $P2 \rightarrow  Q$ là đúng.

Do $P_1 \rightarrow  Q$ là đúng và $P_2 \rightarrow  Q$ là đúng, hay là $(P_1 \rightarrow  Q ) \wedge ( P_2\rightarrow  Q)$.

Vậy $(P_1 \vee P_2) \rightarrow  Q$.
}%Hết câu hỏi 1


\baituluan{logic:2}{%Câu hỏi 2
a) Phát biểu định nghĩa 4 phần của lý thuyết tiên đề $L$.

b) Cho công thức $A, B, C$ tùy ý. Chứng minh rằng
$$((A\rightarrow B)\wedge (B\rightarrow C))\vdash A\rightarrow C. $$
}{%Trả lời
a) Định nghĩa: Lý thuyết tiên đề $L$ bao gồm:

(1) Các ký hiệu của $L$:
\begin{itemize}
\item $\neg, \rightarrow$ được gọi là hai phép toán nguyên thủy
\item Các dấu ngoặc (,)
\item Các chữ cái Latinh $A, B, C$, ... và các chữ cái la tinh có chỉ số $A_1, B_1, C_1, $ ... các ký hiệu này được gọi là các mệnh đề.
\end{itemize}

(2) Công thức được xây dựng bằng đệ quy:
\begin{description}
\item{(a)} Tất cả các biến mệnh đề là công thức. 
\item{(b)} Nếu $A$ và $B$ là công thức thì $(\neg A)$, $(A\rightarrow B)$ cũng là công thức. 
\item{(c)} Một biểu thức được lập nên từ cơ sở (a) và (b) cũng là một công thức.
\end{description}

(3) Các tiên đề: Đối với các công thức $A, B, C$ tùy ý

A1. $(A\rightarrow(B\rightarrow A))$;

A2. $(A\rightarrow (B\rightarrow C))\rightarrow ((A\rightarrow B)\rightarrow (A\rightarrow C))$;

A3. $(\neg B\rightarrow\neg A)\rightarrow((\neg B\rightarrow A)\rightarrow B)$.

(4) Quy tắc dẫn xuất Modus Ponens: Nếu $A$ và $A\rightarrow B$ thì $B$.

b) Ta xây dựng dẫn xuất sau đây:

1. $A\rightarrow B$ đúng (giả thiết)

2. $B\rightarrow C$ đúng (giả thiết)

3. $A$ đúng (giả thiết)

4. $B$ đúng (1, 3, MP)

5. $C$ đúng (2, 4, MP)

Từ 1 - 5 ta có
$$A\rightarrow B, B\rightarrow C, A\vdash C.$$

Theo định lý suy diễn ta có
$$A\rightarrow B, B\rightarrow C \vdash A\rightarrow C.$$
}%Hết câu hỏi 2

\baituluan{logic:3}{%Câu hỏi 3
Cho công thức
$$(A\rightarrow B)\rightarrow ((B\rightarrow C)\rightarrow((A\vee B)\rightarrow C)).$$
Hãy thực hiện

a) Đưa công thức về dạng chuẩn tắc hội.

b) Chỉ ra công thức là hằng đúng.
}{%Trả lời
a) Dạng chuẩn tắc hội
\begin{align*}
&(A\rightarrow B)\rightarrow ((B\rightarrow C)\rightarrow((A\vee B)\rightarrow C))\\ 
\equiv&\overline{\bar A \vee C}\vee\overline{\bar B \vee C}\vee\overline{A \vee B}\vee C\\ 
\equiv&(A\wedge \bar C)\vee(B\wedge \bar C)\vee(\bar A\wedge \bar B)\vee C\\ 
\equiv&(A\vee(B\wedge \bar C)\vee (\bar A \wedge \bar B)\vee C)
\wedge (\bar C\vee(B\wedge \bar C)\vee (\bar A \wedge \bar B)\vee C)\\ 
\equiv&(A\vee(B\wedge \bar C)\vee C) \vee (\bar A \wedge \bar B)\\ 
\equiv&(A\vee(B\wedge \bar C)\vee C\vee \bar A)\wedge(A\vee(B\wedge \bar C)\vee C\vee \bar B)\\ 
\equiv&(A\vee C\vee \bar B \vee (B\wedge \bar C))\\ 
\equiv&(A\vee C\vee \bar B \vee B)\wedge (A\vee C\vee \bar B \vee\bar C)\\
\equiv &1. 
\end{align*}

b) Với luật kết hợp suy ra công thức là hằng đúng.
}%Hết câu hỏi 3

\baituluan{logic:4}{%Câu hỏi 4
a) Phát biểu định nghĩa thế nào là hạng từ và công thức tân từ trong lý thuyết hệ tân từ.

b) Cho vị từ ba biến $P(x, y, z)\equiv "x.y=z"$ trên trường số thực. Xác định giá trị chân lý của mệnh đề:
$(\forall x) (\forall y) (\exists z) P(x, y, z)$ và 
$(\exists z) (\forall x) (\forall y)  P(x, y, z)$. Diễn giải mệnh đề thành câu nói thông thường.
}{%Trả lời
a) Định nghĩa 1. (a) Tất cả các biến và hằng các thể đều là lượng tử.

(b) Nếu $f_{i}^n$ là một biến hàm và $t_1, i_2, ...t_n$ là các hạng tử, thì $f_i^n(t_1, t_2, ...,t_n)$ là một hạng tử.

(c) Một biểu thức là một hạng tử nếu nó được lập nên bởi (a) và (b).

Định nghĩa 2. (a) Mỗi công thức sơ cấp là một công thức

(b) Nếu $A$ và $B$ là công thức, và $y$ là một biến thì $(\neg A)$, $(A\rightarrow B)$, $(\forall y A)$ là công thức.

(c) Nếu biểu thức được lập nên bởi (a) và (b) ở trên thì nó cũng là công thức.

b) $(\forall x) (\forall y) (\exists z) P(x, y, z)$ là mệnh đề đúng vì khi đó đặt $z=(x.y)$.
 
$(\exists z) (\forall x) (\forall y)  P(x, y, z)$ là mệnh đề sai vì không thể có một $z$ mà với mọi $x, y$ đều có $z=x.y$  
}%Hết câu hỏi 4

\baituluan{logic:5}{%Câu hỏi 5
Trong môn học giải tích toán học người ta định nghĩa hàm liên tục như sau:
"Hàm $f(x)$ được gọi là hàm liên tục tại $x_0\in D$ nếu cho trước một số $\epsilon>0$ tùy ý thì ta có được một số $\delta>0$ tương ứng sao cho với mọi $x\in D$ thỏa mãn $|x-x_0|<\delta$ thì $|f(x)-f(x_0)|<\epsilon$".

a) Hãy viết lại định nghĩa theo các ký hiệu của hệ toán tân từ.

b) Hãy lập mệnh đề phủ định cho định nghĩa trên (nghĩa là hàm không liên tục tại điểm $x_0$)
}{%Trả lời
a) $(\forall \epsilon)(\exists \delta) (\forall x\in D): |x-x_0|<\delta \rightarrow |f(x)-f(x_0)|<\epsilon$.

b) $(\exists \epsilon) (\forall \delta) (\exists x\in D) : |x-x_0|<\delta \rightarrow |f(x)-f(x_0)|>\epsilon$
}%Hết câu hỏi 5


\baituluan{logic:6}{%Câu hỏi 5
Trong môn học giải tích toán học người ta định nghĩa hàm liên tục như sau:
"Hàm $f(x)$ được gọi là hàm liên tục tại $x_0\in D$ nếu cho trước một số $\epsilon>0$ tùy ý thì ta có được một số $\delta>0$ tương ứng sao cho với mọi $x\in D$ thỏa mãn $|x-x_0|<\delta$ thì $|f(x)-f(x_0)|<\epsilon$".

a) Hãy viết lại định nghĩa theo các ký hiệu của hệ toán tân từ.

b) Hãy lập mệnh đề phủ định cho định nghĩa trên (nghĩa là hàm không liên tục tại điểm $x_0$)
}{%Trả lời
a) $(\forall \epsilon)(\exists \delta) (\forall x\in D): |x-x_0|<\delta \rightarrow |f(x)-f(x_0)|<\epsilon$.

b) $(\exists \epsilon) (\forall \delta) (\exists x\in D) : |x-x_0|<\delta \rightarrow |f(x)-f(x_0)|>\epsilon$
}%Hết câu hỏi 5

% \begin{enumerate}[]
% \item \textbf{Câu 01.} \useproblem{logic:5}
% \end{enumerate}
% \begin{cauhoi}
% \foreachproblem[dttuluan]{\item\thisproblemo}
% \end{cauhoi}

\vspace*{1cm}
{\bf Chú ý:} Sinh viên không được mang sách giáo trình và vở ghi chép vào phòng thi.

\end{document}
\newpage
%\hideproblems
\setcounter{page}{1}
\lamtieude
\indebailoigiai
\begin{center}
{\bf ĐỀ BÀI VÀ ĐÁP ÁN }
\end{center}

\begin{cauhoi}
\foreachproblem[dttuluan]{\item\thisproblemo}
\end{cauhoi}

\newpage
%\hideproblems
\setcounter{page}{1}
\lamtieude
\indapanloigiai
\begin{center}
{\bf ĐÁP ÁN }
\end{center}

\begin{cauhoi}
\foreachproblem[dttuluan]{\item\thisproblemo}
\end{cauhoi}

\end{document}
