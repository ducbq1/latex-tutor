%-[Ví dụ TIẾNG PHÁP, Khối D]: trong gói đề thi Phiên bản 1.0
%Tác giả Nguyễn Hữu Điển: 15/10/2007
%ĐHKHTN Hà Nội, ĐHQG HN 
%%%%%%%%%%%%%%%%%%%
\documentclass[11pt]{article}
\usepackage{amsmath,amsxtra,latexsym, amssymb, amscd}
\usepackage[utf8]{vietnam}
\usepackage{color}
\usepackage{graphicx}
\usepackage{picinpar}
\usepackage{enumerate}
\usepackage{multicol}
\usepackage{shortlst}
\usepackage[baithi]{dethi} %Gói lệnh cho đề thi Việt Nam
 \usepackage{lastpage}
% \usepackage{fancybox}
% \cornersize*{3.6mm}
\Fullpages %Định dạng trang đề thi
\ContinuousNumbering %Đánh số liên tục các bài thi
%\OneKey %Lệnh chỉ in ra 1 bản đáp án
%\NoKey %Lệnh không in ra phần đáp án    
\NumberOfVersions{20} %20 là số bài thi khác nhau được in ra
\SectionPrefix{\relax }%\bf Phần \Roman{sectionindex}. \space}

\tentruong{BỘ GIÁO DỤC VÀ ĐÀO TẠO}
\tenkhoa{ĐỀ MINH HỌA}
\loaidethi{Đề gồm có 06 trang}%{ĐỀ THI LẠI}%%{ĐỀ CHÍNH THỨC}
\tenkythi{KÌ THI TRUNG HỌC PHỔ THÔNG QUỐC GIA NĂM 2017}
\tenmonhoc{Môn: Toán}
\madethi{100}
\thoigian{\underline{Thời gian làm bài: 90 phút, không kể thời gian phát đề}}
%\chuy{Được mở vở, cán bộ coi thi không giải thích gì thêm}
\tieudetracnghiem

% \tieudedapan
\tieudeduoi     
\daungoac{\Ovalbox}{}%Dấu quanh phương án trả lời: {(}{)};{}{.};{}{)}
%\chuphuongan{\alph}%Ký tự cho các phương án
%\chuphuongan{\arabic}%\Roman%\roman%kể cả số cho các phương án
\chucauhoi{\textsf{Question}} %Chữ trước các số câu hỏi
%\mauchu{red}

\setlength{\shortitemwidth}{3.5truecm}
\def\v#1{\overrightarrow{#1}}
\def\ktrang{\makebox[1.5cm]{\hrulefill}\ }
%\NoRearrange
\setlength{\leftmargin}{0truept}
 %\setlength{\labelwidth}{12truept}
\setlength{\rightmargin}{0pc}
\setlength{\parskip}{0pc}
 \setlength{\topsep}{0pc}
 \setlength{\partopsep}{0pc}
\setlength{\baselineskip}{12truept}
 %\setlength{\runitemsep}{0truept}
%]-%%%

\tieudetracnghiem
%\tieudethiviet
\tieudedapan
%\tieudetren
\tieudeduoi
\daungoac{}{.}%Dấu quanh phương án trả lời: {(}{)};{}{.};{}{)}
%\chuphuongan{\alph}%Ký tự cho các phương án
%\chuphuongan{\arabic}%\Roman%\roman%kể cả số cho các phương án
% \chucauhoi{Câu} %Chữ trước các số câu hỏi
\mauchu{red}

\setlength{\baselineskip}{12truept}
\def\v#1{\overrightarrow{#1}}
% \NoRearrange
\usepackage{fancybox}
\cornersize*{3.6mm}
\graphicspath{{hinh-cauhoi/} }
\khoanh{\Ovalbox}
\hovaten{Họ và tên}
% \tenlop{Tên lớp}
\sobaodanh{Số báo danh}
% \ketqua{}
% \giamkhao{}
\motphieuthi
%  \nhieuphieuthi
\ShortKey
%  \coloigiai\chohienpafalse
\beforesectsep =0pt
\beforeinstsep=0pt
% \aftersectsep=0pt
% \afterinstsep=0pt
\usepackage{centerpage}
\begin{document}
\sf
% 
%   \liencau

\begin{multiplechoice}[rearrange=yes, keycolumns=2]%

\examvspace*{1cm}
\textit{\textbf{Mark the letter A, B, C, or D on your answer sheet to indicate the word whose underlined part
differs from the other three in pronunciation in each of the following questions.}}

\begin{question}%1
\datcot
\bonpat
{\sai{wanted}}
{\dung{stopped} }
{\sai{decided} }
{\sai{hated}}
\end{question}

\begin{question}%2
\datcot
\bonpat
{\dung{century} }
{\sai{culture}}
{\sai{secure} }
{\sai{applicant}}
\end{question}
%\end{block}
\end{multiplechoice}

\begin{multiplechoice}[rearrange=yes, keycolumns=2]%
\examvspace*{0.7cm}
\textit{\textbf{Mark the letter A, B, C, or D on your answer sheet to indicate the word that differs from the other
three in the position of primary stress in each of the following questions.}}

\begin{question}
\datcot
\bonpat
{\sai{offer}}
{\dung{canoe} }
{\sai{country}}
{\sai{standard}}
\end{question}

\begin{question}
\datcot
\bonpat
{\sai{pollution}}
{\sai{computer}}
{\dung{currency} }
{\sai{allowance}}
\end{question}

\end{multiplechoice}


\begin{multiplechoice}[rearrange=yes, keycolumns=2]%
\examvspace*{0.7cm}
\textit{\textbf{Mark the letter A, B, C, or D on your answer sheet to indicate the underlined part that needs
correction in each of the following questions.}}

\begin{question}
\saih{Measles} \dungh{are} an \saih{infectious} disease that causes fever \saih{and} small red spots.
\datcot
\bonpah
{\sai{Measles}}
{\dung{are}}
{\sai{infectious}}
{\sai{and}}
\end{question}

\begin{question}
He \saih{passed} the exams \saih{with} high scores, \dungh{that} made his parents \saih{happy}.
\datcot
\bonpah
{\sai{passed}}
{\sai{with}}
{\dung{that}}
{\sai{happy}}
\end{question}

\begin{question}
For \saih{such} a demanding job, you \saih{will need} \saih{qualifications}, soft skills and \dungh{having full commitment}.
\datcot
\bonpah
{\sai{such}}
{\sai{will need}}
{\sai{qualifications}}
{\dung{having full commitment} }
\end{question}
\end{multiplechoice}


\begin{multiplechoice}[rearrange=yes, keycolumns=2]%
\examvspace*{0.7cm}
\textit{\textbf{Mark the letter A, B, C, or D on your answer sheet to indicate the correct answer to each of the
following questions.}}

\begin{question}%%8
I haven‘t met him again since we \ktrang  school ten years ago.
\datcot
\bonpat
{\sai{have left}}
{\sai{leave}}
{\dung{left} }
{\sai{had left}}
\end{question}

\begin{question} %%%9
A recent survey has shown that \ktrang increasing number of men are willing to share the
housework with their wives.
\datcot
\bonpat
{\dung{a} }
{\sai{an}}
{\sai{the}}
{\sai{some}}
\end{question}

\begin{question} %%%10
The more demanding the job is, \ktrang  I like it.
\datcot
\bonpa
{\sai{more}}
{\sai{most}}
{\dung{the more} }
{\sai{the most}}
\end{question}

\begin{question} %%%11
John wanted to know \ktrang  in my family.
\datcot[2]
\bonpa
{\sai{there were how many people}}
{\sai{how many people were there}}
{\sai{were there how many people }}
{\dung{how many people there were} }
\end{question}

\begin{question} %%%12
 Richard, my neighbor, \ktrang  in World War II.
\datcot[2]
\bonpa
{\sai{says to fight}}
{\sai{says to have fought}}
{\sai{is said to fight}}
{\dung{is said to have fought}}
\end{question}

\begin{question} %%%13
Students are \ktrang  less pressure as a result of changes in testing procedures.
\datcot
\bonpa
{\dung{under} }
{\sai{above}}
{\sai{upon}}
{\sai{out of}}
\end{question}

\begin{question} %%%14
Tom is getting ever keener on doing research on \ktrang .
\datcot
\bonpa
{\dung{biology} }
{\sai{biological}}
{\sai{biologist}}
{\sai{biologically}}
\end{question}

\begin{question} %%15
Many people and organizations have been making every possible effort in order to save \ktrang 
species.
\datcot
\bonpa
{\dung{endangered}}
{\sai{dangerous}}
{\sai{fearful}}
{\sai{threatening}}
\end{question}

\begin{question} %%16
A number of young teachers nowadays \ktrang  themselves to teaching disadvantaged children.
\datcot
\bonpa
{\sai{offer}}
{\sai{stick}}
{\sai{give}}
{\dung{devote}}
\end{question}

\begin{question} %%17
Whistling or clapping hands to get someone‘s attention is considered \ktrang  and even rude in
some circumstances.
\datcot
\bonpa
{\sai{suitable}}
{\sai{unnecessary}}
{\sai{appropriate}}
{\dung{impolite}}
\end{question}

\begin{question} %%18
 "Sorry for being late. I was \ktrang  in the traffic for more than an hour."
\datcot
\bonpa
{\sai{carried on}}
{\dung{held up}}
{\sai{put off}}
{\sai{taken after}}
\end{question}

\begin{question} %%19
She was tired and couldn‘t keep \ktrang  the group.
\datcot
\bonpa
{\dung{up with}}
{\sai{up against}}
{\sai{on to}}
{\sai{out of}}
\end{question}
\end{multiplechoice}

\begin{multiplechoice}[rearrange=yes, keycolumns=2]%
\examvspace*{0.7cm}
\textit{\textbf{Mark the letter A, B, C, or D on your answer sheet to indicate the most suitable response to
complete each of the following exchanges.}}
\begin{question} %%20
Two friends Diana and Anne are talking about Anne‘s new blouse.

- Diana: "That blouse suits you perfectly, Anne."

- Anne: "\ktrang"
\datcot
\bonpa
{\sai{Never mind.}}
{\sai{Don‘t mention it}}
{\dung{Thank you. }}
{\sai{You‘re welcome.}}
\end{question}

\begin{question} %%21
Mary is talking to a porter in the hotel lobby.

- Porter: "Shall I help you with your suitcase?"

- Mary: "\ktrang"
\datcot
\bonpa
{\sai{Not a chance. }}
{\dung{That‘s very kind of you.}}
{\sai{I can‘t agree more.}}
{\sai{ What a pity!}}
\end{question}
\end{multiplechoice}


\begin{multiplechoice}[rearrange=yes, keycolumns=2]%
\examvspace*{0.7cm}
\textit{\textbf{Mark the letter A, B, C, or D on your answer sheet to indicate the word(s) CLOSEST in meaning to
the underlined word(s) in each of the following questions.}}
\begin{question} %%22
Students are expected to always \underline{adhere to} school regulations.
\datcot
\bonpa
{\sai{question}}
{\sai{violate}}
{\sai{disregard}}
{\dung{follow}}
\end{question}

\begin{question} %%23
A number of programs have been initiated to provide food and shelter for the underprivileged in
the remote areas of the country.
\datcot
\bonpa
{\sai{rich citizens}}
{\sai{active members}}
{\dung{poor inhabitants}}
{\sai{enthusiastic people}}
\end{question}
\end{multiplechoice}

\begin{multiplechoice}[rearrange=yes, keycolumns=2]%
\examvspace*{0.7cm}
\textit{\textbf{ }}
\begin{question} %%24
Drivers are advised to get enough petrol because filling stations are \underline{few and far between} on the
highway.
\datcot
\bonpa
{\dung{easy to find}}
{\sai{difficult to access}}
{\sai{unlikely to happen}}
{\sai{ impossible to reach}}
\end{question}

\begin{question} %%25
We managed to get to school \underline{in time} despite the heavy rain.
\datcot[2]
\bonpa
{\sai{earlier than a particular moment}}
{\dung{later than expected }}
{\sai{early enough to do something}}
{\sai{as long as expected}}
\end{question}
\end{multiplechoice}


\begin{multiplechoice}[rearrange=yes, keycolumns=2]%
\examvspace*{0.7cm}
\textit{\textbf{Mark the letter A, B, C, or D on your answer sheet to indicate the sentence that is closest in
meaning to each of the following questions. }}
\begin{question} %%26
I‘m sure Luisa was very disappointed when she failed the exam.
\datcot[4]
\bonpa
{\sai{Luisa must be very disappointed when she failed the exam.}}
{\dung{Luisa must have been very disappointed when she failed the exam.}}
{\sai{Luisa may be very disappointed when she failed the exam.}}
{\sai{Luisa could have been very disappointed when she failed the exam.}}
\end{question}

\begin{question} %%27
 "You had better see a doctor if the sore throat does not clear up," she said to me.
\datcot[4]
\bonpa
{\sai{She reminded me of seeing a doctor if the sore throat did not clear up.}}
{\sai{She ordered me to see a doctor if the sore throat did not clear up.}}
{\sai{She insisted that I see a doctor unless the sore throat did not clear up.}}
{\dung{She suggested that I see a doctor if the sore throat did not clear up.}}
\end{question}

\begin{question} %%28
Without her teacher‘s advice, she would never have written such a good essay.
\datcot[4]
\bonpa
{\sai{Her teacher advised him and she didn‘t write a good essay.}}
{\sai{ Her teacher didn‘t advise her and she didn‘t write a good essay.}}
{\dung{She wrote a good essay as her teacher gave her some advice.}}
{\sai{If her teacher didn‘t advise her, she wouldn‘t write such a good essay.}}
\end{question}
\end{multiplechoice}


\begin{multiplechoice}[rearrange=yes, keycolumns=2]%
\examvspace*{0.7cm}
\textit{\textbf{ Mark the letter A, B, C, or D on your answer sheet to indicate the sentence that best combines
each pair of sentences in the following questions.}}
\begin{question} %%29
She tried very hard to pass the driving test. She could hardly pass it.
\datcot[4]
\bonpa
{\sai{Although she didn‘t try hard to pass the driving test, she could pass it.}}
{\sai{Despite being able to pass the driving test, she didn‘t pass it.}}
{\dung{No matter how hard she tried, she could hardly pass the driving test.}}
{\sai{ She tried very hard, so she passed the driving test satisfactorily.}}
\end{question}

\begin{question} %%30
We didn‘t want to spend a lot of money. We stayed in a cheap hotel.
\datcot[4]
\bonpa
{\dung{Rather than spending a lot of money, we stayed in a cheap hotel.}}
{\sai{In spite of spending a lot of money, we stayed in a cheap hotel.}}
{\sai{We stayed in a cheap hotel, but we had to spend a lot of money.}}
{\sai{We didn‘t stay in a cheap hotel as we had a lot of money to spend.}}
\end{question}
\end{multiplechoice}


\begin{multiplechoice}[rearrange=yes, keycolumns=2]%
\examvspace*{0.7cm}
\textit{\textbf{Read the following passage and mark the letter A, B, C, or D on your answer sheet to indicate the
correct word or phrase that best fits each of the numbered blanks from 31 to 35.}}
\begin{center}
WAYS TO IMPROVE YOUR MEMORY
\end{center}
A good memory is often seen as something that comes naturally, and a bad memory as something that
cannot be changed, but actually (31)\ktrang  is a lot that you can do to improve your memory.
We all remember the things we are interested in and forget the ones that bore us. This no doubt explains
the reason (32)\ktrang  schoolboys remember football results effortlessly but struggle with dates from their
history lessons! Take an active interest in what you want to remember, and focus on it (33)\ktrang . One way
to 'make‘ yourself more interested is to ask questions - the more the better!

Physical exercise is also important for your memory, because it increases your heart (34)\ktrang  and sends
more oxygen to your brain, and that makes your memory work better. Exercise also reduces stress, which is
very bad for the memory.

The old saying that "eating fish makes you brainy" may be true after all. Scientists have discovered that the
fats (35)\ktrang in fish like tuna, sardines and salmon -- as well as in olive oil -- help to improve the memory.
Vitamin-rich fruits such as oranges, strawberries and red grapes are all good 'brain food‘, too.

\hfill(Source: \textit{"New Cutting Edge" ,  Cunningham, S. \& Moor. 2010. Harlow: Longman}) 

\begin{question} %%31
\datcot
\bonpat
{\dung{there}}
{\sai{it}}
{\sai{that}}
{\sai{this}}
\end{question}

\begin{question} %%32
\datcot
\bonpat
{\dung{why}}
{\sai{what}}
{\sai{how}}
{\sai{which}}
\end{question}

\begin{question} %%33
\datcot
\bonpat
{\sai{hardly}}
{\sai{slightly}}
{\dung{consciously}}
{\sai{easily}}
\end{question}

\begin{question} %%34
\datcot
\bonpat
{\sai{degree}}
{\sai{level}}
{\dung{rate}}
{\sai{grade}}
\end{question}

\begin{question} %%35
\datcot
\bonpat
{\sai{made}}
{\sai{existed}}
{\sai{founded}}
{\dung{found}}
\end{question}
\end{multiplechoice}


\begin{multiplechoice}[rearrange=yes, keycolumns=2]%
\examvspace*{0.7cm}
\textit{\textbf{Read the following passage and mark the letter A, B, C, or D on your answer sheet to indicate the
correct answer to each of the questions from 36 to 42.}}

It used to be that people would drink coffee or tea in the morning to pick them up and get them going for
the day. Then cola drinks hit the market. With lots of caffeine and sugar, these beverages soon became the
pick-me-up of choice for many adults and teenagers. Now drink companies are putting out so-called "energy
drinks." These beverages have the specific aim of giving tired consumers more energy.

One example of a popular energy drink is Red Bull. The company that puts out this beverage has stated in
interviews that Red Bull is not a thirst quencher. Nor is \underline{\textbf{it}} meant to be a fluid replacement drink for athletes.
Instead, the beverage is meant to revitalize a tired consumer's body and mind. In order to do this, the makers
of Red Bull, and other energy drinks, typically add vitamins and certain chemicals to their beverages. The added
chemicals are like chemicals that the body naturally produces for energy. The vitamins, chemicals, caffeine, and
sugar found in these beverages all seem like a sure bet to give a person energy.

Health professionals are not so sure, though. For one thing, there is not enough evidence to show that all of
the vitamins added to energy drinks actually raise a person's energy level. Another problem is that there are so
many things in the beverages. Nobody knows for sure how all of the ingredients in energy drinks work together.

Dr. Brent Bauer, one of the directors at the Mayo Clinic in the US, cautions people about believing all the
claims energy drinks make. He says, "It is \textbf{plausible} if you put all these things together, you will get a good
result." However, Dr. Bauer adds the mix of ingredients could also have a negative impact on the body. "We
just don't know at this point", he says.

(Source: \textit{"Reading Challenge 2", Casey Malarcher \& Andrea Janzen, Compass Publishing)}

\begin{question} %%36
The beverages mentioned in the first paragraph aim to give consumers \ktrang .
\datcot
\bonpa
{\sai{caffeine}}
{\sai{sugar}}
{\dung{more energy}}
{\sai{more choices}}
\end{question}

\begin{question} %%37
The word \textbf{“it”} in the second paragraph refers to \ktrang .
\datcot
\bonpa
{\sai{one example}}
{\sai{the company }}
{\dung{Red Bull}}
{\sai{thirst quencher}}
\end{question}

\begin{question} %%38
According to the passage, what makes it difficult for researchers to know if an energy drink
gives people energy?
\datcot[2]
\bonpa
{\sai{Natural chemicals in a person‘s body }}
{\sai{The average age of the consumer}}
{\sai{The number of beverage makers}}
{\dung{The mixture of various ingredients}}
\end{question}

\begin{question} %%39
The word "\textbf{plausible}" in the passage is closest in meaning to \ktrang .
\datcot
\bonpa
{\sai{impossible}}
{\dung{reasonable}}
{\sai{typical}}
{\sai{unlikely}}
\end{question}

\begin{question} %%40
What has Dr. Bauer probably researched?
\datcot[2]
\bonpa
{\sai{Countries where Red Bull is popular  }}
{\sai{Energy drinks for teenage athletes}}
{\sai{Habits of healthy and unhealthy adults}}
{\dung{Vitamins and chemicals in the body}}
\end{question}

\begin{question} %%41
Which of the following is NOT true according to the passage?
\datcot[4]
\bonpa
{\sai{Bauer does not seem to believe the claims of energy drink makers.}}
{\dung{Colas have been on the market longer than energy drinks.}}
{\sai{It has been scientifically proved that energy drinks work.}}
{\sai{The makers of Red Bull say that it can revitalize a person.}}
\end{question}

\begin{question} %%42
What is the main idea of this passage?
\datcot[4]
\bonpa
{\sai{Caffeine is bad for people to drink.}}
{\dung{It is uncertain whether energy drinks are healthy.}}
{\sai{Red Bull is the best energy drink.}}
{\sai{Teenagers should not choose energy drinks.}}
\end{question}

\end{multiplechoice}


\begin{multiplechoice}[rearrange=yes, keycolumns=2]%
\examvspace*{0.7cm}
\textit{\textbf{Read the following passage and mark the letter A, B, C, or D on your answer sheet to indicate the
correct answer to each of the questions from 43 to 50 }}

What is `extreme‘ weather? Why are people talking about it these days? `Extreme‘ weather is an unusual
weather event such as rainfall, a drought or a heat wave in the wrong place or at the wrong time. In theory,
they are very rare. But these days, our TV screens are constantly showing such extreme weather events. Take
just three news stories from 2010: 28 centimetres of rain fell on Rio de Janeiro in 24 hours, Nashville, USA, had
33 centimetres of rain in two days and there was record rainfall in Pakistan.

The effects of this kind of rainfall are dramatic and \textbf{\underline{lethal}}. In Rio de Janeiro, landslides followed, burying
hundreds of people. In Pakistan, the floods affected 20 million people. Meanwhile, other parts of the world suffer
devastating droughts. Australia, Russia and East Africa have been hit in the last ten years. And then there are
unexpected heat waves, such as in 2003 in Europe. That summer, 35,000 deaths were said to be heat-related.

So, what is happening to our weather? Are these extreme events part of a natural cycle? Or are they caused by
human activity and its effects on the Earth‘s climate? Peter Miller says it‘s probably a mixture of both of these
things. On the one hand, the most important influences on weather events are natural cycles in the climate. Two
of the most famous weather cycles, El Niño and La Niña, originate in the Pacific Ocean. The heat from the warm
ocean rises high into the atmosphere and affects weather all around the world. On the other hand, the
temperature of the Earth‘s oceans is slowly but steadily going up. And this is a result of human activity. We are
producing greenhouse gases \underline{\textbf{that}} trap heat in the Earth‘s atmosphere. This heat warms up the atmosphere, land
and oceans. Warmer oceans produce more water vapour - think of heating a pan of water in your kitchen. Turn
up the heat, it produces steam more quickly. Satellite data tells us that the water vapour in the atmosphere has
gone up by four percent in 25 years. This warm, wet air turns into the rain, storms, hurricanes and typhoons that
we are increasingly experiencing. Climate scientist, Michael Oppenheimer, says that we need to face the reality of
climate change. And we also need to act now to save lives and money in the future.

(Source: \textit{\copyright 2015 National Geographic Learning.www.ngllife.com/wild-weather})

\begin{question} %%43
It is stated in the passage that extreme weather is \ktrang .
\datcot[2]
\bonpa
{\dung{becoming more common}}
{\sai{not a natural occurrence}}
{\sai{difficult for scientists to understand }}
{\sai{ killing more people than ever before}}
\end{question}

\begin{question} %%44
The word "\textbf{lethal}" in the second paragraph probably means \ktrang .
\datcot
\bonpa
{\sai{far-reaching}}
{\sai{long-lasting}}
{\sai{happening soon}}
{\dung{causing deaths}}
\end{question}

\begin{question} %%45
What caused thousands of deaths in 2003?
\datcot[2]
\bonpa
{\dung{a period of hot weather }}
{\sai{floods after a bad summer}}
{\sai{a long spell of heavy rain}}
{\sai{ large-scale landslides}}
\end{question}

\begin{question} %%46
According to the passage, extreme weather is a problem because \ktrang .
\datcot[2]
\bonpa
{\sai{we can never predict it }}
{\sai{it only affects crowded places}}
{\dung{ it‘s often very destructive }}
{\sai{its causes are completely unknown}}
\end{question}


\begin{question} %%47
The word "\textbf{that}" in the third paragraph refers to \ktrang .
\datcot[2]
\bonpa
{\sai{Earth's oceans}}%
{\sai{human activity }}%
{\dung{greenhouse gases}}%
{\sai{Earth's atmosphere}}%
\end{question}

\begin{question} %%48
 Extreme weather can be caused by \ktrang .
\datcot[2]
\bonpa
{\sai{satellites above the Earth}}
{\dung{water vapour in the atmosphere}}
{\sai{very hot summers}}
{\sai{water pans in your kitchen}}
\end{question}

\begin{question} %%49
Satellites are used to \ktrang .
\datcot[2]
\bonpa
{\sai{change the direction of severe storms}}
{\sai{trap greenhouse gases in the atmosphere}}
{\dung{measure changes in atmospheric water vapour}}
{\sai{prevent climate from changing quickly}}
\end{question}

\begin{question} %%50
Which statement is NOT supported by the information in the passage?
\datcot
\bonpa
{\sai{Extreme weather is substantially influenced by human activity.}}
{\sai{Unusual weather events are part of natural cycles.}}
{\sai{We can limit the bad effects of extreme weather.}}
{\dung{Such extreme weather is hardly the consequence of human activity.}}
\end{question}



\begin{examclosing}
\centerline{-- \textbf{THE END} --}
\end{examclosing}
\end{multiplechoice}
\end{document}
[scribd id=328834951 key=key-dC3dB1FkpzJ1nlF9gAby mode=scroll]

%%20:13:21 25/10/2016Last Modification of contents