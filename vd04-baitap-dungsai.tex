% Tệp mẫu làm đề thi điền chỗ trống dựa vào gói lệnh lamdethi.sty
% Tác giả: Nguyên Hữu Điển
% Khoa Toán Cơ Tin học, ĐHKHTN HN, ĐHQGHN
% 334, Nguyễn Trãi, Thanh Xuân, Hà Nội
% huudien@vnu.edu.vn
% Ngày 26/12/2009
%%%%%%%%%%%%%%%%%%%%%%%%%%%%
\documentclass[11pt]{article}
\usepackage{amsmath,amssymb}
\usepackage{graphicx}
\usepackage[utf8]{vietnam}
\usepackage{longtable}%
\usepackage[baitap]{dethi}
\usepackage{mathpazo}
%\voffset=-0.8in
\hoffset=-2cm
\textheight 22truecm 
\textwidth 18truecm 

\tentruong{ĐẠI HỌC KHOA HỌC TỰ NHIÊN}
\tenkhoa{Khoa Toán - Cơ -Tin học}
\loaidethi{Đề gồm có \pageref{LastPage} trang}%{ĐỀ THI LẠI}%%{ĐỀ CHÍNH THỨC}
\tenkythi{ĐỀ THI GIỮA KỲ NĂM HỌC 2016-2017}
\tenmonhoc{Môn: Toán học tính toán}
\madethi{100}
\thoigian{\underline{Thời gian làm bài: 90 phút, không kể thời gian phát đề}}   
\hovaten{Họ và tên}         %Nếu không muốn có dòng này không gõ lệnh
\tenlop{Tên lớp}         %Nếu không muốn có dòng này không gõ lệnh
\sobaodanh{Số báo danh}  %Nếu không muốn có dòng này không gõ lệnh
\renewcommand{\solutionname}{Lời giải}
\PSNrandseed{\time}
\begin{document}
\lamtieude
%Lấy 1 lần duy nhất vào bộ nhớ
% \loadrandomproblems[btdiencho]{5}{cauhoidiencho}
% \loadrandomproblems[btdungsai]{5}{cauhoi-dungsai}
\loadrandomproblems[btdungsai]{5}{vd04-cauhoi-dungsai}

%In ra các câu hỏi
\hideanswers

\noindent {\bf 1. Trả lời đúng sai}
\begin{enumerate}[]
\foreachproblem[btdungsai]{\item\causo\thisproblem}
\end{enumerate}

\vspace*{1cm}
{\bf Chú ý:} Sinh viên không được mang sách giáo trình và vở ghi chép vào phòng thi.


%%Đáp án của câu hỏi
\newpage
%\hideproblems
\setcounter{page}{1}
\lamtieude
\begin{center}
{\bf ĐỀ BÀI VÀ ĐÁP ÁN }
\end{center}
% \showanswers

\medskip
\noindent {\bf 1. Trả lời đúng sai}
\indebailoigiai
\begin{enumerate}[]
\foreachproblem[btdungsai]{\item\causo\thisproblem}
\end{enumerate}
\end{document}
%%%%%%%%%%

