
%Tệp mẫu làm đề thi trắc nghiệm
%Tác giả Nguyễn Hữu Điển
% ĐHKHTN, Hà Nội
% Đề trắc nghiệm được thiết kế trên phông Unicode,
%Đã dùng lớp examdesign.cls có sửa đổi
%Cùng với gói lệnh tracnghiem.sty tạo ra:
%Đề thi trắc nghiệm từ một bộ đề sinh ra các câu hởi được sắp xếp ngẫu nhiên và các chi tiết của câu hỏi cũng được xắp sếp ngẫu nhiên.
%Mỗi đề thi sinh ra đều có thể in ra đáp án riêng biệt.
%examdesign.cls đòi hỏi các gói lệnh enumerate, multicol, shortlst, keyval.
\documentclass[11pt]{examdesign}
\usepackage{amsmath,amsxtra,latexsym, amssymb, amscd}
\usepackage[utf8]{vietnam}
\usepackage{graphicx} 
\usepackage{wrapfig} 
\usepackage{ifthen} 
\usepackage{tracnghiem} 
\usepackage{mathptmx,courier}
\usepackage{indentfirst}
\usepackage[mathscr]{eucal}

\usepackage{picinpar}
\usepackage{floatflt}

\usepackage{epic}
\usepackage{curves}
\usepackage{makeidx}
\usepackage{longtable}%
\usepackage{multicol}%
\usepackage{listings}

\renewcommand{\lstlistingname}{\bf Chương trình}

\newcommand{\vd}[1]{\par\medskip\noindent{\large\bf #1}}
\newcommand{\chuong}[1]{\par\bigskip\noindent{\large\bfh #1}}
\renewcommand{\figurename}{{Hình}}

\Fullpages
\ContinuousNumbering %Đánh số liên tục các bài thi
\ShortKey 
%\OneKey %Lệnh chỉ in ra 1 bản đáp án
\NoKey %Lệnh không in ra phần đáp án
\NumberOfVersions{20} %10 là số bài thi khác nhau được in ra
\SectionPrefix{\relax }%\bf Phần \Roman{sectionindex}. \space}

\renewcommand{\tentruong}{ĐẠI HỌC BÁCH KHOA HÀ NỘI}
\renewcommand{\tenkhoa}{Khoa Toán  - Tin học ứng dụng}
\renewcommand{\tenkythi}{ĐỀ KIỂM TRA GIỮA KÌ I}
\renewcommand{\tenmonhoc}{Giải tích số}
\renewcommand{\tenlop}{SP/2007-2008}
\renewcommand{\thoigian}{60 phút}
\renewcommand{\chuy}{Được mở vở, cán bộ coi thi không giải thích gì thêm}
%\tieudetracnghiem
\tieudethi
\tieudedapan
%\tieudetren
%\tieudeduoi
%Lệnh đường dẫn đến các câu hỏi
\newcommand{\cauhoichoice}[1]{\input{c:/vietex/thitracnghiem/multiplechoice/#1}}
\newcommand{\cauhoifillin}[1]{\input{c:/vietex/thitracnghiem/fillin/#1}}
\newcommand{\cauhoitruefalse}[1]{\input{c:/vietex/thitracnghiem/truefalse/#1}}
\newcommand{\cauhoimatching}[1]{\input{c:/vietex/thitracnghiem/matching/#1}}
%\UnderlineCorrectMultipleChoiceAnswer
\BoldfaceCorrectMultipleChoiceAnswer
\def\v#1{\overrightarrow{#1}}
\begin{document}



%\usepackage{program}

\begin{document}
\setlength{\baselineskip}{14truept}

\centerline{\bf Đề thi 1:  Lí thuyết đồ thị và ứng dụng}    
\medskip     
\centerline{\bf (K49-2007: Thời gian 90 phút)}
\vspace*{.2cm}
\noindent{\bf Câu 1.}  Phát biểu các quy tắc tìm chu trình Hamilton. Hãy chứng minh rằng đồ thị trong hình ~\ref{fig:hinh1} có chu trình Hamilton.
\begin{figure}[!ht]
\begin{minipage}[b]{5cm}
\centering
\input hamilton.tex
\caption{}\label{fig:hinh1}
\end{minipage}
\hfill
\begin{minipage}[b]{9cm}
\begin{tabular}{ l  l }
 Tên sinh viên & Môn đã chọn học \\
\hline
 An & Vật lí, Toán, Tiếng Anh \\
Vân& Vật lí, Khoa học trái đất, Kinh tế\\
Lan & Khoa học trái đất, Kinh doanh\\
Hào &Thống kê, Kinh tế\\
Hùng&Toán, Kinh doanh\\
Bảo&Vật lí, Khoa học trái đất\\
Hồng& Kinh doanh, Thống kê\\
Điền &Toán, Khoa học trái đất\\
Cường&Vật lí, Môi trường, Thống kê\\
Quý&Vật lí, Kinh tế, Môi trường.
\end{tabular}
\caption{}\label{fig:hinh00}
\end{minipage}
\end{figure}

\noindent{\bf Câu 2.} Có 10 sinh viên trong một học kì học các môn tùy chọn và các môn bắt buộc theo bảng ở hình ~\ref{fig:hinh00}. Hãy thiết lập lịch thi có ít đợt nhất   sao cho  các sinh viên này không bị nhỡ thi một môn đã học nào.


\noindent{\bf Câu 3.} Hãy phát biểu định nghĩa và các tính chất của đồ thị Kuratowski. Chứng minh rằng đồ thị  hình ~\ref{fig:hinh5} là không phẳng.

\begin{figure}[!ht]
\begin{minipage}[b]{5cm}
\centering
\input dothiphang1.tex
\caption{}\label{fig:hinh5}
\end{minipage}
\hfill
\begin{minipage}[b]{9cm}
\centering
\input thuattoan1.tex
\caption{}\label{fig:hinh7}
\end{minipage}
\end{figure}


\noindent{\bf Câu 4.} Trình bày thuật toán Prime. Dùng thuật toán Prime tìm cây bao trùm có trọng số nhỏ nhất trong đồ thị trong hình ~\ref{fig:hinh7}, bắt đầu từ đỉnh $S$.




\noindent{\bf Câu 5.} Cho chương trình nguồn sau đây bằng ngôn ngữ C. 
 Hãy giải thích các dòng lệnh của chương trình có ý nghĩa gì.
 Hãy vẽ lại đồ thị mà dữ liệu đã cho trong chương trình.
 Dựa vào đồ thị  và chương trình nguồn để tìm ra kết quả chạy chương trình. 
\lstset{language=C++,emph={printf,*},emphstyle=\bf,numbers=left, xleftmargin=16pt, numberblanklines=false, numberstyle=\footnotesize, numbersep=5pt, tabsize=4}
\lstset{emph={printf,fprintf,scanf,fscanf,fclose,fopen,gotoxy,clrscr},emphstyle=\bf}
\setlength{\columnseprule}{1pt}
\setlength{\columnsep}{10pt}
{\small
\begin{multicols}{2}
\begin{lstlisting}[caption={\bf Nội dung}, label={ct:dfs}]
#include <stdio.h>
#define MAXN 200
const unsigned n = 14;
const unsigned v = 5;
const char A[MAXN][MAXN] = {
{0, 1, 0, 0, 0, 0, 0, 0, 0, 0, 0, 0, 0, 0},
{1, 0, 1, 1, 1, 0, 0, 0, 0, 0, 0, 0, 0, 0},
{0, 1, 0, 0, 0, 1, 0, 0, 0, 0, 0, 0, 0, 0},
{0, 1, 0, 0, 0, 0, 0, 0, 0, 0, 0, 1, 0, 0},
{0, 1, 0, 0, 0, 0, 1, 0, 0, 0, 0, 0, 0, 0},
{0, 0, 1, 0, 0, 0, 1, 0, 0, 0, 0, 1, 0, 0},
{0, 0, 0, 0, 1, 1, 0, 0, 0, 1, 0, 0, 0, 0},
{0, 0, 0, 0, 0, 0, 0, 0, 1, 0, 0, 0, 0, 0},
{0, 0, 0, 0, 0, 0, 0, 1, 0, 0, 0, 0, 1, 1},
{0, 0, 0, 0, 0, 0, 1, 0, 0, 0, 1, 0, 0, 0},
{0, 0, 0, 0, 0, 0, 0, 0, 0, 1, 0, 0, 0, 0},
{0, 0, 0, 1, 0, 1, 0, 0, 0, 0, 0, 0, 0, 0},
{0, 0, 0, 0, 0, 0, 0, 0, 1, 0, 0, 0, 0, 1},
{0, 0, 0, 0, 0, 0, 0, 0, 1, 0, 0, 0, 1, 0}
};

char used[MAXN];

void DFS(unsigned i)
{ 
  unsigned k;
  used[i] = 1;
  printf("%u ", i+1);
  for (k = 0; k < n; k++)
    if (A[i][k] && !used[k]) DFS(k);
}

int main(void) {
  unsigned k;
  for (k = 1; k < n; k++) used[k] = 0;
  printf("Duyet tu dinh %u: \n", v);
  DFS(v-1);
  printf("\n");
  return 0;
}
\end{lstlisting}
\end{multicols}
}

\end{document}