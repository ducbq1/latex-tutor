%%%%%17:33:53 21/10/2016 -VieTeX creates 
\baitracnghiemba{tracnghiemba:b01}{%
Hệ số của số hạng không chứa $x$ trong khai triển $(\dfrac{1}{2}\sqrt x+\dfrac{2}{\sqrt[3]{x}})^{10}$ là:
}{
\datcot
\bapa
{\dung{$840$}}
{\sai{Không có}}
{\sai{$210$}}
}
%%%%%%%%%%%%%
\baitracnghiemba{tracnghiemba:b02}{%
Cho $A=\{ 0,1,2,3,4,5,6\}$. Số các số chẵn có 4 chữ số đôi một khác nhau được lập từ $A$ là:
}{
\datcot
\bapa
{\dung{$420$}}
{\sai{$360$}}
{\sai{$15$}}
}
%%%%%%%%%%%%%
\baitracnghiemba{tracnghiemba:b03}{%
Thể tích vật thể tròn xoay khi quay miền phẳng giới hạn bởi các đường $y=x^3-3x-4$, $y=0$,  $x=0, x=1$ quanh $Ox$ có số đơn vị thể tích là:
}{
\datcot
\bapa
{\dung{$27\dfrac{33}{35}\pi$}}
{\sai{$29\dfrac{33}{35}\pi$}}
{\sai{$\dfrac{9\pi}{4}$}}
}
%%%%%%%%%%%%%
\baitracnghiemba{tracnghiemba:b04}{%
Diện tích miền phẳng giới hạn bởi các đường $(C): y=x^3-3x-4, y=0, x=1, x=2$ có số đơn vị diện tích là:
}{
\datcot
\bapa
{\dung{$\dfrac{19}{4}$}}
{\sai{$\dfrac{35}{4}$}}
{\sai{$\dfrac{17}{4}$}}
}
%%%%%%%%%%%%%
\baitracnghiemba{tracnghiemba:b05}{%
Trong các cặp véc tơ sau, cặp véc tơ nào có phương vuông góc với nhau:
}{
\datcot
\bapa
{\dung{$(0,-1)$ và $(3,0)$}}
{\sai{$(3,2)$ và $(-4,1)$}}
{\sai{$(1,3)$ và $(2,-1)$}}
}
%%%%%%%%%%%%%
\baitracnghiem{tracnghiemba:b06}{%
Bán kính đường tròn có phương trình 
 $\left\{	\begin{array}{l}	x^2+y^2+z^2-2x-2y-2z-22=0\\	
3x-2y-6z+14=0	\end{array}\right.$ là:
}{
\datcot
\bonpab
{\dung{$r=3$}}
{\sai{$r=4$}}
{\sai{$r=2$}}
{\sai{Các kết quả trên đều sai}}
}
%%%%%%%%%%%%%
\baitracnghiemba{tracnghiemba:b07}{%
Cho đường thẳng $(d_1): x+2y-1=0$ và đường thẳng $(d_2): \left\{	\begin{array}{l}	x=1-2t\\	y=3+t	\end{array}\right.$
cosin của góc giữa $(d_1)$ và $(d_2)$ là:
}{
\datcot
\bapa
{\dung{$1$}}
{\sai{$0$}}
{\sai{$-1$}}
}
%%%%%%%%%%%%%
\baitracnghiemba{tracnghiemba:b08}{%
Cho đường thẳng $(d_1): x+2y-1=0$ và $M(1;2)$. Điểm đối xứng của $M$ qua $(d_1)$ là:
}{
\datcot
\bapa
{\dung{$(-\dfrac{3}{5};-\dfrac{6}{5})$}}
{\sai{$(1;0)$}}
{\sai{$(0;2)$}}
}
%%%%%%%%%%%%%
\baitracnghiemba{tracnghiemba:b09}{%
Cho đường tròn $(C): x^2+y^2-2x+4y-4=0$. 
Đường thẳng nào sau đây là tiếp tuyến của đường tròn:
}{
\datcot
\bapa
{\dung{$y=1$}}
{\sai{$x+y-2=0$}}
{\sai{$2x+y-1=0$}}
}
%%%%%%%%%%%%%
\baitracnghiemhai{tracnghiemba:b10}{%
Phương trình elíp nào dưới đây có tiêu điểm $F(-3;0)$ và đường chuẩn $x=-\dfrac{25}{3}$
}{
\datcot
\haipa
{\dung{$\dfrac{x^2}{25}+\dfrac{y^2}{16}=1$}}
{\sai{$\dfrac{x^2}{16}+\dfrac{y^2}{25}=1$}}
}
%%%%%%%%%%%%%
\baitracnghiemba{tracnghiemba:b11}{%
Cho hypebol $(H): \dfrac{x^2}{9}-\dfrac{y^2}{4}=1$, cặp đường thẳng nào là tiệm cận của $(H)$:
}{
\datcot
\bapa
{\dung{$y=\pm\dfrac{2}{3}x$}}
{\sai{$y=\pm\dfrac{3}{2}x$}}
{\sai{$y=\pm\dfrac{\sqrt{13}}{3}x$}}
}
%%%%%%%%%%%%%
\baitracnghiemhai{tracnghiemba:b12}{%
Cho parabol $(P): y^2=4x$. Tiếp tuyến với parabol $(P)$ tại $(1;-2)$ là:
}{
\datcot
\haipa
{\dung{$x+y+1=0$}}
{\sai{$x+y-1=0$}}
}
%%%%%%%%%%%%%
\baitracnghiemba{tracnghiemba:b13}{%
Cho $F(2;3)$ là tiêu điểm của conic và $\Delta : x+y-1=0$ là đường chuẩn, $e=\dfrac{1}{\sqrt 2}$ là tâm sai conic đó. Phương trình của conic đó là:
}{
\datcot[4]
\bapa
{\dung{$3x^2+3y^2-14x-22y-2xy+51=0$}}
{\sai{$3x^2+3y^2+14x+22y+2xy-51=0$}}
{\sai{$3x^2+3y^2-18x-26y-2xy+51=0$}}
}
%%%%%%%%%%%%%
\baitracnghiemhai{tracnghiemba:b14}{%
Phương trình mặt phẳng qua $A(1;2;3)$, $B(0;2;4)$ và vuông góc với mặt phẳng $(\alpha): x+2y+3z+1=0$ là:
}{
\datcot
\haipa
{\dung{$x-2y+z=0$}}
{\sai{$x+2y-z-2=0$}}
}
%%%%%%%%%%%%%
\baitracnghiem{tracnghiemba:b15}{%
Phương trình đường thẳng qua $(1;2;-1)$ và song song với đường thẳng 
$\left\{	\begin{array}{l}	x+y-z+3=0\\	2x-y+5z-4=0	\end{array}\right.$ là:
}{
\datcot[2]
\bonpab
{\dung{$\left\{	\begin{array}{l}	7x+4y-15=0\\	3y-7z-13=0	\end{array}\right.$}}
{\sai{$\dfrac{x-1}{4}=\dfrac{y-2}{-7}=\dfrac{z-1}{-3}$}}
{\sai{$\left\{	\begin{array}{l}	x=1+4t\\ y=2-7t\\ z=-1-3t	\end{array}\right.$}}
{\sai{Các kết quả trên đều đúng}}
}
%%%%%%%%%%%%%
\baitracnghiemba{tracnghiemba:b16}{%
Đường thẳng qua $(0;1;-1)$, vuông góc và cắt đường thẳng $\left\{	\begin{array}{l}	x+4y-1=0\\	x+z=0	\end{array}\right.$ là:
}{
\datcot[2]
\bapa
{\dung{$\left\{	\begin{array}{l}	4x-y-4z-3=0\\	4x+4y+3z-1=0	\end{array}\right.$}}
{\sai{$\left\{	\begin{array}{l}	4x+y-4z-3=0\\	4x+4y+3z-1=0	\end{array}\right.$}}
{\sai{$\left\{	\begin{array}{l}	4x-y-4z-3=0\\	x+y+3z-1=0	\end{array}\right.$}}
}
%%%%%%%%%%%%%
\baitracnghiemba{tracnghiemba:b17}{%
Trong không gian Oxyz cho 3 véc tơ: $\v{a}(4;2;5), \v{b}(3;1;3), \v{c}(2;0;1)$. Kết luận nào sau đây đúng:
}{
\datcot
\bapa
{\dung{3 véc tơ đồng phẳng}}
{\sai{$\v{c}=[\v{a},\v{b}]$}}
{\sai{3 véc tơ cùng phương}}
}
%%%%%%%%%%%%%
\baitracnghiemhai{tracnghiemba:b18}{%
Cho $A(1;2;5), B(1;0;2), C(4;7;-1), D(4;1;a)$. Để 4 điểm $A, B, C, D$ đồng phẳng thì $a$ bằng:
}{
\datcot
\haipa
{\dung{$-10$}}
{\sai{$0$}}
}
%%%%%%%%%%%%%
\baitracnghiem{tracnghiemba:b19}{%
Khoảng cách từ $M(1;-1;1)$ đến đường thẳng $(d): \dfrac{x+1}{1}=\dfrac{y-1}{2}=\dfrac{z+1}{-2}$ là:
}{
\datcot
\bonpa
{\dung{$2\sqrt 2$}}
{\sai{$6\sqrt 2$}}
{\sai{$0$}}
{\sai{$4\sqrt 2$}}
}
%%%%%%%%%%%%%
\baitracnghiem{tracnghiemba:b20}{%
Phương trình mặt phẳng qua $A(1;0;-1)$ và qua giao tuyến của 2 mặt phẳng \\ 
$x-3y+2z-1=0$ và $2x+y-3z+1=0$ là:
}{
\datcot[2]
\bonpa
{\dung{$5x-5y+3z-2=0$}}
{\sai{$x-y+3z+2=0$}}
{\sai{$x+y+3z-2=0$}}
{\sai{$5x+5y+3z+2=0$}}
}
