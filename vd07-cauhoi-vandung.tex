%-[câu hỏi 1]%%%%%%%%%%
\baitracnghiem{vandung:b01}{%
Cho hình chóp ${S.ABCD}$ có đáy ${ABCD}$ là hình vuông cạnh ${a}$ . Hai mặt phẳng $(SAB)$ và $(SAD)$ cùng vuông góc với đáy. Tính thể tích khối chóp ${S.ABCD}$ biết rằng giữa ${SC}$ và $(ABCD)$ bằng $45^\circ$. 
}{
\datcot
\bonpa
{\dung{$\dfrac{{{a}^{3}}\sqrt{2}}{3}$}}
{\sai{$\dfrac{{{a}^{3}}\sqrt{2}}{6}$}}
{\sai{$\dfrac{{{a}^{3}}}{6}$}}
{\sai{$\dfrac{{{a}^{3}}}{3}$}}
}
%-[câu hỏi 2]%%%%%%%%%%
\baitracnghiem{vandung:b02}{%
Cho hình chóp ${S.ABC}$ có đáy ${ABC}$ là tam giác cân tại ${A}$, trong đó $AB = AC = a, \widehat{BAC}=120^\circ$. Mặt bên ${SAB}$ là tam giác đều và nằm trong mặt phẳng vuông góc với đáy. Thể tích khối chóp ${S.ABC}$ theo ${a}$ là: 
}{
\datcot
\bonpa
{\dung{$\dfrac{{{a}^{3}}}{8}$}}
{\sai{${{a}^{3}}$}}
{\sai{$\dfrac{{{a}^{2}}}{2}$}}
{\sai{$2{{a}^{3}}$}}
}
%-[câu hỏi 3]%%%%%%%%%%
\baitracnghiem{vandung:b03}{%
Cho hình  chóp ${S.ABCD}$ có đáy ${ABCD}$ là hình thoi, cạnh bằng $a\sqrt{3}$; $SA\bot(ABCD)$; $\widehat{BAD}={{120}^\circ}$. Tính thể tích khối chóp ${S.ABCD}$ biết rằng góc giữa mặt phẳng $(SCD)$ và $(ABCD)$ bằng $30^o$ . 
}{
\datcot
\bonpa
{\dung{$\dfrac{3{{a}^{3}}}{4}$}}
{\sai{$\dfrac{{{a}^{3}}\sqrt{3}}{3}$}}
{\sai{$\dfrac{2{{a}^{3}}\sqrt{3}}{3}$}}
{\sai{$\dfrac{{{a}^{3}}}{8}$}}
}
%-[câu hỏi 4]%%%%%%%%%%
\baitracnghiem{vandung:b04}{%
Cho hình  chóp ${S.ABC}$ có đáy ${ABC}$ là tam giác vuông cân tại $B,  AC=a\sqrt{2}$; $SA\bot(ABC)$; $SA=a$. Gọi ${G}$ là trọng tâm của tam giác ${ABC}$, mặt phẳng $\left( \alpha  \right)$ qua ${AG}$ và song song với ${BC}$ cắt ${SC, SB}$ lần lượt tại ${M , N}$. Tính thể tích khối chóp ${S.AMN}$ .
}{
\datcot
\bonpa
{\dung{$\dfrac{2{{a}^{3}}}{27}$}}
{\sai{$\dfrac{{{a}^{3}}}{6}$}}
{\sai{$\dfrac{{{a}^{3}}}{27}$}}
{\sai{$\dfrac{2{{a}^{3}}}{6}$}}
}
%-[câu hỏi 5]%%%%%%%%%%
\baitracnghiem{vandung:b05}{%
Cho hình  chóp ${S.ABC}$ có đáy ${ABC}$ là tam giác đều cạnh ${2a}$,  $SA=a\sqrt{3}$; $SA\bot(ABC)$. Gọi ${M, N}$ lần lượt là trung điểm ${SB, SC}$. Thể tích khối chóp ${S.AMN}$ và ${A.BCNM}$ lần lượt là: 
}{
\datcot
\bonpa
{\dung{$\dfrac{{{a}^{3}}}{4}$và $\dfrac{3{{a}^{3}}}{4}$}}
{\sai{$\dfrac{{{a}^{3}}}{8}$và $\dfrac{3{{a}^{3}}}{8}$}}
{\sai{$\dfrac{{{a}^{3}}}{4}$và $\dfrac{3{{a}^{3}}}{8}$}}
{\sai{$\dfrac{{{a}^{3}}}{8}$và $\dfrac{3{{a}^{3}}}{4}$}}
}
%-[câu hỏi 6]%%%%%%%%%
\baitracnghiem{vandung:b06}{%
Cho hình chóp $S.ABCD$ có đáy $ABCD$ là hình vuông cạnh $a,SA=a\sqrt{3}$  và vuông góc với đáy. Tính khoảng cách từ $A$ đến mặt phẳng $(SBC)$ bằng
}{
\datcot
\bonpa
{\dung{$\dfrac{a\sqrt{3}}{2}$}}
{\sai{$\dfrac{a}{2}$}}
{\sai{$\dfrac{a\sqrt{2}}{2}$}}
{\sai{$\dfrac{a\sqrt{2}}{4}$}}
}
%-[câu hỏi 7]%%%%%%%%%%
\baitracnghiem{vandung:b07}{%
Cho tứ diện đều $ABCD$ cạnh bằng $a,M$ là trung điểm của $CD$. Tính cosin góc giữa $AC$ và $BM$ bằng
}{
\datcot
\bonpa
{\dung{$\dfrac{\sqrt{3}}{6}$}}
{\sai{$\dfrac{\sqrt{3}}{4}$}}
{\sai{$\dfrac{\sqrt{3}}{3}$}}
{\sai{$\dfrac{\sqrt{3}}{2}$}}
}