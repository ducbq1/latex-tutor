%MẪU CUỐN SÁCH TRONG VieTeX
%Dùng với gói lệnh lamdethi.sty
%Dùng Vietex 2.8. với phông Unicode
%Người soạn : Nguyễn Hữu Điển, ĐHKHTN, ĐHQG HN
%Mail: huudien@vnu.edu.vn, CQ: (84 - 4) 557 2869
%NR: (84 - 4) 641 8848, DĐ: 0989061951
%%%%%%%%%%%%%%%%%%%%%%%%%
%-[Phần khai báo
\documentclass[11pt,openany]{book} 
\usepackage{amsmath,amssymb,amsthm}
\usepackage{indentfirst}
\usepackage{fancyhdr}
\pagestyle{fancyplain}
\pagestyle{fancy}
\usepackage{makeidx}
\usepackage{graphicx}
\usepackage{longtable}%
\usepackage{multicol}%
\usepackage{fancybox}
\usepackage[utf8]{vietnam}
\usepackage{color, graphicx}
\usepackage[chapter]{algorithm}
\usepackage{algorithmic}
\floatname{algorithm}{\hfil Thuật toán}
\usepackage{titledot}
\usepackage{mathpazo}


%Chiều dài và rộng của trang sách
\voffset-2cm
\textheight 24truecm %%16.2truecm
\textwidth 16.5truecm %11.3truecm
\parskip 3pt
\headsep=12pt
\usepackage[baitap]{dethi}
\renewcommand{\headwidth}{16.5truecm}
\renewcommand{\chaptermark}[1]%
              {\markboth{\it #1}{}}
\renewcommand{\sectionmark}[1]%
           {\markright{\it \thesection\ #1}}
\lhead[\fancyplain{}{\thepage}]%
{\fancyplain{}{\rightmark}}
\rhead[\fancyplain{}{\leftmark}]%
   {\fancyplain{}{\thepage}}
\cfoot{}
\sloppy

\renewcommand{\binom}[2]{C_{#1}^{#2}} 
\def\cung#1{\buildrel \frown \over{#1}}
\def\g.#1.{\widehat{#1}}
\newcommand{\chia}{\;\vdots\;} 
\newcommand{\kchia}{\not\vdots\;\;} 
\makeindex
\theoremstyle{definition}
\newtheorem{definition}{Định nghĩa}[chapter]
\theoremstyle{plain}
\newtheorem{theorem}{Định lý}[chapter]
\newtheorem{proposition}{Mệnh đề}[chapter]
\newtheorem{lemma}{Bổ đề}[chapter]
\usepackage[tight,vietnam]{minitoc}%
\renewcommand{\mtctitle}{Nội dung của chương} %dinh nghia lai ten
\setcounter{minitocdepth}{1}
% \khoanh{\cbox}
\renewcommand{\solutionname}{\textbf{Lời giải}}
\PSNrandseed{\time}
\usepackage[colorlinks,hyperindex,unicode]{hyperref}
\graphicspath{{hinh-cauhoi/}} 
% \newcommand*\tikzcircled[1]{\tikz[baseline=(char.base)]{
%             \node[shape=circle,draw,inner sep=1pt] (char) {\small #1};}}

\khoanh{\cboxx}
\daungoac{\cboxx}{}
% \chuphuongan{\small\bfseries\Alph}
\mauchu{blue}
\usepackage{centerpage}
\begin{document}
% \setlength{\shortitemwidth}{0.15\textwidth}
% \setlength{\runitemsep}{1em plus .5em minus .5em}.
% \setlength{\shortitemwidth}{\textwidth/4-1.8\labelwidth}
\dominitoc
\begin{titlepage}
 \centerline{\bf Nguyễn Hữu Điển}
\vspace*{5cm}
 \centerline{\Huge\bf LÀM SÁCH BÀI TẬP TOÁN}
\vspace*{0.5cm}

\vfill
 \centerline{\bf NHÀ XUẤT BẢN GIÁO DỤC}
\end{titlepage}
\newpage
\vspace*{5cm}
\vfill
\hrule

$\dfrac{\mbox{51}}{\mbox{GD-05}}$89/176-05\hfill Mã số: 8I092M5
\thispagestyle{empty}

\newpage
\markboth{{\it Mục lục}}{{\it Mục lục}}
\addcontentsline{toc}{section}{{\bf Mục lục\rm }}
\tableofcontents

\hideanswers
% \setlist{noitemsep}
% \setlist{nolistsep}
%%%%%%%%%%%%%%%%
%  \setcounter{chapter}{13}
\chapter{Bài tập tự luận}
\minitoc %
\thispagestyle{empty}

\section{Giới thiệu}

\subsection{Giới thiệu}

\section{Nguyên lý cơ bản}
  %\renewcommand{\labelenumi}{$\triangleright$\textbf{\thechapter.\arabic{enumi}}}%%%\textcolor{blue}{\bf Câu \ifnum\value{enumi}<10 \thechapter.\arabic{enumi}.\else \thechapter.\arabic{enumi}.\fi}}
\section{Bài tập}
\loadrandomproblems[bttuluan]{5}{vd01-cauhoi-tuluan}

 % \renewcommand{\thesocauhoi}{$\triangleright$\textbf{\thechapter.\arabic{socauhoi}}}
\chucauhoi{Bài }
\sotheo{\thechapter.}
\begin{enumerate}[]
\foreachproblem[bttuluan]{\item\causo \thisproblem}
\end{enumerate}

\section{Lời giải}
\indapanloigiai
\begin{enumerate}[]
% \setlength{\itemindent}{3.5em}
\foreachproblem[bttuluan]{\item\causo\thisproblem}
\end{enumerate}

\section{Đề bài và lời giải}
\indebailoigiai
\begin{enumerate}[]
\foreachproblem[bttuluan]{\item\causo\thisproblem}
\end{enumerate}

% \end{document}
%%%%%%%%%%%%%%%%
\chapter{Câu hỏi trắc nghiệm}
\minitoc %
\thispagestyle{empty}


\section{Giới thiệu}

\section{Nguyên lý cơ bản}

\section{Bài tập}

\loadrandomproblems[bttracnghiem]{50}{vd03-cauhoi-toan-baitracnghiem}
\indebai
% \renewcommand{\theenumi}{$\triangleright$\textbf{\thechapter.\arabic{enumi}}}
\begin{enumerate}[]
\foreachproblem[bttracnghiem]{\item\causo\thisproblem}
\end{enumerate}

\section{Lời giải}

\indapanlietke
%\renewcommand{\theenumi}{\thechapter.\arabic{enumi}}
\begin{multicols}{5}
\begin{enumerate}[\causo]
\foreachproblem[bttracnghiem]{\item\thisproblem}
\end{enumerate}
\end{multicols}

% \end{document}

%%%%%%%%%%%%%%%%
\chapter{Điền vào chỗ trống}
\minitoc %
\thispagestyle{empty}

\section{Giới thiệu}

\section{Nguyên lý cơ bản}

\section{Bài tập}

\loadrandomproblems[btdiencho]{6}{vd04-cauhoi-dienvao}

\indebai
% \renewcommand{\theenumi}{$\triangleright$\textbf{\thechapter.\arabic{enumi}}}
\begin{enumerate}[]
\foreachproblem[btdiencho]{\item\causo\thisproblem}
\end{enumerate}

\section{Lời giải}

\indebailoigiai

\begin{enumerate}[]
\foreachproblem[btdiencho]{\item\causo\thisproblem}
\end{enumerate}
% \end{document}
%%%%%%%%%%%%%%%%
\chapter{Câu hỏi đúng sai}
\minitoc %
\thispagestyle{empty}

\section{Giới thiệu}

\section{Nguyên lý cơ bản}

\section{Bài tập}

\loadrandomproblems[btdungsai]{5}{vd04-cauhoi-dungsai}

% \renewcommand{\theenumi}{\thechapter.\arabic{enumi}}
% \renewcommand{\theenumi}{$\triangleright$\textbf{\thechapter.\arabic{enumi}}}
\indebai
\begin{enumerate}[]
\foreachproblem[btdungsai]{\item\causo\thisproblem}
\end{enumerate}

\section{Lời giải}

\indebailoigiai
\begin{enumerate}[]
\foreachproblem[btdungsai]{\item\causo\thisproblem}
\end{enumerate}
%%%%%%%%%%%%%%%%%%


\chapter{Câu hỏi thiết kế}
\minitoc %
\thispagestyle{empty}

\section{Giới thiệu}

\section{Nguyên lý cơ bản}

\section{Bài tập}

% \renewcommand{\PSNitem}{\refstepcounter{problem}%
% \theproblem. }
% \renewcommand{\endPSNitem}{\\}
% 
% \renewenvironment{solution}{}{}
\newcommand{\selected}{\fbox{$\times$}}
\newcommand{\notselected}{\fbox{\phantom{$\times$}}}

\loadrandomproblems[thietke]{2}{vd04-cauhoi-baithietke}
\newcounter{sobai}
\setcounter{sobai}{0}
\hideanswers
\begin{longtable}{lrrrl}
\bfseries Câu hỏi & \bfseries A & \bfseries B & 
\bfseries C & \ifshowanswers \bfseries Reason\fi\\
\foreachproblem[thietke]{\addtocounter{sobai}{1}\thesobai.\ \thisproblem \\ }
%\selectrandomly{cauhoithietke}{2}
\end{longtable}



\section{Lời giải}

\showanswers
\begin{longtable}{lrrrl}
\bfseries Câu hỏi & \bfseries A & \bfseries B & 
\bfseries C & \ifshowanswers \bfseries Lý giải\fi\\
% \selectrandomly{cauhoi07-baithietke}{2}
\foreachproblem[thietke]{\addtocounter{sobai}{1}\thesobai.\ \thisproblem \\ }
\end{longtable}

%%%%%%%%%%%%%%%%%%
% \end{document}

\chapter{Trắc nghiệm theo bảng}
\minitoc %
\thispagestyle{empty}

\section{Giới thiệu}

\section{Nguyên lý cơ bản}

\section{Bài tập}
\newcounter{problem}
\renewcommand{\PSNitem}{\refstepcounter{problem}%
\theproblem. }
\renewcommand{\endPSNitem}{  }

\loadrandomproblems[btbangtn]{11}{vd04-cauhoi-tracnghiem-bang}

\setcounter{problem}{0}
\indebai

 \renewcommand{\arraystretch}{1.5}

\begin{longtable}{| p{0.6\textwidth} |c|}
\hline
\centering \textbf{Câu hỏi} & \textbf{Trả lời}\\ 
\hline %
\foreachproblem[btbangtn]{\addtocounter{problem}{1}\theproblem.\thisproblem}
&\\
\hline
\end{longtable}


\section{Lời giải}


\setcounter{problem}{0}

\indapanloigiai
\begin{longtable}{| p{0.6\textwidth} |c|}
\hline
\centering \textbf{Câu hỏi} & \textbf{Trả lời}\\ 
\hline %
\foreachproblem[btbangtn]{\addtocounter{problem}{1}\theproblem.\thisproblem}
&\\
\hline
\end{longtable}
%%%%%%%%%%%%%%%%%%

% \end{document}
\chapter{Câu hỏi gạch dưới}
\minitoc %
\thispagestyle{empty}

\section{Giới thiệu}

\section{Nguyên lý cơ bản}

\section{Bài tập}

 \def\dssuatu{suatu:1,suatu:2,suatu:3,suatu:4,suatu:5}
 \loadselectedproblems[btsuatu]{\dssuatu}{vd04-cauhoi-gachduoi}

\indebai
\tieude{Chọn phương án (A hoặc B, C, D) ứng với từ/ cụm từ có gạch dưới cần phải sửa để các câu sau trở thành chính xác.}
\begin{enumerate}[] 
\foreachproblem[btsuatu]{\item\causo\thisproblem}
\end{enumerate}

\section{Lời giải}
\indebailoigiai
\tieude{Chọn phương án (A hoặc B, C, D) ứng với từ/ cụm từ có gạch dưới cần phải sửa để các câu sau trở thành chính xác.}
\begin{enumerate}[]
\foreachproblem[btsuatu]{\item\causo\thisproblem}
\end{enumerate}

\indapanrutgon
\begin{multicols}{3}
\begin{enumerate}[\causo]
\foreachproblem[btsuatu]{\item\thisproblem}
\end{enumerate}
\end{multicols}
%%%%%%%%%%%%%%%%%%


%\appendix
\chapter{Đề bài và lời giai}
\minitoc %
\thispagestyle{empty}


\indebailoigiai
\section{Bài tập chương 1}
\setcounter{chapter}{1}
\begin{enumerate}[]
\foreachproblem[bttuluan]{\item\causo\thisproblem}
\end{enumerate}

\indebailoigiai
\section{Bài tập chương 2}
\setcounter{chapter}{2}
% \begin{multicols}{3}
\begin{enumerate}[]
\foreachproblem[bttracnghiem]{\item\causo\thisproblem}
\end{enumerate}
% \end{multicols}
%  \end{document}

\indebailoigiai
\section{Bài tập chương 3}
\setcounter{chapter}{3}
\begin{enumerate}[]
\foreachproblem[btdiencho]{\item\causo\thisproblem}
\end{enumerate}

\indebailoigiai
\section{Bài tập chương 4}
\setcounter{chapter}{4}
\begin{enumerate}[]
\foreachproblem[btdungsai]{\item\causo\thisproblem}
\end{enumerate}

\indebailoigiai
\section{Bài tập chương 6}
\setcounter{chapter}{6}
\setcounter{problem}{0}

\showanswers
\begin{longtable}{| p{0.6\textwidth} |c|}
\hline
\centering \textbf{Câu hỏi} & \textbf{Trả lời}\\ 
\hline %
\foreachproblem[btbangtn]{\addtocounter{problem}{1}\theproblem.\thisproblem}
&\\
\hline
\end{longtable}
%%%%%%%%%%%%%%%%%%
\indebailoigiai
\section{Bài tập chương 7}
\setcounter{chapter}{7}
\tieude{Chọn phương án (A hoặc B, C, D) ứng với từ/ cụm từ có gạch dưới cần phải sửa để các câu sau trở thành chính xác.}
\begin{enumerate}[]
\foreachproblem[btsuatu]{\item\causo\thisproblem}
\end{enumerate}

\begin{thebibliography}{99}
\addcontentsline{toc}{section}{{\bf Tài liệu tham khảo}\rm }%
\bibitem{rade}
H. Rademacher,
{\it Higher Mathematics from an Elementary point of view},
Birkhauser, 1983.
\bibitem{aigner}
Martin Aigner  and Gunter M. Ziegler,
{\it Proofs from the book},
Springer, 1999.
\bibitem{titu} 
T. Andreescu, R. Gelca,
{\it Mathematical olympiad challenges},  Birkhauser, 2002. 
\bibitem{Larson}
Loren C. Larson,
{\it Problem-Solving through problems},
Springer-Verlag, 1983.
\bibitem{dickson}
L. E. Dickson,
{\it New first course in the theory of equations}
John Wiley \& Sons, 1946.
\bibitem{cofman}
Judita Cofman,
{\it What to solve? Problems and Suggestions for Young Mathematicians},
Clarendon Press-Oxford, 1989.
\bibitem{nhddl}
Nguyễn Hữu Điển,
{\it Phương pháp Đirichle và ứng dụng},
NXBKHKT, 1999.
\bibitem{nhdqn}
Nguyễn Hữu  Điển,
{\it Phương pháp Quy nạp toán học},
NXBGD, 2000.
%\bibitem{nhdsp}
%Nguyễn Hữu  Điển,
%{\it Phương pháp Số phức với hình học phẳng},
%NXB ĐHQG, 2000.
\bibitem{nhdpc}
Nguyễn Hữu  Điển,
{\it Những phương pháp điển hình trong giải toán phổ thông},
NXBGD, 2001.
\bibitem{nhdct}
Nguyễn Hữu  Điển,
{\it Những phương pháp giải bài toán cực trị trong hình học},
NXBKHKT, 2001.
\bibitem{nhdst}
Nguyễn Hữu  Điển,
{\it Sáng tạo trong giải toán phổ thông},
NXBGD, 2002.
\bibitem{nhddt}
Nguyễn Hữu  Điển,
{\it Đa thức và ứng dụng},
NXBGD, 2003.

\bibitem{nhdptvd}
Nguyễn Hữu  Điển,
{\it Giải phương trình vô định nghiệm nguyên},
NXBĐHQG, 2004.

\bibitem{nhdinvar}
Nguyễn Hữu  Điển,
{\it Giải toán bằng phương pháp đại lượng bất biến},
NXBGD, 2004.


\end{thebibliography}

\end{document}

%%23:7:45 30/4/2017Last Modification of contents