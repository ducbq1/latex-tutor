%MẪU LÀM CÂU HỎI TRẮC NGHIỆM
%Dùng với gói lệnh lamdethi.sty
%Dùng Vietex 2.8. với phông mã Unicode
%Người soạn : Nguyễn Hữu Điển, ĐHKHTN, ĐHQG HN
%Mail: huudien@vnu.edu.vn, CQ: (84 - 4) 557 2869
%NR: (84 - 4) 641 8848, DĐ: 0989061951
%Ngày 26/12/2009
%%%%%%%%%%%%%%%%%%%%%%%%%

\baitracnghiem{mc:prod}{%Câu hỏi 1
Which of the following is the derivative of $x\sin(x)$?
}{%Phương án trả lời
\sai[2]{$\sin(x)$}
\dung[2]{$\sin(x) + x\cos(x)$}
\sai[2]{$x\cos(x)$}
\sai[2]{$x\sin(x)$}
}%Hết một bài

\baitracnghiem{mc:quot}{%Câu hỏi 2
Which of the following is the derivative of $\frac{\sin(x)}{x}$?
}{%Phương án trả lời
\sai {$\sin(x)$}
\sai {$\cos(x)$}
\dung {$\frac{\cos(x)x-\sin(x)}{x^2}$}
\sai {$x\cos(x)$}
}%Hết một bài

\baitracnghiem{mc:ba}{%Câu hỏi 3
Đẳng thức  $3^n+4^n=5^n$ đúng với $n$ bằng
}{%Phương án trả lời
\dung {$2$; }
\sai  {$1$;} 
\sai  {$3$;} 
\sai  {$4$;} 
}%Hết một bài
%%%%%%%%%%%%%
\baitracnghiem{mc:bon}{%Câu hỏi 4
Giải bất phương trình $||5x-3|+4x|<5$. Chỉ ra nghiệm nguyên dương nhỏ nhất.
}{%Phương án trả lời
\sai  {$1$;} 
\sai  {$3$;} 
\sai  {$0$;} 
\dung {$-1$;} 
}%Hết một bài
%%%%%%%%%%%%%
\baitracnghiem{mc:nam}{%Câu hỏi 5
Tìm các nghiệm nguyên của phương trình
$4x^2+12x+\frac{12}{x}+\frac{4}{x^2}=47.$
}{%Phương án trả lời
\dung {$2$;} 
\sai  {$-3$;} 
\sai  {$1$;} 
\sai  {$0$;} 
}%Hết một bài
%%%%%%%%%%%%%
\baitracnghiem{mc:sau}{%Câu hỏi 6
Giải phương trình $2^{3\frac{x-1}{x }}\cdot 3^x=\sqrt{9}$ và chỉ ra nghiệm không nguyên của nó.
}{%Phương án trả lời
\sai  {$\frac{3}{2}$;} 
\dung {$-3\log_32 $;} 
\sai  {$\frac{5}{7}$;} 
\sai  {$\log_23 $;} 
}%Hết một bài
%%%%%%%%%%%%%
\baitracnghiem{mc:bay}{%Câu hỏi 7
Giải hệ phương trình
$$\begin{cases}
x(x+3y)&=18, \\
y(3y+x)&=6 \\
\end{cases}$$
và chỉ ra đại lượng $n(x^2+y^2)$, ở đây $n$ là số nghiệm của hệ phương trình.
}{%Phương án trả lời
\sai  {$10$;} 
\sai  {$1$;} 
\dung {$20$;} 
\sai  {$6$;} 
}%Hết một bài
%%%%%%%%%%%%%
\baitracnghiem{mc:tam}{%Câu hỏi 8
Giải phương trình $\log_2\dfrac{x-2}{x+2}+\log_{\frac{1}{2}} \dfrac{2x-1}{6x+7}=0.$
}{%Phương án trả lời
\dung {$3$;} 
\sai  {$-1$;} 
\sai  {$2$;} 
\sai  {$4$;} 
}%Hết một bài
%%%%%%%%%%%%%
\baitracnghiem{mc:chin}{%Câu hỏi 9
Đơn giản biểu thức  $\tan\left( \frac{\pi}{4}+\frac{\alpha}{2}\right)\cdot \frac{1-\sin\alpha}{\cos\alpha}$.
}{%Phương án trả lời
\sai  {$\cos\alpha$;} 
\sai  {$\sin\alpha $; }
\dung {$1$;} 
\sai  {$2$;} 
}%Hết một bài
%%%%%%%%%%%%%
\baitracnghiem{mc:muoi}{%Câu hỏi 10
Tính
$\dfrac{4\tan 17^\circ}{\tan 34^\circ(1-\tan^2 17^\circ)}.$
}{%Phương án trả lời
\sai  {$2$;} 
\dung {$-1$;} 
\sai  {$3$;} 
\sai  {$0,5$;} 
}%Hết một bài
%%%%%%%%%%%%%
\baitracnghiem{mc:muoimot}{%Câu hỏi 11
Phân tích thành các thừa số tuyến tính của đa thức  $2x^3+11x^2+19x+10$.
}{%Phương án trả lời
\sai [2]{$(x-1)(x-2)(2x+5)  $;} 
\sai [2]{$(x+1)(2x+2)(x+5)$;} 
\dung[2]{$(x+1)(x+2)(2x+5)$;} 
\sai[2]{$(2x-1)(x+2)(x+5)$;} 
}%Hết một bài
%%%%%%%%%%%%%
\baitracnghiem{mc:muoihai}{%Câu hỏi 12
Năm người làm một số công việc. Ba người đầu tiên trong họ làm cùng nhau để hoàn thành công việc trong thời gian 7.5h; Người thứ nhất, thứ ba và thứ năm - trong thời gian  5h; Người thứ nhất, thứ ba và thứ tư - trong  6h; Người thứ hai, thứ tư và thứ năm - trong 4h. Hỏi trong bao lâu công việc sẽ hoàn thành khi cả năm người đều cùng làm?
}{%Phương án trả lời
    \sai  {$2h$;} 
    \sai  {$2.5h$;} 
    \dung {$3h$;} 
    \sai  {$4h$;} 
}%Hết một bài
%%%%%%%%%%%%%%%%%%%%%
\baitracnghiem{mc:muoiba}{%Câu hỏi 13
Đơn giản biểu thức
$ \dfrac{1+\tan 2\alpha+\tan^2 2\alpha}{1+\cot 2\alpha+\cot^2 2\alpha}.$
}{%Phương án trả lời
\sai {$\sin\alpha$;} 
\sai  {$\cos2\alpha$;} 
\sai  {$\cot\alpha$;} 
\dung {$\tan^22\alpha$;} 
}%Hết một bài
%%%%%%%%%%%%%%%%%%%%
\baitracnghiem{mc:muoibon}{%Câu hỏi 14
Giải phương trình $3.5^{2x-1}-2.5^{x-1}=0,2$.
}{%Phương án trả lời
\sai  {$\emptyset$;} 
\sai  {$2$;} 
\sai  {$1$;} 
\dung {$0$;} 
}%Hết một bài
%%%%%%%%%%%%%%%%%%%%
\baitracnghiem{mc:muoinam}{%Câu hỏi 15
Trong một cấp số nhân $b_1=54; S_3=78$. Tìm công bội của cấp số này. Trong trả lời chỉ ra công bội này nếu bài toán có một nghiệm hoặc tổng của các nghiệm nếu bài toán có nhiều hơn một nghiệm.
}{%Phương án trả lời
\sai  {$\frac{1}{3}$;} 
\sai  {$2$;} 
\sai  {$-\frac{4}{3}$;} 
\dung {$-1$;} 
}%Hết một bài
%%%%%%%%%%%%%%%%%%%%
\baitracnghiem{mc:muoisau}{%16
Giải bất phương trình $\lg(x^2-5x+7)< 0$. Hãy chỉ ra trung điểm của đoạn thẳng mà nó thỏa mãn bất đẳng thức trên.
}{%Phương án trả lời
\dung {$2,5$;} 
\sai  {$1$;} 
\sai  {$6$;} 
\sai  {$3$;} 
}%Hết một bài
%%%%%%%%%%%%%%%%%%%%
\baitracnghiem{mc:muoibay}{%Câu hỏi 17
Hai đầu của đường kính cách xa tiếp tuyến tương ứng là $1,6$ $m$ và $0,6$ $m$. Tìm độ dài của đường kính. Đưa ra trả lời bằng các số thập phân.
}{%Phương án trả lời
\sai  {$1 $ $m$;} 
\sai  {$ 2$ $m$;} 
\sai  {$3,8 $ $m$;} 
\dung {$2,2 $ $m$;} 
}%Hết một bài
%%%%%%%%%%%%%%%%%%%%
\baitracnghiem{mc:muoitam}{%Câu hỏi 18
Cho $b^c=64, b^a=8, c^a=7$. Tính $c^c$.
}{%Phương án trả lời
\sai  {$45$;} 
\sai  {$38$;} 
\dung {$49$;} 
\sai  {$20$;} 
}%Hết một bài
%%%%%%%%%%%%%%%%%%%%
\baitracnghiem{mc:muoichin}{%Câu hỏi 19
Mặt phẳng vuông góc với đường kính hình cầu, chia đường kính thành hai phần $3$ và $9$ $cm$.Thể tích thành phần thế nào? Hãy chỉ ra phần thể tích lớn hơn.
}{%Phương án trả lời
\sai  {$\pi $ $cm^3$;} 
\dung {$243\pi $ $cm^3$;} 
\sai  {$45 $ $cm^3$;} 
\sai { $54\pi $ $cm^3$;} 
}%Hết một bài
%%%%%%%%%%%%%%%%%%%%
\baitracnghiem{mc:haimuoi}{%Câu hỏi 20
Giải bất phương trình $\log_{0,1}\left[ \log_2\frac{x^2+1}{|x-1|}\right]<0$. Chỉ ra nghiệm âm nguyên lớn nhất.
}{%Phương án trả lời
\sai  {$-4$;} 
\dung {$-3$;} 
\sai  {$-3$;} 
\sai  {$-2$;} 
}%Hết một bài
%%%%%%%%%%%%%%%%%%%%
\baitracnghiem{mc:haimot}{%Câu hỏi 21
Trong hình tròn về hai phía của tâm ta vẽ hai cung song song có độ dài tương ứng là $12$ và $16$. Khoảng cách giữa chúng bằng $14$. Tìm bán kính đường tròn.
}{%Phương án trả lời
\sai  {$9$;} 
\dung {$10$;} 
\sai  {$8$;} 
\sai  {$7$;} 
}%Hết một bài
%%%%%%%%%%%%%%%%%%%%
\baitracnghiem{mc:haihai}{%%Câu hỏi 22
Giải phương trình 
$2.3^{x+1}-6.3^{x-1}-3^{x}=9$.
}{%Phương án trả lời
{\dung {$1$;}} 
{\sai  {$2$;}} 
{\sai  {$-3$;}} 
{\sai  {$-1$;}} 
}%Hết một bài

