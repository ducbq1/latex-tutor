%Tệp mẫu làm đề thi trắc nghiệm phiên bản 3.0
%Tác giả Nguyễn Hữu Điển (ĐHKHTN, Hà Nội)
% Đề trắc nghiệm được thiết kế trên phông Unicode,
%Đã dùng lớp examdesign.cls có sửa đổi
%Cùng với gói lệnh dethi.sty tạo ra:
%Đề thi trắc nghiệm từ một bộ đề sinh ra các câu hởi được 
%sắp xếp ngẫu nhiên và các chi tiết của câu hỏi cũng được 
%xắp sếp ngẫu nhiên. Mỗi đề thi sinh ra đều có thể in ra đáp án riêng biệt.
%examdesign.cls đòi hỏi các gói lệnh enumerate, multicol, shortlst, keyval.
\documentclass[11pt]{article}
\usepackage{amsmath,amsxtra,latexsym, amssymb, amscd}
\usepackage[utf8]{vietnam}
\usepackage{color}
\usepackage{graphicx}
\usepackage{lastpage}

\usepackage{mathptmx} 
% \usepackage{mathpazo} 
\usepackage{enumerate}
\usepackage{multicol}
\usepackage{shortlst}
\usepackage[baithi]{dethi} %Gói lệnh cho đề thi Việt Nam
% \usepackage{fancybox}
% \cornersize*{3.6mm}
\Fullpages %Định dạng trang đề thi
\ContinuousNumbering %Đánh số liên tục các bài thi
\NumberOfVersions{5} %10 là số bài thi khác nhau được in ra
\SectionPrefix{\relax }%\bf Phần \Roman{sectionindex}. \space}
\tieudetracnghiem
%\tieudethiviet
\tieudedapan
%\tieudetren
\tieudeduoi
\daungoac{}{.}                  %Dấu quanh phương án trả lời: {(}{)};{}{.};{}{)}
%\chuphuongan{\alph}    %Ký tự cho các phương án
%\chuphuongan{\arabic} %\Roman%\roman%kể cả số cho các phương án
\chucauhoi{Câu}                %Chữ trước các số câu hỏi
\mauchu{red}                     %Mầu số câu hỏi và phương án
\setlength{\baselineskip}{12truept}
\def\v#1{\overrightarrow{#1}} %Làm vectơ
\graphicspath{{hinh-cauhoi/}} %Đường dẫn của nơi để hình
\khoanh{\cbox}         %Khoanh các phương án: \cbox, \fbox
\hovaten{Họ và tên}         %Nếu không muốn có dòng này không gõ lệnh
 \tenlop{Tên lớp}         %Nếu không muốn có dòng này không gõ lệnh
\sobaodanh{Số báo danh}  %Nếu không muốn có dòng này không gõ lệnh
%\ketqua{}          %In ra phần Kết quả
%\giamkhao{}     %In ra phần chữ ký giám khảo ở phiếu thi
%\NoRearrange  %Lệnh không trộn đề
 \motphieuthi      %In ra một phiếu thi, Mặc định là không hiện ra phiếu thi
%\nhieuphieuthi   %In ra mỗi đề một phiếu thi
%\coloigiai           %In ra đáp án có lời giải
\ShortKey             %Lệnh hiện ra đáp án mỗi đề thi
%\OneKey            %Lệnh chỉ in ra 1 bản đáp án
%\NoKey               %Lệnh không in ra phần đáp án

\tentruong{BỘ GIÁO DỤC VÀ ĐÀO TẠO}
\tenkhoa{ĐỀ MINH HỌA}
\loaidethi{Đề gồm có \pageref{LastPage} trang}%{ĐỀ THI LẠI}%%{ĐỀ CHÍNH THỨC}
\tenkythi{KÌ THI TRUNG HỌC PHỔ THÔNG QUỐC GIA NĂM 2017}
\tenmonhoc{Bài thi: Khoa học tự nhiên;  Môn: LỊCH SỬ}
\madethi{100}
\thoigian{\underline{Thời gian làm bài: 50 phút, không kể thời gian phát đề}}
%\soanthao %Lệnh dùng khi soạn cauu hỏi, khi dịch 1 bản và không đảo
\begin{document}

\setlength{\baselineskip}{12truept}
 \begin{vnmultiplechoice}[ rearrange=yes, keycolumns=3]%

\begin{question} %%01
Hội nghị Ianta (2-1945) diễn ra khi cuộc Chiến tranh thế giới thứ hai
\datcot[4]
\bonpa
{\dung{bước vào giai đoạn kết thúc.}}
{\sai{đã hoàn toàn kết thúc.}}
{\sai{đang diễn ra vô cùng ác liệt.}}
{\sai{bùng nổ và ngày càng lan rộng.}}
\end{question}

\begin{question} %%02
Một trong những mục đích của tổ chức Liên hợp quốc là
\datcot[4]
\bonpa
{\dung{duy trì hòa bình và an ninh thế giới.}}
{\sai{trừng trị các hoạt động gây chiến tranh.}}
{\sai{thúc đẩy quan hệ thương mại tự do.}}
{\sai{ ngăn chặn tình trạng ô nhiễm môi trường.}}
\end{question}

\begin{question} %%03
 Chính sách đối ngoại của Liên bang Nga từ năm 1991 đến năm 2000 là ngả về
phương Tây, khôi phục và phát triển quan hệ với các nước ở
\datcot
\bonpa
{\dung{ châu Á.}}
{\sai{châu Âu.}}
{\sai{châu Phi.}}
{\sai{châu Mĩ.}}
\end{question}

\begin{question} %%04
Sự kiện nào dưới đây được xem là sự kiện khởi đầu cuộc“Chiến tranh lạnh”?
\datcot[4]
\bonpa
{\dung{Thông điệp của Tổng thống Mĩ Truman.}}
{\sai{ Đạo luật viện trợ nước ngoài của Quốc hội Mĩ.}}
{\sai{Diễn văn của ngoại trưởng Mĩ Macsan.}}
{\sai{Chiến lược toàn cầu của Tổng thống Mĩ Rudơven.}}
\end{question}

\begin{question} %%05
Những quốc gia Đông Nam Á tuyên bố độc lập trong năm 1945 là
\datcot[4]
\bonpa
{\dung{Inđônêxia, Việt Nam, Lào.}}
{\sai{Campuchia, Malaixia, Brunây.}}
{\sai{Inđônêxia, Xingapo, Malaixia.}}
{\sai{Miến Điện, Việt Nam, Philippin.}}
\end{question}

\begin{question} %%06
Miến Điện, Việt Nam, Philippin.
\datcot[2]
\bonpa
{\dung{Đơn cực.}}
{\sai{Đa cực.}}
{\sai{ Một cực nhiều trung tâm.}}
{\sai{Đa cực nhiều trung tâm.}}
\end{question}

\begin{question} %%07
Xu thế toàn cầu hoá trên thế giới là hệ quả của
\datcot
\bonpa
{\dung{cuộc cách mạng khoa học – công nghệ.}}
{\sai{sự phát triển quan hệ thương mại quốc tế.}}
{\sai{sự ra đời của các công ty xuyên quốc gia.}}
{\sai{quá trình thống nhất thị trường thế giới.}}
\end{question}

\begin{question} %%08
Đặc điểm lớn nhất của cuộc cách mạng khoa học- kĩ thuật sau Chiến tranh thế
giới thứ hai là
\datcot[4]
\bonpa
{\dung{khoa học trở thành lực lượng sản xuất trực tiếp.}}
{\sai{kĩ thuật trở thành lực lượng sản xuất trực tiếp.}}
{\sai{sự bùng nổ của các lĩnh vực khoa học - công nghệ.}}
{\sai{mọi phát minh kĩ thuật đều bắt nguồn từ sản xuất.}}
\end{question}

\begin{question} %%09
Tờ báo nào dưới đây là của tiểu tư sản trí thức ở Việt Nam giai đoạn 1919-
1925?
\datcot
\bonpa
{\dung{Người nhà quê.}}
{\sai{Tin tức.}}
{\sai{Tiền phong.}}
{\sai{Dân chúng.}}
\end{question}

\begin{question} %%10
Tư tưởng cốt lõi trong Cương lĩnh chính trị đầu tiên của Đảng Cộng sản Việt
Nam là
\datcot[2]
\bonpa
{\dung{độc lập và tự do.}}
{\sai{tự do và dân chủ.}}
{\sai{ruộng đất cho dân cày.}}
{\sai{đoàn kết với cách mạng thế giới.}}
\end{question}
%%%%%%%%%%%%%%%%%%%%%%
\begin{question} %%11
 Luận cương chính trị (10-1930) của Đảng Cộng sản Đông Dương xác định giai
cấp lãnh đạo cách mạng là
\datcot
\bonpa
{\dung{công nhân.}}
{\sai{nông dân.}}
{\sai{tư sản dân tộc.}}
{\sai{tiểu tư sản trí thức.}}
\end{question}

\begin{question} %%12
Cuộc khởi nghĩa Yên Bái (2-1930) do tổ chức nào dưới đây lãnh đạo?
\datcot
\bonpa
{\dung{Việt Nam Quốc dân Đảng.}}
{\sai{Đảng Thanh niên.}}
{\sai{Đảng Lập hiến.}}
{\sai{Việt Nam nghĩa đoàn.}}
\end{question}

\begin{question} %%13
Cho các sự kiện sau:\\
1. Nhật đầu hàng Đồng minh không điều kiện.\\
2. Quân Nhật vượt biên giới Việt-Trung, tiến vào miền Bắc Việt Nam.\\
3. Nhật đảo chính lật đổ Pháp ở Đông Dương.\\
Hãy sắp xếp các sự kiện trên theo đúng trình tự thời gian.
\datcot
\bonpa
{\dung{2, 3 ,1.}}
{\sai{1, 2, 3.}}
{\sai{3, 2, 1.}}
{\sai{1, 3, 2.}}
\end{question}

\begin{question} %%14
Khó khăn nghiêm trọng nhất của nước Việt Nam Dân chủ Cộng hòa sau Cách
mạng tháng Tám năm 1945 là
\datcot
\bonpa
{\dung{giặc ngoại xâm.}}
{\sai{tài chính.}}
{\sai{nạn đói.}}
{\sai{giặc dốt.}}
\end{question}

\begin{question} %%15
“Chúng ta thà hi sinh tất cả chứ nhất định không chịu mất nước, nhất định
không chịu làm nô lệ...” là lời của Chủ tịch Hồ Chí Minh trong
\datcot[4]
\bonpa
{\dung{Lời kêu gọi Toàn quốc kháng chiến (1946).}}
{\sai{Tuyên ngôn Độc lập của nước Việt Nam Dân chủ Cộng hòa (1945).}}
{\sai{Lời kêu gọi chống Mĩ cứu nước (1966).}}
{\sai{Báo cáo chính trị tại Đại hội lần thứ hai của Đảng (1951).}}
\end{question}

\begin{question} %%16
Chiến dịch nào dưới đây là chiến dịch chủ động tiến công lớn đầu tiên của bộ
đội chủ lực Việt Nam trong cuộc kháng chiến chống thực dân Pháp (1945-1954)?
\datcot[2]
\bonpa
{\dung{Biên giới thu - đông năm 1950.}}
{\sai{Việt Bắc thu - đông năm 1947.}}
{\sai{Điện Biên Phủ năm 1954.}}
{\sai{Thượng Lào năm 1954.}}
\end{question}

\begin{question} %%17
Thắng lợi nào của quân dân Việt Nam ở miền Nam đã buộc Mĩ phải tuyên bố
“phi Mĩ hóa” chiến tranh xâm lược?
\datcot[4]
\bonpa
{\dung{ Cuộc Tổng tiến công và nổi dậy Xuân 1968.}}
{\sai{Cuộc Tiến công chiến lược năm 1972.}}
{\sai{Trận “Điện Biên Phủ trên không” năm 1972.}}
{\sai{Cuộc Tổng tiến công và nổi dậy Xuân 1975.}}
\end{question}

\begin{question} %%18
Hiệp định Sơ bộ (6-3-1946) công nhận nước Việt Nam Dân chủ Cộng hòa là
một quốc gia
\datcot
\bonpa
{\dung{tự do.}}
{\sai{tự trị.}}
{\sai{ tự chủ.}}
{\sai{độc lập.}}
\end{question}

\begin{question} %%19
Ngày 12-12-1946, Ban Thường vụ Trung ương Đảng ra Chỉ thị
\datcot[2]
\bonpa
{\dung{Toàn dân kháng chiến.}}
{\sai{Kháng chiến kiến quốc.}}
{\sai{Kháng chiến toàn diện.}}
{\sai{ Trường kì kháng chiến.}}
\end{question}

\begin{question} %%20
Khi thực dân Pháp mở cuộc tiến công lên Việt Bắc năm 1947, Trung ương
Đảng ra chỉ thị nào?
\datcot[4]
\bonpa
{\dung{Phải phá tan cuộc tiến công mùa đông của giặc Pháp.}}
{\sai{Phải chủ động đón đánh địch ở mọi nơi chúng xuất hiện.}}
{\sai{Chủ động giữ thế phòng ngự chiến lược trên chiến trường.}}
{\sai{Nhanh chóng triển khai lực lượng tiêu diệt sinh lực địch.}}
\end{question}
%%%%%%%%%%%%%%%%%%%%%%%%%
\begin{question} %%21
Chiến thắng Việt Bắc năm 1947 của quân dân Việt Nam đã buộc thực dân
Pháp phải chuyển từ chiến lược đánh nhanh thắng nhanh sang
\datcot
\bonpa
{\dung{đánh lâu dài.}}
{\sai{phòng ngự.}}
{\sai{đánh phân tán.}}
{\sai{đánh tiêu hao.}}
\end{question}

\begin{question} %%22
Phương hướng chiến lược của quân đội và nhân dân Việt Nam trong Đông -
Xuân 1953-1954 là tiến công vào
\datcot[4]
\bonpa
{\dung{ những hướng quan trọng về chiến lược mà địch tương đối yếu.}}
{\sai{vùng đồng bằng Bắc bộ, nơi tập trung quân cơ động của Pháp.}}
{\sai{Điện Biên Phủ, trung tâm điểm của kế hoạch quân sự Nava.}}
{\sai{ toàn bộ các chiến trường ở Việt Nam, Lào và Campuchia.}}
\end{question}

\begin{question} %%23
Đại hội đại biểu toàn quốc lần thứ VI của Đảng Cộng sản Việt Nam xác định
nhiệm vụ trước mắt của kế hoạch 5 năm 1986- 1990 là
\datcot[4]
\bonpa
{\dung{thực hiện mục tiêu của Ba chương trình kinh tế lớn.}}
{\sai{đổi mới toàn diện, đồng bộ về kinh tế và chính trị.}}
{\sai{xây dựng cơ sở vật chất của chủ nghĩa xã hội.}}
{\sai{đẩy mạnh sự nghiệp công nghiệp hoá đất nước.}}
\end{question}

\begin{question} %%24
Nền tảng chính sách đối ngoại của Nhật Bản từ năm 1951 đến năm 2000 là
\datcot[2]
\bonpa
{\dung{liên minh chặt chẽ với Mĩ.}}
{\sai{hướng mạnh về Đông Nam Á.}}
{\sai{hướng về các nước châu Á.}}
{\sai{cải thiện quan hệ với Liên Xô.}}
\end{question}

\begin{question} %%25
 Nhân tố chủ yếu chi phối quan hệ quốc tế trong phần lớn nửa sau thế kỉ XX là
\datcot[2]
\bonpa
{\dung{cục diện “Chiến tranh lạnh”.}}
{\sai{xu thế toàn cầu hóa.}}
{\sai{sự hình thành các liên minh kinh tế.}}
{\sai{sự ra đời các khối quân sự đối lập.}}
\end{question}

\begin{question} %%26
Mâu thuẫn chủ yếu trong xã hội Việt Nam trước Cách mạng tháng Tám năm
1945 là mâu thuẫn giữa
\datcot[4]
\bonpa
{\dung{toàn thể nhân dân với đế quốc xâm lược và phản động tay sai.}}
{\sai{giai cấp vô sản với giai cấp tư sản.}}
{\sai{giai cấp nông dân với giai cấp địa chủ phong kiến.}}
{\sai{nhân dân lao động với thực dân Pháp và các giai cấp bóc lột.}}
\end{question}

\begin{question} %%27
Lí luận nào sau đây đã được cán bộ của Hội Việt Nam Cách mạng Thanh niên
truyền bá vào Việt Nam?
\datcot
\bonpa
{\dung{Lí luận giải phóng dân tộc.}}
{\sai{Lí luận Mác - Lênin.}}
{\sai{Lí luận đấu tranh giai cấp.}}
{\sai{Lí luận cách mạng vô sản.}}
\end{question}

\begin{question} %%28
Điểm mới của Hội nghị tháng 5-1941 so với Hội nghị tháng 11-1939 Ban
Chấp hành Trung ương Đảng Cộng sản Đông Dương là
\datcot[4]
\bonpa
{\dung{giải quyết vấn đề dân tộc trong khuôn khổ từng nước ở Đông Dương.}}
{\sai{thành lập mặt trận thống nhất dân tộc rộng rãi để chống đế quốc.}}
{\sai{đề cao nhiệm vụ giải phóng dân tộc, chống đế quốc và phong kiến.}}
{\sai{tạm gác khẩu hiệu cách mạng ruộng đất, thực hiện giảm tô, giảm tức.}}
\end{question}

\begin{question} %%29
 Sự kiện nào là mốc đánh dấu kết thúc cuộc kháng chiến của dân tộc Việt Nam
chống thực dân Pháp xâm lược (1945-1954)?
\datcot[4]
\bonpa
{\dung{Hiệp định Giơnevơ về Đông Dương được kí kết.}}
{\sai{Thắng lợi của chiến dịch Điện Biên Phủ.}}
{\sai{Bộ đội Việt Nam tiến vào tiếp quản Hà Nội.}}
{\sai{Quân Pháp xuống tàu rút khỏi Hải Phòng.}}
\end{question}

\begin{question} %%30
 Kẻ thù nguy hiểm nhất của nước Việt Nam Dân chủ Cộng hòa trong năm đầu
sau ngày Cách mạng tháng Tám (1945) thành công là
\datcot
\bonpa
{\dung{Thực dân Pháp.}}
{\sai{Phát xít Nhật.}}
{\sai{Đế quốc Anh.}}
{\sai{Trung Hoa Dân Quốc.}}
\end{question}
%%%%%%%%%%%%%%%%%%%%%%%%%%
\begin{question} %%31
Trong thời kì 1954-1975, phong trào nào là mốc đánh dấu bước phát triển của
cách mạng ở miền Nam Việt Nam từ thế giữ gìn lực lượng sang thế tiến công?
\datcot[4]
\bonpa
{\dung{“Đồng khởi”.}}
{\sai{Phá “ấp chiến lược”.}}
{\sai{“Thi đua Ấp Bắc, giết giặc lập công”.}}
{\sai{“Tìm Mỹ mà đánh, lùng ngụy mà diệt”.}}
\end{question}

\begin{question} %%32
 Điều khoản nào của Hiệp định Pari năm 1973 có ý nghĩa quyết định đối với sự
phát triển của cuộc kháng chiến chống Mĩ, cứu nước?
\datcot[4]
\bonpa
{\dung{Hoa Kì rút hết quân viễn chinh và quân các nước đồng minh.}}
{\sai{Hai bên ngừng bắn và giữ nguyên vị trí ở miền Nam.}}
{\sai{Nhân dân miền Nam tự quyết định tương lai chính trị.}}
{\sai{Các bên thừa nhận thực tế ở miền Nam có hai chính quyền.}}
\end{question}

\begin{question} %%33
Điểm chung trong kế hoạch Rơve năm 1949, kế hoạch Đờ Lát đơ Tátxinhi
năm 1950 và kế hoạch Nava năm 1953 là
\datcot[4]
\bonpa
{\dung{kết thúc chiến tranh trong danh dự.}}
{\sai{bảo vệ chính quyền Bảo Đại do Pháp lập ra.}}
{\sai{muốn xoay chuyển cục diện chiến tranh.}}
{\sai{phô trương thanh thế, tiềm lực, sức mạnh.}}
\end{question}

\begin{question} %%34
Yếu tố nào sau đây quyết định sự phát triển của phong trào giải phóng dân tộc
ở các nước châu Á sau Chiến tranh thế giới thứ hai?
\datcot[4]
\bonpa
{\dung{Ý thức độc lập và sự lớn mạnh của các lực lượng dân tộc.}}
{\sai{Sự suy yếu của các nước đế quốc chủ nghĩa phương Tây.}}
{\sai{Thắng lợi của phe Đồng minh trong chiến tranh chống phát xít.}}
{\sai{Hệ thống xã hội chủ nghĩa hình thành và ngày càng phát triển.}}
\end{question}

\begin{question} %%35
Việt Nam có thể rút ra kinh nghiệm gì từ sự phát triển kinh tế của các nước tư
bản sau Chiến tranh thế giới thứ hai để đẩy mạnh sự nghiệp công nghiệp hóa, hiện
đại hóa đất nước?
\datcot[4]
\bonpa
{\dung{Ứng dụng các thành tựu khoa học - kĩ thuật.}}
{\sai{Khai thác và sử dụng hợp lí nguồn tài nguyên.}}
{\sai{Tăng cường xuất khẩu công nghệ phần mềm.}}
{\sai{Nâng cao trình độ tập trung vốn và lao động.}}
\end{question}

\begin{question} %%36
Nhiệm vụ hàng đầu của cách mạng Việt Nam thời kì 1930-1945 là
\datcot[4]
\bonpa
{\dung{đánh đuổi đế quốc xâm lược giành độc lập dân tộc.}}
{\sai{đánh đổ các giai cấp bóc lột giành quyền tự do dân chủ.}}
{\sai{lật đổ chế độ phong kiến giành ruộng đất cho dân cày.}}
{\sai{ lật đổ chế độ phản động thuộc địa, cải thiện dân sinh.}}
\end{question}

\begin{question} %%37
Nhà nước Việt Nam Dân chủ Cộng hòa được thành lập năm 1945 là nhà nước
của
\datcot[2]
\bonpa
{\dung{toàn thể nhân dân.}}
{\sai{công, nông, binh.}}
{\sai{công nhân và nông dân.}}
{\sai{công, nông và trí thức.}}
\end{question}

\begin{question} %%38
Nguyên tắc quan trọng nhất của Việt Nam trong việc kí kết Hiệp định Sơ bộ
(6-3-1946) và Hiệp định Giơnevơ về Đông Dương (21-7-1954) là
\datcot[4]
\bonpa
{\dung{không vi phạm chủ quyền dân tộc.}}
{\sai{phân hóa và cô lập cao độ kẻ thù.}}
{\sai{đảm bảo giành thắng lợi từng bước.}}
{\sai{giữ vững vai trò lãnh đạo của Đảng.}}
\end{question}

\begin{question} %%39
Phong trào dân chủ 1936-1939 ở Việt Nam là một phong trào
\datcot[2]
\bonpa
{\dung{có tính chất dân tộc.}}
{\sai{chỉ có tính dân chủ.}}
{\sai{không mang tính cách mạng.}}
{\sai{không mang tính dân tộc.}}
\end{question}

\begin{question} %%40
Hội nghị lần thứ 15 Ban Chấp hành Trung ương Đảng Lao động Việt Nam (1-
1959) quyết định để nhân dân miền Nam sử dụng bạo lực cách mạng là do
\datcot[4]
\bonpa
{\dung{không thể tiếp tục sử dụng biện pháp hoà bình được nữa.}}
{\sai{các lực lượng vũ trang cách mạng miền Nam đã phát triển.}}
{\sai{Mĩ và chính quyền Sài Gòn phá hoại Hiệp định Giơnevơ.}}
{\sai{đã có lực lượng chính trị và lực lượng vũ trang lớn mạnh.}}
\end{question}

\begin{examclosing}
\centerline{-- HẾT --}
\end{examclosing}
 \end{vnmultiplechoice}
\end{document}
[scribd id=327589426 key=key-oJpAkf8g7Gy0j9JNOeAs mode=scroll]
