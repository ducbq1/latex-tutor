% Tệp mẫu làm đề thi tự luận dựa vào gói lệnh lamdethi.sty
% Tác giả: Nguyên Hữu Điển
% Khoa Toán Cơ Tin học, ĐHKHTN HN, ĐHQGHN
% 334, Nguyễn Trãi, Thanh Xuân, Hà Nội
% huudien@vnu.edu.vn
% Ngày 26/12/2009
\documentclass[12pt]{article}
\usepackage{amsmath,amsxtra,amssymb,latexsym, amscd,amsthm}
\usepackage{graphicx}
\usepackage{picinpar}
\usepackage[utf8]{vietnam}
\usepackage{longtable}%
\usepackage{multicol}%
\usepackage{color}
\usepackage{fancybox}
\usepackage{lastpage}
%%%%%%%%%%%%%%%%%%%%%%%
\usepackage{enumerate}
 \usepackage{shortlst}
\usepackage{tikz}
\usetikzlibrary{arrows}
\usepackage{tkz-tab}
%%%%%%%%%%%%%%%%%%%%%
\usepackage{mathptmx} 
% \usepackage{mathpazo} 
\voffset=-3cm
% \hoffset=-2cm
\textheight 24truecm 
\textwidth 17truecm 
\usepackage[baitap]{dethi}
\tentruong{ĐẠI HỌC KHOA HỌC TỰ NHIÊN}
\tenkhoa{Khoa Toán - Cơ -Tin học}
\loaidethi{Đề gồm có \pageref{LastPage} trang}%{ĐỀ THI LẠI}%%{ĐỀ CHÍNH THỨC}
\tenkythi{ĐỀ THI GIỮA KỲ NĂM HỌC 2016-2017}
\tenmonhoc{Môn: Tối ưu hóa}
\madethi{100}
\thoigian{\underline{Thời gian làm bài: 90 phút, không kể thời gian phát đề}}   
\hovaten{Họ và tên}         %Nếu không muốn có dòng này không gõ lệnh
\tenlop{Tên lớp}         %Nếu không muốn có dòng này không gõ lệnh
\sobaodanh{Số báo danh}  %Nếu không muốn có dòng này không gõ lệnh
% \cornersize*{3.6mm}
%\daungoac{\Ovalbox}{}
%\daungoac{(}{)}%%{[}{]}%Dấu quanh phương án trả lời: {(}{)};{}{.};{}{)}
% \chuphuongan{\small\bfseries\Alph}
\mauchu{blue}
\PSNrandseed{\time}
% \coloigiai
\renewcommand{\solutionname}{\textbf{Lời giải. }}

\PSNrandseed{\time}
\usepackage{centerpage}
\usepackage{systeme}
\sysdelim.. \syseqsep{:}
\setlength{\baselineskip}{16truept}
\graphicspath{{hinh-tuluan/}}
\begin{document}
\thispagestyle{empty}
\lamtieude
\vspace*{1cm}

\loadrandomproblems[dttuluan]{5}{vd01-cauhoi-tuluan}
% \loadrandomproblems[dttuluan]{3}{vd14-cauhoi-gk01-2017}

\indebai
\begin{enumerate}[]
\foreachproblem[dttuluan]{\item\causo\thisproblem}
\end{enumerate}

\vspace*{1cm}
{\bf Chú ý:} Sinh viên không được mang sách giáo trình và vở ghi chép vào phòng thi.


\newpage
\setcounter{page}{1}
\lamtieude
\indebailoigiai
\begin{center}
{\bf ĐỀ BÀI VÀ ĐÁP ÁN }
\end{center}
\begin{enumerate}[]
\foreachproblem[dttuluan]{\item\causo\thisproblem}
\end{enumerate}

\newpage
\setcounter{page}{1}
\lamtieude
\indapanloigiai
\begin{center}
{\bf ĐÁP ÁN }
\end{center}

\begin{enumerate}[]
% \setlength{\itemindent}{3.5em}
% \setcounter{socauhoi}{-1}
\foreachproblem[dttuluan]{\item\causo\thisproblem}
\end{enumerate}

\end{document}
