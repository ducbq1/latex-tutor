% Tệp mẫu làm đề thi trắc nghiệm dựa vào gói lệnh lamdethi.sty
% Tác giả: Nguyên Hữu Điển
% Khoa Toán Cơ Tin học, ĐHKHTN HN, ĐHQGHN
% 334, Nguyễn Trãi, Thanh Xuân, Hà Nội
% huudien@vnu.edu.vn
% Ngày 26/12/2009
%%%%%%%%%%%%%%%%%%%%%%%%%%%%
\documentclass[11pt]{article}
\usepackage{amsmath,amsxtra,amssymb,latexsym, amscd,amsthm}
\usepackage{graphicx}
\usepackage{tikz}
\usetikzlibrary{arrows}
\usepackage{tkz-tab}
\usepackage[utf8]{vietnam}
\usepackage{longtable}%
\usepackage{mathpazo} 

\voffset=-3cm
% \hoffset=-2cm
\textheight 24truecm 
\textwidth 18truecm 
\usepackage[baitap]{dethi}
\tentruong{ĐẠI HỌC KHOA HỌC TỰ NHIÊN}
\tenkhoa{Khoa Toán - Cơ -Tin học}
\loaidethi{Đề gồm có \pageref{DebaiPage} trang}
\tenkythi{ĐỀ THI GIỮA KỲ NĂM HỌC 2016-2017}
\tenmonhoc{Môn: Toán học tính toán}
\madethi{100}
\tieudeduoi
\thoigian{\underline{Thời gian làm bài: 90 phút, không kể thời gian phát đề}}   
\hovaten{Họ và tên}         %Nếu không muốn có dòng này không gõ lệnh
\tenlop{Tên lớp}         %Nếu không muốn có dòng này không gõ lệnh
\sobaodanh{Số báo danh}  %Nếu không muốn có dòng này không gõ lệnh
\khoanh{\cboxv}
\daungoac{\cboxx}{}
\chuphuongan{\small\bfseries\Alph}
\mauchu{blue}
\PSNrandseed{\time}
\usepackage{centerpage}
\graphicspath{{hinh-tracnghiem/}{hinh/}{hinh-cauhoi/}{images/}} 
\parindent 10pt
\usepackage[colorlinks,hyperindex,plainpages=false,unicode]{hyperref}
\usepackage{titledot}



\titlecontents{section}
[3.5em]{}
{\contentslabel{2.5em}}
{}{}
\titlecontents{subsection}
[5.5em]{}
{\contentslabel{3.5em}}
{}{}
\begin{document}
\setlength{\baselineskip}{12truept}
\def\v#1{\overrightarrow{#1}} %Làm vectơ
\renewcommand{\contentsname}{ }
%%%%%%%%%%%%%
% \sotheo{\thesection.}
% \section{Aaa}

\lamtieude
\loadrandomproblems[bttracnghiem]{50}{2017-cauhoi-toan}
% \loadrandomproblems[bttracnghiem]{5}{vd12-cauhoi-lephulu}
\indebailoigiai
\addcontentsline{toc}{section}{Đề thi THPT QG 2016-2017 (Lần 3)}
\begin{center}
\textbf{ĐỀ BÀI LÀM ĐỀ}
\end{center}
\begin{enumerate}[]
% \foreachproblem[bttracnghiem]{\item\causo\thisproblem}
\foreachproblem[bttracnghiem]{\item\causo[\textcolor{red}{\textbf{\thisproblemlabel}}]\addcontentsline{toc}{subsection}{Câu \thesocauhoi.[\thisproblemlabel]}\thisproblem}
 \end{enumerate}
\label{DebaiPage}

\newpage
\setcounter{page}{1}
\loaidethi{(Lời giải  \pageref{magocPage} trang)}
\trangcuoi{magocPage}
\indebai
\lamtieude
\begin{center}
\textbf{ĐỀ BÀI VÀ MÃ GỐC}
\end{center}
\begin{enumerate}[]
\foreachproblem[bttracnghiem]{\item\causo[\textcolor{red}{\textbf{\thisproblemlabel}}]\thisproblem}
 \end{enumerate}
\label{magocPage}
% \begin{enumerate}[]
% \foreachproblem[bttracnghiem]{\item\ref{ab:\thisproblemlabel}\textcolor{red}{\textbf{\thisproblemlabel}}\addcontentsline{toc}{section}{\thisproblemlabel}\thisproblem}
%  \end{enumerate}

\newpage
\thispagestyle{empty}
\lamtieude
\indebai
\begin{center}
\textbf{CÂU HỎI VÀ MÃ GỐC}
\end{center}
\begin{multicols}{3}
\begin{enumerate}[]
\foreachproblem[bttracnghiem]{\item\causo\textcolor{red}{\textbf{\thisproblemlabel}}}
 \end{enumerate}
\end{multicols}

\newpage
\thispagestyle{empty}
\lamtieude
\begin{center}
\textbf{MÃ GỐC LIÊN KẾT VỚI ĐỀ}
\end{center}
\begin{multicols}{3}
\tableofcontents[MÃ ĐỀ THI]
\end{multicols}
\end{document}





\begin{enumerate}[]
\foreachsolution[bttracnghiem]{\item\causo\thisproblem}
 \end{enumerate}
\end{document}
\newpage
\indebailoigiai
\begin{enumerate}[]
\foreachproblem[bttracnghiem]{\item\causo\thisproblem}
 \end{enumerate}

\indapanrutgon
\begin{enumerate}[\causo]
\foreachproblem[bttracnghiem]{\item\thisproblem}
 \end{enumerate}
\end{document}
\newpage
\section{BBBBBBB}
\setcounter{socauhoi}{0}
% \indebailoigiai
% \begin{cauhoi}
% \foreachproblem[bttracnghiem]{\item\causo\thisproblem}
% \end{cauhoi}

\newpage
\setcounter{page}{1}
\lamtieude
\indebaidapan
\begin{center}
{\bf ĐỀ BÀI VÀ ĐÁP ÁN ĐÁNH DẤU}
\end{center}
\begin{cauhoi}
\foreachproblem[bttracnghiem]{\item\causo\thisproblem}
\end{cauhoi}
% 
\newpage
\setcounter{page}{1}
\lamtieude
\indebailoigiai
\begin{center}
{\bf ĐỀ BÀI - ĐÁP ÁN - LỜI GIẢI}
\end{center}
\begin{cauhoi}
\foreachproblem[bttracnghiem]{\item\causo\thisproblem}
\end{cauhoi}



\newpage
% \setcounter{page}{1}
\lamtieude
\indapanrutgon
\thispagestyle{empty}
\begin{center}
{\bf ĐÁP ÁN RÚT GỌN}
\end{center}
\begin{cauhoi}[2]
% \begin{multicols}[2]
 \foreachproblem[bttracnghiem]{\item\thisproblem}
% \end{multicols}
\end{cauhoi}



\newpage
% \setcounter{page}{1}
\indapanlietke
\thispagestyle{empty}
\begin{center}
{\bf ĐÁP ÁN LIỆT KÊ}
\end{center}
\begin{cauhoi}[2]
\begin{multicols}{5}
\foreachproblem[bttracnghiem]{\item\thisproblem}
\end{multicols}
\end{cauhoi}

\end{document}
\vspace*{1cm}
\lamtieude
\indapanso
\begin{center}
{\bf ĐÁP ÁN LIỆT KÊ THEO SỐ}
\end{center}
\begin{center}
\begin{cauhoi}[3]
\begin{multicols}{6}
\foreachproblem[bttracnghiem]{\item\thisproblem}
\end{multicols}
\end{cauhoi}
\end{center}


\newpage
\setcounter{page}{1}
\lamtieude
\begin{center}
{\bf ĐÁP ÁN VÀ LỜI GIẢI}
\end{center}
\indapanloigiai
\begin{cauhoi}
\foreachproblem[bttracnghiem]{\item\causo\thisproblem}
\end{cauhoi}
\end{document}
\newpage
\setcounter{page}{1}
\lamtieude
\indapanloigiaio
\begin{center}
{\bf LỜI GIẢI}
\end{center}
\begin{cauhoi}[2]
\foreachproblem[bttracnghiem]{\item\thisproblem}
\end{cauhoi}

\newpage
\setcounter{page}{1}
\inphieuduclo
\lamphieuthi
\begin{center}
\begin{multicols}{3}
\begin{cauhoi}[2]
\foreachproblem[bttracnghiem]{\item\thisproblem}
\end{cauhoi}
\end{multicols}
\end{center}
\newpage
 
\inphieuthi
\lamphieuthi
\begin{multicols}{3}
\begin{cauhoi}[2]
\foreachproblem[bttracnghiem]{\item\thisproblem}
\end{cauhoi}
\end{multicols}


\addtocounter{page}{-2}


\end{document}
https://drive.google.com/open?id=0B6t6keI_Xwz5dmtjbHlIbjIyc0k