%Tệp mẫu làm đề thi trắc nghiệm phiên bản 3.0
%Tác giả Nguyễn Hữu Điển (ĐHKHTN, Hà Nội)
% Đề trắc nghiệm được thiết kế trên phông Unicode,
%Đã dùng lớp examdesign.cls có sửa đổi
%Cùng với gói lệnh dethi.sty tạo ra:
%Đề thi trắc nghiệm từ một bộ đề sinh ra các câu hởi được 
%sắp xếp ngẫu nhiên và các chi tiết của câu hỏi cũng được 
%xắp sếp ngẫu nhiên. Mỗi đề thi sinh ra đều có thể in ra đáp án riêng biệt.
%examdesign.cls đòi hỏi các gói lệnh enumerate, multicol, shortlst, keyval.
\documentclass[11pt]{article}
\usepackage{amsmath,amsxtra,latexsym, amssymb, amscd}
\usepackage[utf8]{vietnam}
\usepackage{color}
\usepackage{graphicx}
\usepackage{picinpar}
\usepackage{mathptmx} 
 \usepackage{textcomp} 
\usepackage{enumerate}
\usepackage{multicol}
\usepackage{shortlst}
\usepackage[baithi]{dethi} %Gói lệnh cho đề thi Việt Nam
\usepackage{lastpage}
\usepackage{tikz}
\usepackage{fancybox}
\cornersize*{3.6mm}
\Fullpages %Định dạng trang đề thi
\ContinuousNumbering %Đánh số liên tục các bài thi
\NumberOfVersions{1} %10 là số bài thi khác nhau được in ra
\SectionPrefix{\relax }%\bf Phần \Roman{sectionindex}. \space}
\tieudetracnghiem
%\tieudetuluan
\tieudedapan
%\tieudetren
\tieudeduoi
\daungoac{\cboxx}{}                  %Dấu quanh phương án trả lời: {(}{)};{}{.};{}{)}
%\chuphuongan{\alph}    %Ký tự cho các phương án
%\chuphuongan{\arabic} %\Roman%\roman%kể cả số cho các phương án
\chucauhoi{Câu}                %Chữ trước các số câu hỏi
\mauchu{blue}                     %Mầu số câu hỏi và phương án
\setlength{\baselineskip}{12truept}
\def\v#1{\overrightarrow{#1}} %Làm vectơ
\graphicspath{{hinh-cauhoi/}} %Đường dẫn của nơi để hình
\def\kbox#1{{\Large\textcircled{\normalsize #1}}}


% \usepackage{harmony}
\khoanh{\cboxx}         %Khoanh các phương án: \cbox, \fbox
\hovaten{Họ và tên}         %Nếu không muốn có dòng này không gõ lệnh
% \tenlop{Tên lớp}         %Nếu không muốn có dòng này không gõ lệnh
\sobaodanh{Số báo danh}  %Nếu không muốn có dòng này không gõ lệnh
%\ketqua{}          %In ra phần Kết quả
%\giamkhao{}     %In ra phần chữ ký giám khảo ở phiếu thi
% \NoRearrange  %Lệnh không trộn đề
\motphieuthi      %In ra một phiếu thi, Mặc định là không hiện ra phiếu thi
%  \nhieuphieuthi   %In ra mỗi đề một phiếu thi
%  \coloigiai           %In ra đáp án có lời giải\\
\ShortKey             %Lệnh hiện ra đáp án mỗi đề thi
% \OneKey            %Lệnh chỉ in ra 1 bản đáp án
% \NoKey               %Lệnh không in ra phần đáp án
\tentruong{BỘ GIÁO DỤC VÀ ĐÀO TẠO}
\tenkhoa{ĐỀ MINH HỌA}
\loaidethi{Đề gồm có \pageref{LastPage} trang}%{ĐỀ THI LẠI}%%{ĐỀ CHÍNH THỨC}
\tenkythi{KÌ THI TRUNG HỌC PHỔ THÔNG QUỐC GIA NĂM 2017}
\tenmonhoc{Môn: Toán}
\madethi{100}
\thoigian{\underline{Thời gian làm bài: 90 phút, không kể thời gian phát đề}}
% \everymath{\displaystyle} 
% \soanthao

\begin{document}

\setlength{\baselineskip}{12truept}
% \setlength{\shortitemwidth}{0.20\textwidth}
% \loadrandomproblems[dtracnghiem]{12}{cauhoi02-dtracnghiem} %PA(1)
 \begin{vnmultiplechoice}[title={\bf I. Các câu hỏi Địa lý}, keycolumns=3]%

\begin{question} %%01
Nước Việt Nam nằm ở
\datcot[4]
\bonpa
{\sai{bán đảo Trung Ấn, khu vực cận nhiệt đới.}}
{\sai{phía đông Thái Bình Dương, khu vực kinh tế sôi động của thế giới.}}
{\sai{rìa phía đông châu Á, khu vực ôn đới.}}
{\dung{rìa phía đông bán đảo Đông Dương, gần trung tâm Đông Nam Á.}}
\end{question}

\begin{question} %%02
 Lãnh thổ Việt Nam là khối thống nhất và toàn vẹn, bao gồm
\datcot[2]
\bonpa
{\sai{vùng đất, vùng biển, vùng núi.}}
{\sai{vùng đất, hải đảo, thềm lục địa.}}
{\sai{vùng đất liền, hải đảo, vùng trời.}}
{\dung{vùng đất, vùng biển, vùng trời.}}
\end{question}

\begin{question} %%03
Đặc điểm nào sau đây chứng tỏ Việt Nam là đất nước nhiều đồi núi?
\datcot[4]
\bonpa
{\sai{Cấu trúc địa hình khá đa dạng.}}
{\sai{Địa hình thấp dần từ tây bắc xuống đông nam.}}
{\sai{Địa hình núi cao chiếm 1\% diện tích lãnh thổ.}}
{\dung{Địa hình đồi núi chiếm 3/4 diện tích lãnh thổ.}}
\end{question}

\begin{question} %%04
Đặc điểm đô thị hoá ở nước ta là
\datcot[2]
\bonpa
{\sai{tỉ lệ dân thành thị giảm.}}
{\sai{phân bố đô thị đều giữa các vùng.}}
{\sai{quá trình đô thị hoá diễn ra nhanh.}}
{\dung{trình độ đô thị hoá thấp.}}
\end{question}

\begin{question} %%05
Vùng sản xuất lương thực lớn nhất nước ta là
\datcot[2]
\bonpa
{\sai{Đồng bằng sông Hồng.}}
{\sai{Bắc Trung Bộ.}}
{\sai{Duyên hải Nam Trung Bộ.}}
{\dung{Đồng bằng sông Cửu Long.}}
\end{question}

\begin{question} %%06
Vùng nào sau đây có nghề nuôi cá nước ngọt phát triển mạnh nhất ở nước ta?
\datcot[4]
\bonpa
{\sai{Đông Nam Bộ và Đồng bằng sông Cửu Long.}}
{\sai{Đồng bằng sông Hồng và Bắc Trung Bộ.}}
{\sai{Bắc Trung Bộ và Đông Nam Bộ.}}
{\dung{Đồng bằng sông Cửu Long và Đồng bằng sông Hồng.}}
\end{question}

\begin{question} %%07
Ngành nào sau đây \textbf{không được} xem là ngành công nghiệp trọng điểm của nước ta hiện nay?
\datcot[2]
\bonpa
{\sai{Năng lượng.}}
{\sai{Chế biến lương thực, thực phẩm.}}
{\sai{Dệt - may.}}
{\dung{Luyện kim.}}
\end{question}

\begin{question} %%08
Cây công nghiệp quan trọng số một của Tây Nguyên là
\datcot
\bonpa
{\sai{chè.}}
{\sai{hồ tiêu.}}
{\sai{cao su.}}
{\dung{cà phê.}}
\end{question}

\begin{question} %%09
Loại đất nào sau đây chiếm diện tích lớn nhất ở Đồng bằng sông Cửu Long?
\datcot
\bonpa
{\sai{Đất phù sa ngọt.}}
{\sai{Đất mặn.}}
{\sai{Đất xám.}}
{\dung{Đất phèn.}}
\end{question}

\begin{question} %%10
Điều kiện nào sau đây của vùng biển nước ta thuận lợi để phát triển giao thông vận tải biển?
\datcot[4]
\bonpa
{\sai{Có nhiều bãi tắm rộng, phong cảnh đẹp, khí hậu tốt.}}
{\sai{Các hệ sinh thái vùng ven biển rất đa dạng và giàu có.}}
{\sai{Có nhiều sa khoáng với trữ lượng công nghiệp.}}
{\dung{Nằm gần các tuyến hàng hải quốc tế trên Biển Đông.}}
\end{question}
%%%%%%%%%%%%%%%%%%%%%%
\begin{question} %%11
Căn cứ vào Atlat Địa lí Việt Nam trang 4 - 5, hãy cho biết trong số 7 tỉnh biên giới trên đất liền
giáp với Trung Quốc, \textbf{không có} tỉnh nào sau đây?
\datcot
\bonpa
{\sai{Lạng Sơn. }}
{\sai{Cao Bằng.}}
{\sai{Hà Giang.}}
{\dung{Tuyên Quang.}}
\end{question}

\begin{question} %%12
Căn cứ vào Atlat Địa lí Việt Nam trang 15, hãy cho biết các đô thị nào sau đây là đô thị đặc biệt
ở nước ta?
\datcot
\bonpa
{\sai{Hà Nội, Cần Thơ.}}
{\sai{TP. Hồ Chí Minh, Hải Phòng.}}
{\sai{TP. Hồ Chí Minh, Đà Nẵng.}}
{\dung{Hà Nội, TP. Hồ Chí Minh.}}
\end{question}

\begin{question} %%13
Căn cứ vào Atlat Địa lí Việt Nam trang 17, hãy cho biết khu kinh tế ven biển nào dưới đây
\textbf{không thuộc} Bắc Trung Bộ?
\datcot
\bonpa
{\sai{Vũng Áng.}}
{\sai{Nghi Sơn.}}
{\sai{Hòn La.}}
{\dung{Chu Lai.}}
\end{question}

\begin{question} %%14
Căn cứ vào Atlat Địa lí Việt Nam trang 25, các trung tâm du lịch có ý nghĩa vùng của Trung du
và miền núi Bắc Bộ là
\datcot
\bonpa
{\sai{Hạ Long, Thái Nguyên.}}
{\sai{Hạ Long, Điện Biên Phủ.}}
{\sai{Thái Nguyên, Việt Trì.}}
{\dung{Hạ Long, Lạng Sơn.}}
\end{question}

 \end{vnmultiplechoice}
% \loadrandomproblems[ktracnghiem]{12}{cauhoi02-ktracnghiem} %PA(1)

 \begin{vnmultiplechoice}[title={\examvspace*{0.5cm}\bf II. Các câu hỏi Toán học}, keycolumns=3]%
\begin{block}
\begin{question}
Tập xác định của hàm số $y=\dfrac{\sqrt{x^2-5x+6}}{x+2}$ là:
\datcot
\bapa
{\sai{$R\setminus \{ 3;2;-2\}$}}
{\sai{$R\setminus [2;3]$}}
{\dung{$(-\infty ,2]\cup [3,+\infty)\setminus \{-2\}$}}
\end{question}
\begin{question}
Tập xác định của hàm số $y=\dfrac{\sqrt{x^2-5x+6}}{x+2}$ là:
\datcot
\haipa
{\dung{$R\setminus \{ 3;2;-2\}$}}
{\sai{$R\setminus [2;3]$}}
\end{question}
\begin{question}
Hàm số $y=x^3-3x-4$ đồng biến trên miền nào dưới đây:
\datcot
\bapa
{\sai{$(-\infty ,-1)\cup (1, +\infty)$}}
{\dung{$(-\infty ,-1)$ và $(1, +\infty)$}}
{\sai{$R\setminus [-1;1]$ }}
\end{question} 
\end{block}
%\foreachproblem[ktracnghiem]{\thisproblem} %P(1)
\selectrandomly{vd02-cauhoi-ktracnghiem}{20} %P(2)
%  \selectallproblems{cauhoi02-ktracnghiem} %P(3)
%  \selectallproblems{01cauhoi-toan-2017} %P(3)

\begin{examclosing}
\centerline{-- HẾT --}
\end{examclosing}

 \end{vnmultiplechoice}
\end{document}
[scribd id=330151056 key=key-JLeLWjI0GmisBwp14lnc mode=scroll]
