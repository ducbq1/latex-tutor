% Tệp mẫu làm đề thi trắc nghiệm dựa vào gói lệnh lamdethi.sty
% Tác giả: Nguyên Hữu Điển
% Khoa Toán Cơ Tin học, ĐHKHTN HN, ĐHQGHN
% 334, Nguyễn Trãi, Thanh Xuân, Hà Nội
% huudien@vnu.edu.vn
% Ngày 26/12/2009
%%%%%%%%%%%%%%%%%%%%%%%%%%%%
\documentclass[12pt]{article}
\usepackage{amsmath,amsxtra,amssymb,latexsym, amscd,amsthm}
\usepackage[utf8]{vietnam}
\usepackage{longtable}%
%%%%%%%%%%%%%%%%%%%%%
\usepackage{mathpazo} 
\voffset=-2cm
% \hoffset=-2cm
\textheight 24truecm 
\textwidth 18truecm 
\usepackage{centerpage}
\usepackage[baitap]{dethi}
\usepackage[bookmarksnumbered, colorlinks,hyperindex, unicode]{hyperref}%
\usepackage{titledot}
\usepackage{fancyhdr}
\pagestyle{fancy}
\renewcommand{\sectionmark}[1]%
           {\markright{\it \thesection\ #1}}
\lhead[\fancyplain{}{
     \footnotesize{Page \thepage}}]
{\fancyplain{}{ https://nhdien.wordpress.com - {\it Nguyễn Hữu Điển}}}
\rhead[\fancyplain{}{\leftmark}]%
   {\fancyplain{}{\footnotesize{Trang số \thepage}}}
\cfoot{}
\sloppy
\tentruong{ĐẠI HỌC KHOA HỌC TỰ NHIÊN}
\tenkhoa{Khoa Toán - Cơ -Tin học}
\loaidethi{Đề gồm có \pageref{debaiPage} trang}%{ĐỀ THI LẠI}%%{ĐỀ CHÍNH THỨC}
\tenkythi{ĐỀ THI GIỮA KỲ NĂM HỌC 2016-2017}
\tenmonhoc{Môn: Toán học tính toán}
\madethi{100}
\thoigian{\underline{Thời gian làm bài: 90 phút, không kể thời gian phát đề}}   
\hovaten{Họ và tên}         %Nếu không muốn có dòng này không gõ lệnh
\tenlop{Tên lớp}         %Nếu không muốn có dòng này không gõ lệnh
\sobaodanh{Số báo danh}  %Nếu không muốn có dòng này không gõ lệnh
\tieudeduoi
\khoanh{\cboxv}
\daungoac{\cboxx}{}
% \khoanh{\cbox}
%\daungoac{(}{)}%%{[}{]}%Dấu quanh phương án trả lời: {(}{)};{}{.};{}{)}
\chuphuongan{\small\bfseries\Alph}
\mauchu{blue}
\graphicspath{{hinh-cauhoi/}} 

\begin{document}
\title{\bf PHIẾU CHẤM ĐỤC LỖ TRONG DETHI.STY 3.3} % Ten bai
\author{{\bf Nguyễn Hữu Điển}\\
Khoa Toán - Cơ - Tin học\\
ĐHKHTN Hà Nội, ĐHQGHN
} % Tac gia
\date{} % Ngay

\maketitle
\vspace*{1cm}

\tableofcontents
\setlength{\baselineskip}{12truept}
\def\v#1{\overrightarrow{#1}} %Làm vectơ
%%%%%%%%%%%%%

 \setlength{\shortitemwidth}{\textwidth/4-3em}
 \setlength{\runitemsep}{1ex}
 \setlength{\labelsep}{4pt}

\loadrandomproblems[bttracnghiem]{5}{2017-cauhoi-toan}

% \indebai
% \indebaidapan
% \indebailoigiai
% \indapanrutgon
% \indapanlietke
% \indapanso
% \indapanchu
% \indapanloigiaio
% \indapanloigiai
% \inphieuduclo
% \inphieuthi
%%%---------
\newpage
\section{Đề bài}
\indebai
\trangcuoi{debaiPage}
\lamtieude
\begin{enumerate}[]
\foreachproblem[bttracnghiem]{\item\causo\thisproblem}
\end{enumerate}
\label{debaiPage}

\newpage
\section{Đề bài đáp án đánh dấu}
\setcounter{page}{1}
\indebaidapan
\loaidethi{Đề gồm có \pageref{danhdauPage} trang}
\trangcuoi{danhdauPage}
\lamtieude
\begin{center}
{\bf ĐỀ BÀI VÀ ĐÁP ÁN ĐÁNH DẤU}
\end{center}
\begin{enumerate}[]
\foreachproblem[bttracnghiem]{\item\causo\thisproblem}
\end{enumerate}
\label{danhdauPage}
%%%%%%%%%%%%%%%%%%%%%%
\newpage
\section{Đề bài đáp án lời giải}
\setcounter{page}{1}
\indebailoigiai
\loaidethi{Đề gồm có \pageref{dblgPage} trang}
\trangcuoi{dblgPage}
\lamtieude
\begin{center}
{\bf ĐỀ BÀI - ĐÁP ÁN - LỜI GIẢI}
\end{center}
\begin{enumerate}[]
\foreachproblem[bttracnghiem]{\item\causo\thisproblem}
\end{enumerate}
\label{dblgPage}
%%%%%%%%%%%%%%%%%%%%%%

\newpage

\section{Đáp án rút gọn}
\thispagestyle{empty}
\indapanrutgon
\lamtieude
\begin{center}
{\bf ĐÁP ÁN RÚT GỌN}
\end{center}
\begin{multicols}{3}
\begin{enumerate}[\causo]
\foreachproblem[bttracnghiem]{\item\thisproblem}
\end{enumerate}
\end{multicols}
%%%%%%%%%%%%%%%%%%%%%%
\newpage
\section{Đáp án lời giải liệt kê}
\thispagestyle{empty}
\setcounter{page}{1}
\indapanlietke
\lamtieude
\thispagestyle{empty}
\begin{center}
{\bf ĐÁP ÁN LIỆT KÊ}
\end{center}
\begin{multicols}{3}
\begin{enumerate}[\causo]
\foreachproblem[bttracnghiem]{\item\thisproblem}
\end{enumerate}
\end{multicols}
%%%%%%%%%%%%%%%%%%%%%%

\newpage
\section{Đáp án theo số}
\thispagestyle{empty}
\setcounter{page}{1}
\khoanh{\cboxv}
\indapanso
\lamtieude
\begin{center}
{\bf ĐÁP ÁN THEO SỐ}
\end{center}
\chucauhoi{ }
\begin{center}
\begin{multicols}{6}
\begin{enumerate}[\causo]
\foreachproblem[bttracnghiem]{\item\thisproblem}
\end{enumerate}
\end{multicols}
\end{center}
\chucauhoi{Câu }
%%%%%%%%%%%%%%%%%%%%%%
\newpage
\section{Đáp án theo chứ}
\thispagestyle{empty}
\indapanchu
\lamtieude
\begin{center}
{\bf ĐÁP ÁN  THEO CHỮ}
\end{center}
\begin{center}
\begin{multicols}{10}
\begin{enumerate}[]
\foreachproblem[bttracnghiem]{\item\thisproblem}
\end{enumerate}
\end{multicols}
\end{center}
%%%%%%%%%%%%%%%%%%%%%%
\newpage
\section{Lời giải}
\setcounter{page}{1}
\inloigiai
\lamtieude
\begin{center}
{\bf LỜI GIẢI}
\end{center}
\begin{enumerate}[]
\foreachproblem[bttracnghiem]{\item\thisproblem}
\end{enumerate}
\newpage
\section{Làm phiếu đục lỗ}
\setcounter{page}{1}
\inphieuduclo
\lamphieuthi
\begin{center}
\begin{multicols}{3}
\begin{enumerate}[\causo]
\foreachproblem[bttracnghiem]{\item\thisproblem}
\end{enumerate}
\end{multicols}
\end{center}
%%%%%%%%%%%%%%%%%%%%%%
\newpage
\section{Làm phiếu phiếu thi}
\inphieuthi
\lamphieuthi
\begin{center}
\begin{multicols}{3}
\begin{enumerate}[\causo]
\foreachproblem[bttracnghiem]{\item\thisproblem}
\end{enumerate}
\end{multicols}
\end{center}
 

\end{document}
