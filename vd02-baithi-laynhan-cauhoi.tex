%Tệp mẫu làm đề thi trắc nghiệm phiên bản 3.0
%Tác giả Nguyễn Hữu Điển (ĐHKHTN, Hà Nội)
% Đề trắc nghiệm được thiết kế trên phông Unicode,
%Đã dùng lớp examdesign.cls có sửa đổi
%Cùng với gói lệnh dethi.sty tạo ra:
%Đề thi trắc nghiệm từ một bộ đề sinh ra các câu hởi được 
%sắp xếp ngẫu nhiên và các chi tiết của câu hỏi cũng được 
%xắp sếp ngẫu nhiên. Mỗi đề thi sinh ra đều có thể in ra đáp án riêng biệt.
%examdesign.cls đòi hỏi các gói lệnh enumerate, multicol, shortlst, keyval.twocolumn
\documentclass[11pt]{article}
\usepackage{amsmath,amsxtra,latexsym, amssymb, amscd}
\usepackage[utf8]{vietnam}
\usepackage{color}
\usepackage{graphicx}
\usepackage{tikz}
\usepackage{picinpar}
\usepackage{mathptmx} 
% \usepackage{mathpazo} 
\usepackage{enumerate}
\usepackage{multicol}
\usepackage{shortlst}
\usepackage[baithi]{dethi} %Gói lệnh cho đề thi Việt Nam
\usepackage{lastpage}
% \usepackage{fancybox}
% \cornersize*{3.6mm}
\Fullpages %Định dạng trang đề thi
\ContinuousNumbering %Đánh số liên tục các bài thi
\NumberOfVersions{1} %10 là số bài thi khác nhau được in ra
\SectionPrefix{\relax }%\bf Phần \Roman{sectionindex}. \space}
\tieudetracnghiem
\tieudedapan
%\tieudetren
\tieudeduoi
\mauchu{blue}                     %Mầu số câu hỏi và phương án
\daungoac{\cboxx}{}          %Dấu quanh phương án trả lời: {(}{)};{}{.};{}{)}
\khoanh{\cboxx}         %Khoanh các phương án: \cbox, \fbox
\chuphuongan{\bf\Alph}    %Ký tự cho các phương án
%\chuphuongan{\arabic} %\Roman%\roman%kể cả số cho các phương án
\chucauhoi{Câu}                %Chữ trước các số câu hỏi
% \chucauhoi{ }                %Chữ trước các số câu hỏi
\def\v#1{\overrightarrow{#1}} %Làm vectơ
\graphicspath{{hinh-cauhoi/}} %Đường dẫn của nơi để hình
\hovaten{Họ và tên}         %Nếu không muốn có dòng này không gõ lệnh
% \tenlop{Tên lớp}         %Nếu không muốn có dòng này không gõ lệnh
\sobaodanh{Số báo danh}  %Nếu không muốn có dòng này không gõ lệnh
%\ketqua{}          %In ra phần Kết quả
%\giamkhao{}     %In ra phần chữ ký giám khảo ở phiếu thi
%\NoRearrange  %Lệnh không trộn đề
% \motphieuthi      %In ra một phiếu thi, Mặc định là không hiện ra phiếu thi
%  \nhieuphieuthi   %In ra mỗi đề một phiếu thi
% \coloigiai           %In ra đáp án có lời giải
% \lietkedatrue
 \ShortKey             %Lệnh hiện ra đáp án mỗi đề thi
% \OneKey            %Lệnh chỉ in ra 1 bản đáp án
% \NoKey               %Lệnh không in ra phần đáp án

\tentruong{BỘ GIÁO DỤC VÀ ĐÀO TẠO}
\tenkhoa{ĐỀ MINH HỌA}
\loaidethi{Đề gồm có \pageref{LastPage} trang}%{ĐỀ THI LẠI}%%{ĐỀ CHÍNH THỨC}
\tenkythi{KÌ THI TRUNG HỌC PHỔ THÔNG QUỐC GIA NĂM 2017}
\tenmonhoc{Môn: Toán}
\madethi{100}
\thoigian{\underline{Thời gian làm bài: 90 phút, không kể thời gian phát đề}}
\usepackage{centerpage}
\usepackage{balance}
% \soanthao
\begin{document}
% \setlength{\columnsep}{1cm}
\setlength{\baselineskip}{12truept}
% \setlength{\shortitemwidth}{0.20\textwidth}
% \loadrandomproblems[tracnghiem]{12}{cauhoi-toan-2017} %PA(1)
\loadselectedproblems[bttn]{t2017:b10,t2017:b15}{2017-cauhoi-toan}
 \begin{vnmultiplechoice}[title={\textbf{Lấy theo nhãn chỉ định sẵn vào bộ nhớ}},keycolumns=3]%
\useproblem[bttn]{t2017:b10}
\useproblem[bttn]{t2017:b15}
 \end{vnmultiplechoice}
%%%%%%%%%%%%%%%%
\def\dsnhan{t2017:b20,t2017:b21,t2017:b22}
\loadselectedproblems[bttnnew]{\dsnhan}{2017-cauhoi-toan}
 \begin{vnmultiplechoice}[title={\textbf{Lấy 3 câu hỏi theo danh sách nhãn}},keycolumns=3]%
\foreachproblem[bttnnew]{\thisproblem}
 \end{vnmultiplechoice}
%%%%%%%%%%%%%%%%%
\loadrandomproblems[tracnghiem]{5}{vd02-cauhoi-ktracnghiem} 
 \begin{vnmultiplechoice}[title={\textbf{Lấy vào bộ nhớ 5 câu hỏi sau mới dùng}}, keycolumns=3]%
\foreachproblem[tracnghiem]{\thisproblem}
 \end{vnmultiplechoice}
%%%%%%%%%%%%%%%%%%%%%
 \begin{vnmultiplechoice}[title={\textbf{Lấy số lượng ngẫu nhiên 5 câu hỏi}},keycolumns=3]%
\selectrandomly{vd02-cauhoi-dtracnghiem}{5} 
 \end{vnmultiplechoice}
%%%%%%%%%%%%
 \begin{vnmultiplechoice}[title={\textbf{Lấy toàn bộ câu hỏi}}, keycolumns=3]%
 \selectallproblems{vd02-cauhoi-tracnghiem-gachduoi} 
%%%%%%%%%%%%%
% \foreachproblem[tracnghiem]{\thisproblem} %P(1)
%  \selectrandomly{01cauhoi-toan-2017}{55} %P(2)
%  \selectallproblems{01cauhoi-toan-2017} %P(3)
\begin{examclosing}
\centerline{-- HẾT --}
\end{examclosing}
 \end{vnmultiplechoice}
\end{document}
[scribd id=329964583 key=key-349ghO1NhysH3lvPnPbC mode=scroll]
