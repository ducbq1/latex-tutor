% Tệp mẫu làm đề thi trắc nghiệm dựa vào gói lệnh lamdethi.sty
% Tác giả: Nguyên Hữu Điển
% Khoa Toán Cơ Tin học, ĐHKHTN HN, ĐHQGHN
% 334, Nguyễn Trãi, Thanh Xuân, Hà Nội
% huudien@vnu.edu.vn
% Ngày 26/12/2009
%%%%%%%%%%%%%%%%%%%%%%%%%%%%

\documentclass[11pt]{article}
\usepackage{amsmath,amsxtra,amssymb,latexsym, amscd,amsthm}
\usepackage[utf8]{vietnam}
\usepackage{longtable}%
\usepackage{mathpazo} 
\voffset=-3cm
% \hoffset=-2cm
\textheight 24truecm 
\textwidth 19truecm 
\usepackage[baitap]{dethi}
\tentruong{ĐẠI HỌC KHOA HỌC TỰ NHIÊN}
\tenkhoa{Khoa Toán - Cơ -Tin học}
\loaidethi{Đề gồm có \pageref{dbPage} trang}
\trangcuoi{dbPage}
\tenkythi{ĐỀ THI GIỮA KỲ NĂM HỌC 2016-2017}
\tenmonhoc{Môn: Toán học tính toán}
\tieudeduoi
\thoigian{\underline{Thời gian làm bài: 90 phút, không kể thời gian phát đề}}   
\hovaten{Họ và tên}         %Nếu không muốn có dòng này không gõ lệnh
\tenlop{Tên lớp}         %Nếu không muốn có dòng này không gõ lệnh
\sobaodanh{Số báo danh}  %Nếu không muốn có dòng này không gõ lệnh
\khoanh{\cboxv}
\daungoac{\cboxx}{}
\chuphuongan{\small\bfseries\Alph}
\mauchu{blue}
\PSNrandseed{\time}
\usepackage{centerpage}
\graphicspath{{hinh-cauhoi/}} 
\begin{document}
\setlength{\baselineskip}{12truept}
\def\v#1{\overrightarrow{#1}} %Làm vectơ
%%%%%%%%%%%%%
\loadrandomproblems[made100]{50}{2017-cauhoi-toan}
\loadrandomproblems[made150]{50}{2017-cauhoi-toan}
\loadrandomproblems[made160]{50}{2017-cauhoi-toan}
\loadrandomproblems[made201]{50}{2017-cauhoi-toan}
\loadrandomproblems[made250]{50}{2017-cauhoi-toan}

\madethi{100}
\lamtieude
\indebai
\begin{enumerate}[] 
\foreachproblem[made100]{\item\causo\thisproblem}
\end{enumerate}
\label{dbPage}

\newpage
\madethi{150}
\setcounter{page}{1}
\lamtieude
\indebai
\begin{enumerate}[]
\foreachproblem[made150]{\item\causo\thisproblem}
\end{enumerate}

\newpage
\madethi{160}
\setcounter{page}{1}
\lamtieude
\indebai
\begin{enumerate}[] 
\foreachproblem[made160]{\item\causo\thisproblem}
\end{enumerate}

\newpage
\madethi{201}
\setcounter{page}{1}
\lamtieude
\indebai
\begin{enumerate}[] 
\foreachproblem[made201]{\item\causo\thisproblem}
\end{enumerate}

\newpage
\madethi{250}
\setcounter{page}{1}
\lamtieude
\indebai
\begin{enumerate}[] 
\foreachproblem[made250]{\item\causo\thisproblem}
\end{enumerate}
%%%%%%%%%%%%%%%%%%%%%%%%%%%%
\newpage
\thispagestyle{empty}
\indapanso
\setlength{\columnsep}{0pt}
\setlength{\columnseprule}{0pt}
\chucauhoi{}
\begin{center}
\textbf{ĐỀ THI GIỮA HỌC KỲ 2016-2017: MÔN TOÁN}\\
\textbf{ĐÁP ÁN MÃ ĐỀ 100}
\end{center}
\begin{center}
\begin{multicols}{10}
\begin{enumerate}[\causo]
\foreachproblem[made100]{\item\thisproblem}
\end{enumerate}
\end{multicols}
\end{center}

\begin{center}
\textbf{ĐÁP ÁN MÃ ĐỀ 150}
\end{center}
\indapanso
\begin{center}
\begin{multicols}{10}
\begin{enumerate}[\causo]
\foreachproblem[made150]{\item\thisproblem}
\end{enumerate}
\end{multicols}
\end{center}

\begin{center}
\textbf{ĐÁP ÁN MÃ ĐỀ 160}
\end{center}
\indapanso
\begin{center}
\begin{multicols}{10}
\begin{enumerate}[\causo]
\foreachproblem[made160]{\item\thisproblem}
\end{enumerate}
\end{multicols}
\end{center}

\begin{center}
\textbf{ĐÁP ÁN MÃ ĐỀ 201}
\end{center}
\indapanso
\begin{center}
\begin{enumerate}[\causo]
\begin{multicols}{10}
\foreachproblem[made201]{\item\thisproblem}
\end{multicols}
\end{enumerate}
\end{center}

\begin{center}
\textbf{ĐÁP ÁN MÃ ĐỀ 250}
\end{center}
\indapanso
\begin{center}
\begin{enumerate}[\causo]
\begin{multicols}{10}
\foreachproblem[made250]{\item\thisproblem}
\end{multicols}
\end{enumerate}
\end{center}

\end{document}
