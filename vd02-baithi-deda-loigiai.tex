%\title{DeThuNghiem2017}
\documentclass[12pt]{article}
\usepackage{amsmath,amsxtra,latexsym, amssymb, amscd,txfonts}
\usepackage[utf8]{vietnam}
\usepackage{color}
\usepackage{graphicx}
\usepackage{pgf,tikz}
\usetikzlibrary{arrows}
\usepackage{picinpar}
\usepackage{enumerate}
\usepackage{multicol}
\usepackage{shortlst}
\usepackage[baithi]{dethi}
\usepackage{lastpage}
\tentruong{BỘ GIÁO DỤC VÀ ĐÀO TẠO}
\tenkhoa{ĐỀ MINH HỌA}
\loaidethi{Đề gồm có \pageref{LastPage} trang}%{ĐỀ THI LẠI}%%{ĐỀ CHÍNH THỨC}
\tenkythi{KÌ THI TRUNG HỌC PHỔ THÔNG QUỐC GIA NĂM 2017}
\tenmonhoc{Môn: Toán}
\madethi{100}
\thoigian{\underline{Thời gian làm bài: 90 phút, không kể thời gian phát đề}}
\Fullpages
\ContinuousNumbering
\NumberOfVersions{1}
\SectionPrefix{\relax }
\tieudetracnghiem
\tieudedapan
\newcommand*\tikzcircled[1]{\tikz[baseline=(char.base)]{
            \node[shape=circle,draw,inner sep=1pt] (char) {\small #1};}}
\newcommand*\circled[1]{\tikz[baseline=(char.base)]{
            \node[shape=circle,draw,inner sep=1pt] (char) {\small #1};}}

\daungoac{\bfseries\circled}{}
\chucauhoi{Câu}
\setlength{\baselineskip}{12truept}
\def\v#1{\overrightarrow{#1}}
\graphicspath{{images/}}
\khoanh{\cbox}
% \soanthao % không xáo thứ tự phương án
% \NoRearrange % không xáo thứ tự câu
%%%%%%%%%%%%
%%Mẫu in ra đề thi - Đáp án - Lời giải
%%Tùy chọn câu hỏi[keycolumns=1]
%%Tắt lệnh \tieudeduoi để in ra số thứ tự trang bình thường.
%%Tắt lệnh\ShortKey
\coloigiai % đáp án có lời giải chi tiết [keycolumns=1]
%%%%%%%%%%%%
\begin{document}
% \begin{examtop}
% {\bf\large ĐỀ THI TRUNG HỌC PHỔ THÔNG THỬ NGHIỆM}
% \hfill\textbf{\fbox{Mã đề thi 01}}
% \vspace{.5cm}
% \end{examtop}
\setlength{\baselineskip}{12truept}
\begin{vnmultiplechoice}[keycolumns=1]
\selectallproblems{vd08-cauhoi20170120}
\begin{examclosing}
\centerline{------------------ HẾT ------------------}
\end{examclosing}
% \begin{keytop}
% \thispagestyle{empty}
% {\bf\large ĐÁP ÁN ĐỀ THI TRUNG HỌC PHỔ THÔNG THỬ NGHIỆM}
% \hfill\textbf{\fbox{Mã đề thi 01}}
% \end{keytop}
 \end{vnmultiplechoice}
\end{document}
Tôi cũng muốn thống nhất lắm nhưng chưa được, mỗi cái mạnh một thứ, chung tệp dữ liệu là được rồi.
1. Tùy chọn [soanthao] đưa vào băng \input{...} thế mạnh của nó là soạn thảo, nhẩy về câu hỏi cần tìm. Giả sử muốn chữ câu hỏi ở một tệp ta đưa vào và biên dịch, xem có lỗi nhẩy về đúng vị trí TeX (nhiều người chưa biết chức năng này).
2. tùy chọn [baitap] làm sách, báo cáo, văn bản vẫn tệp dữ liệu cũ. Đưa vào bằng lệnh \select.... chọn ngẫu nhiên, một số câu hỏi, cái này cũng có thể làm đề thì được, nhưng làm nhiều đề thì khó,....
3. Tùy chọn [baithi] chuyên làm đề thi, các định dạng, đầu ra khác nhau, cùng cơ sở dữ liệu.
Trong quá trình phát triển mọi tùy chọn rất gần nhau, cung cấp công cụ soạn sách, làm đề, tài liệu dễ dàng và phong phú

Tùy chọn [baithi]
https://drive.google.com/open?id=0B6t6keI_Xwz5NFJwejUyWkx5VHc
Tùy chọn [baitap]
https://drive.google.com/open?id=0B6t6keI_Xwz5d1J5X0xMeWtvRFk
Tùy chọn [soanthao]
https://drive.google.com/open?id=0B6t6keI_Xwz5STVEM2FoNHZ3bjQ