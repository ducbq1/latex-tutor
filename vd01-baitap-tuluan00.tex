\documentclass[12pt,a4letter]{article}
%Các gói lệnh cơ bản
\usepackage{amsmath,amsxtra,amssymb,latexsym, amscd,amsthm}
\usepackage{indentfirst}
\usepackage[mathscr]{eucal}
\usepackage{graphicx}
\usepackage{picinpar}
\usepackage{makecell}
\usepackage{longtable}
\usepackage[utf8]{vietnam}
\usepackage{lastpage}
\usepackage{enumerate}
\usepackage{shortlst}
%Kích thước của trang đề thi
\voffset=-1.5cm
% \hoffset=-2cm
% \textheight 23truecm 
% \textwidth 17truecm 
\textheight 23truecm 
\textwidth 16truecm 
\parskip 3pt
\headsep=12pt

\usepackage[baitap]{dethi}
%Định nghĩa tiêu đề
\def\tendhqg{\small \textbf{ĐH QUỐC GIA HÀ NỘI}}
\def\tentruong{\small \textbf{ĐH KHTN}}
\def\tenkythi{ĐỀ THI KẾT THÚC HỌC KÌ I}
\def\namhoc{NĂM HỌC 2011-2012}
\def\tenmonhoc{Toán Lô gic}
\def\mamonhoc{49ABCD}
\def\sotinchi{2}
\def\svkhoa{Lớp K52A2}
\def\nganhhoc{Toán - Tin ứng dụng}
\def\thoigian{90 phút }
\def\deso{1 }
\def\dapan{ĐÁP ÁN VÀ THANG ĐIỂM}
\def\nguoilamdapan{PGS.TS. Nguyễn Hữu Điển}
\def\ngaylam{Hà nội, ngày 15 tháng 12 năm 2011}

\usepackage{mathpazo}
\usepackage{centerpage}

\begin{document}
\setlength{\baselineskip}{16truept}
\renewcommand{\solutionname}{\textbf{Lời giải.}}
%Phần tiêu đề của đề thi
\thispagestyle{empty}
\begin{minipage}[b]{0.4\textwidth}
\centering
{\bf \tendhqg}\\
{\bf \tentruong}\\ 
-------------\\
\quad \\
\end{minipage}
\hfill
\begin{minipage}[b]{0.6\textwidth}
\centering
{\bf \tenkythi}\\
{\bf  \namhoc}\\
------oOo-------\\ 
\qquad 
\end{minipage}

\centerline{\textbf{Môn thi:} \textbf{\tenmonhoc}}
\leftline{Mã môn học: \textbf{\mamonhoc}\ \hfil Số tín chỉ:  \textbf{\sotinchi} \hfil Đề số: \textbf{\deso} \hfil }
\leftline{Dành cho sinh viên khoá: \textbf{\svkhoa} \hfil Ngành học: \textbf{\nganhhoc}\hfil }
\centerline{Thời gian làm bài \textbf{\thoigian} (không kể thời gian phát đề)}
\noindent \hrulefill
%%%%%%%%%%%%%%%%%%%%%%%%%%%%%%%%%%%%
\renewcommand{\PSNitem}{}
\renewcommand{\endPSNitem}{}
%  \selectrandomly{cauhoi01-tuluan}{3} %P(2)
\loadrandomproblems[dttuluan]{3}{vd01-cauhoi-tuluan}
% 
% \foreachproblem[dttuluan]{\thisproblem}
% \loadrandomproblems[dttuluan]{5}{cauhoi01-tuluan}
% \hideanswers
\begin{enumerate}[]
\foreachproblem[dttuluan]{\item\causo\thisproblem}
\end{enumerate}
\vfill
\noindent \hrulefill

{\bf Chú ý:} Cán bộ coi thi không giải thích gì thêm.
%%%%%%%%%%%%%%%%%%%%%%%%%%%%%%%%%%%%
%Phần đóng tệp đáp án và in ra
\newpage
\setcounter{socauhoi}{0}
\begin{center}
\textbf{\tendhqg} \\
\textbf{ \tentruong} \\                
 -----------------------\\
\textbf{\dapan}\\
\textbf{\tenkythi, \namhoc}\\
\centerline{\textbf{Môn thi:} \textbf{\tenmonhoc}}
\end{center}

\leftline{Mã môn học: \textbf{\mamonhoc}\ \hfil Số tín chỉ:  \textbf{\sotinchi} \hfil Đề số: \textbf{\deso} \hfil }
\leftline{Dành cho sinh viên khoá: \textbf{\svkhoa} \hfil Ngành học: \textbf{\nganhhoc}\hfil }
\hideproblems
\showanswers
\begin{enumerate}[]
\foreachproblem[dttuluan]{\item\causo \thisproblem}
\end{enumerate}

\begin{flushright}
\parbox[c]{8cm}{\centering
\ngaylam\\
NGƯỜI LÀM ĐÁP ÁN\\
(ký và ghi rõ họ tên)\\

\vspace*{1.5cm}
\nguoilamdapan
}
\end{flushright}
\end{document}
 \begin{enumerate}%1
(2 điểm) Cho hai tập lồi $X, Y$ và 

(i) $X\cap Y$;

(ii) $\alpha X=\{\alpha x | x\in X\}$;

(iii) $\alpha X + \beta Y=\{\alpha x+\beta y| x\in X, y\in Y\}$.

\begin{traloi}(2 điểm)
 (i) Cho $x, y\in X\cap Y$ 

(ii) Giả sử ta có

(iii) Chứng minh tương tự phần trên.
\end{traloi}
\end{enumerate}


\begin{enumerate}%2
(4 điểm)  

i) Đưa về dạng chính tắc;

ii) Giải bài toán trên bằng phương pháp đơn hình;

\begin{traloi}(4 điểm) 

i) Dạng chính tắc: 	

ii) Giải bằng phương pháp đơn hình: 

Vấn đề là

 Do còn tồn tại giá trị $\Delta$ lớn hơn 0 

Giá trị hàm mục tiêu đạt được là : $F(x) = 5$
 \end{traloi}
\end{enumerate}


\begin{enumerate}%3
 (4 điểm) Trong vụ bão lụt vừa qua có 4 điểm 

1. Thiết lập bảng vận tải với phương án cơ sở theo phương pháp góc tây bắc.

2. Giải bài toán bằng phương pháp thế vị.
\begin{traloi}
\noindent 1. Đây là bài toán vận tải dạng $\min$.

2.  Phương án cơ sở theo phương pháp góc tây bắc trong  bảng 

\noindent  2. Vòng lặp thứ nhất:

Tính $\Delta_{ij}$ như bảng~2

Ô vi phạm dấu hiệu tối ưu (2,3) là ô đưa vào.

Tìm các thế vị 

Tính $\Delta_{ij}$ như bảng~3

Chi phí tối ưu là $F(x)=350$.
\end{traloi}
\end{enumerate}


