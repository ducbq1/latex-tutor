%Tệp mẫu làm đề thi trắc nghiệm phiên bản 3.0
%Tác giả Nguyễn Hữu Điển (ĐHKHTN, Hà Nội)
% Đề trắc nghiệm được thiết kế trên phông Unicode,
%Đã dùng lớp examdesign.cls có sửa đổi
%Cùng với gói lệnh dethi.sty tạo ra:
%Đề thi trắc nghiệm từ một bộ đề sinh ra các câu hởi được 
%sắp xếp ngẫu nhiên và các chi tiết của câu hỏi cũng được 
%xắp sếp ngẫu nhiên. Mỗi đề thi sinh ra đều có thể in ra đáp án riêng biệt.
%examdesign.cls đòi hỏi các gói lệnh enumerate, multicol, shortlst, keyval.
\documentclass[11pt]{article}
\usepackage{amsmath,amsxtra,latexsym, amssymb, amscd}
\usepackage[utf8]{vietnam}
\usepackage{color}
\usepackage{graphicx}
\usepackage{picinpar}
\usepackage{mathptmx} 
% \usepackage{mathpazo} 
\usepackage{enumerate}
\usepackage{multicol}
\usepackage{shortlst}
\usepackage[baithi]{dethi} %Gói lệnh cho đề thi Việt Nam
% \usepackage{fancybox}
% \cornersize*{3.6mm}
\Fullpages %Định dạng trang đề thi
\ContinuousNumbering %Đánh số liên tục các bài thi
\NumberOfVersions{1} %10 là số bài thi khác nhau được in ra
\SectionPrefix{\relax }%\bf Phần \Roman{sectionindex}. \space}
\tieudetracnghiem
\tieudedapan
\tieudeduoi
\daungoac{}{.}                  %Dấu quanh phương án trả lời: {(}{)};{}{.};{}{)}
%\chuphuongan{\alph}    %Ký tự cho các phương án
%\chuphuongan{\arabic} %\Roman%\roman%kể cả số cho các phương án
\chucauhoi{Câu}                %Chữ trước các số câu hỏi
\mauchu{red}                     %Mầu số câu hỏi và phương án
\setlength{\baselineskip}{12truept}
\def\v#1{\overrightarrow{#1}} %Làm vectơ
\graphicspath{{hinh-cauhoi/}{hinh-vidu/}} %Đường dẫn của nơi để hình
\khoanh{\cbox}         %Khoanh các phương án: \cbox, \fbox
\hovaten{Họ và tên}         %Nếu không muốn có dòng này không gõ lệnh
% \tenlop{Tên lớp}         %Nếu không muốn có dòng này không gõ lệnh
\sobaodanh{Số báo danh}  %Nếu không muốn có dòng này không gõ lệnh
%\ketqua{}          %In ra phần Kết quả
%\giamkhao{}     %In ra phần chữ ký giám khảo ở phiếu thi
%\NoRearrange  %Lệnh không trộn đề
%\motphieuthi      %In ra một phiếu thi, Mặc định là không hiện ra phiếu thi
%\nhieuphieuthi   %In ra mỗi đề một phiếu thi
% \coloigiai           %In ra đáp án có lời giải
\ShortKey             %Lệnh hiện ra đáp án mỗi đề thi
%\OneKey            %Lệnh chỉ in ra 1 bản đáp án
%\NoKey               %Lệnh không in ra phần đáp án

\tentruong{BỘ GIÁO DỤC VÀ ĐÀO TẠO}
\tenkhoa{ĐỀ MINH HỌA}
\loaidethi{Đề gồm có 06 trang}%{ĐỀ THI LẠI}%%{ĐỀ CHÍNH THỨC}
\tenkythi{KÌ THI TRUNG HỌC PHỔ THÔNG QUỐC GIA NĂM 2017}
\tenmonhoc{Môn: Toán}
\madethi{100}
\thoigian{\underline{Thời gian làm bài: 90 phút, không kể thời gian phát đề}}

\begin{document}
\begin{vnmultiplechoice}[title={\bf  Câu hỏi gạch dưới},  keycolumns=3]
\begin{question}
\saih{That} is \saih{the} man \dungh{which} told me \saih{the} bad news.
\datcot
\bonpah
{\sai{That}}
{\sai{the}} 
{\dung{which}} 
{\sai{the}}
\end{question}


\begin{question}
 My \saih{younger} brother \saih{has} worked in \saih{a} bank \dungh{since} a long time.
\datcot
\bonpah
{\sai{younger}}
{\sai{has}}
{\sai{a}}
{\dung{since}}
\end{question}

\begin{question}
\saih{It is} \saih{the English} pronunciation that \dungh{cause} me \saih{a lot of} difficulties.
\datcot
\bonpah
{\sai{It is}} 
{\sai{the English}}
{\dung{cause}} 
{\sai{a lot of}}
\end{question}

\begin{question}
 I \dungh{go} to Mexico \saih{with} my girlfriend in \saih{the} summer \saih{of} 2006.
\datcot
\bonpah
{\dung{go}}
{\sai{with}}
{\sai{the}}
{\sai{of}}
\end{question}

\begin{question}
He was \dungh{angrily} \saih{when} he \saih{saw} what \saih{was happening}.
\datcot
\bonpah
{\dung{angrily}} 
{\sai{when}}
{\sai{saw}} 
{\sai{was happening}}
\end{question}

\begin{examclosing}
\centerline{-- HẾT --}
\end{examclosing}
\end{vnmultiplechoice}
\end{document}
%%12:52:31 26/10/2016Last Modification of contents