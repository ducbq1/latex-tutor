% Tệp mẫu làm đề thi trắc nghiệm dựa vào gói lệnh lamdethi.sty
% Tác giả: Nguyên Hữu Điển
% Khoa Toán Cơ Tin học, ĐHKHTN HN, ĐHQGHN
% 334, Nguyễn Trãi, Thanh Xuân, Hà Nội
% huudien@vnu.edu.vn
% Ngày 26/12/2009
%%%%%%%%%%%%%%%%%%%%%%%%%%%%
\documentclass[11pt]{article}
\usepackage{amsmath,amsxtra,amssymb,latexsym, amscd,amsthm}
\usepackage{graphicx}
% \usepackage{picinpar}
\usepackage{tikz}
\usetikzlibrary{arrows}
\usepackage{tkz-tab}
\usepackage[utf8]{vietnam}
\usepackage{longtable}%
\usepackage[bookmarks,colorlinks,hyperindex,plainpages=false,unicode,hypertexnames=false,debug]{hyperref}
\usepackage{bookmark}
% \usepackage{titledot}
% \usepackage{multicol}%
% \usepackage{shortlst}
% \usepackage{color}
% \usepackage{enumerate}
\usepackage{mathpazo} 
% \usepackage{balance} 
\voffset=-3cm
% \hoffset=-2cm
\textheight 25truecm 
\textwidth 19truecm 
\usepackage[baitap]{dethi}
\tentruong{BỘ GIÁO DỤC VÀ ĐÀO TẠO}
\tenkhoa{ĐỀ THAM KHẢO}
\loaidethi{(Đề gồm  \pageref{DebaiPage} trang)}%{ĐỀ THI LẠI}%%{ĐỀ CHÍNH THỨC}
\tenkythi{KỲ THI TRUNG HỌC PHỔ THÔNG QUỐC GIA 2017}
\tenmonhoc{Bài thi: TOÁN}
\madethi{100}
\tieudeduoi
\thoigian{\underline{Thời gian làm bài: 90 phút, không kể thời gian phát đề}}   
% \hovaten{Họ và tên}         %Nếu không muốn có dòng này không gõ lệnh
% \tenlop{Tên lớp}         %Nếu không muốn có dòng này không gõ lệnh
% \sobaodanh{Số báo danh}  %Nếu không muốn có dòng này không gõ lệnh
\khoanh{\cboxv}
\daungoac{\cboxx}{}
\chuphuongan{\small\bfseries\Alph}
\mauchu{blue}
\PSNrandseed{\time}

\usepackage{lastpage}
\graphicspath{{hinh-tracnghiem/}{hinh/}{hinh-cauhoi/}{images/}{dethamkhao/}} 
\parindent 10pt
% \renewcommand{\bonpa}[4]{#1#2#3#4}
\usepackage{centerpage}
\usepackage{forloop}
\begin{document}
\setlength{\baselineskip}{12truept}
\def\v#1{\overrightarrow{#1}} %Làm vectơ
\setlength{\columnseprule}{0pt}
%%%%%%%%%%%%%
%%%Lập đề bài thi
\addcontentsline{toc}{section}{Bài thi Toán 2017 (lần 3)}
% \immediate\openout\tempfile=deluu100.tex
\addcontentsline{toc}{subsection}{Đề bài mã 100}
\loadrandomproblems[ma100]{50}{vd13-cauhoi-detk2017-toan}
 \setlength{\columnsep}{10pt}
\lamtieude
\indebai
\demexcel
% \lapcauhoi{ma100}
% 
\begin{enumerate}[]
\foreachproblem[ma100]{\item\causo\addcontentsline{toc}{subsubsection}{Câu \thesocauhoi.\thisproblemlabel}\thisproblem\dienbang}
 \end{enumerate}

\immediate\closeout\tempfile
\immediate\closeout\file
\label{DebaiPage}

\newpage
\thispagestyle{empty}
\lamtieude
\dempa
\lapbang
\dapanexcel
\setlength{\columnsep}{5pt}
\dapanlietke{ma100}
\dapanma{ma100}
%  \end{document}

% \newpage
%%In ra bản thống kê
% \thispagestyle{empty}
% \lamtieude
% \dempa
% \dapanexcel
% \dapanma{ma100}

%%%%%%%%%%%%%%%%
\newpage
%%%Lời giải
\setcounter{page}{1}
\loaidethi{(Lời giải  \pageref{lgPage} trang)}
\trangcuoi{lgPage}
\lamtieude
\begin{center}
\textbf{LỜI GIẢI}
\end{center}
\inloigiai
\addcontentsline{toc}{subsection}{Lời giải mã 100}
\begin{enumerate}[]
\foreachproblem[ma100]{\item\thisproblem\addcontentsline{toc}{subsubsection}{Câu \thesocauhoi.\thisproblemlabel}}
\end{enumerate}
\label{lgPage}
% \end{document}
\newpage
\setcounter{page}{1}
\loaidethi{(Đề bài và lời giải  \pageref{dblgPage} trang)}
\trangcuoi{dblgPage}
\lamtieude
\begin{center}
\textbf{ĐỀ BÀI VÀ LỜI GIẢI}
\end{center}
\indebailoigiai
\addtocontents{toc}{\quad }
\addcontentsline{toc}{subsection}{Đề bài và Lời giải mã 100}
% \lapcauhoi{ma100}
\begin{enumerate}[]
\foreachproblem[ma100]{\item\causo\addcontentsline{toc}{subsubsection}{Câu \thesocauhoi.\thisproblemlabel}\thisproblem}
\end{enumerate}
\label{dblgPage}

  \end{document}
\newpage
\loaidethi{(Đáp án rút gọn)}
\lamtieude
\indapanrutgon
\thispagestyle{empty}
\begin{center}
{\bf ĐÁP ÁN RÚT GỌN}
\end{center}
\begin{multicols}{3}
\begin{enumerate}[\causo]
 \foreachproblem[ma100]{\item\thisproblem}
\end{enumerate}
\end{multicols}

\newpage
\setcounter{page}{1}
\thispagestyle{empty}
\inphieuduclo
\lamphieuthi
\begin{center}
\begin{multicols}{3}
\begin{enumerate}[\causo]
\foreachproblem[ma100]{\item\thisproblem}
\end{enumerate}
\end{multicols}
\end{center}

\newpage
 \thispagestyle{empty}
\inphieuthi
\lamphieuthi
\begin{multicols}{3}
\begin{enumerate}[\causo]
\foreachproblem[ma100]{\item\thisproblem}
\end{enumerate}
\end{multicols}

\newpage
\setcounter{page}{1}
\addcontentsline{toc}{section}{Bài thi Toán 2017 (lần 2)}
\loaidethi{(Đề gồm  \pageref{DebPage} trang)}
\madethi{150}
\trangcuoi{DebPage}
\addcontentsline{toc}{subsection}{Đề bài mã 150}
\loadrandomproblems[ma150]{50}{2017-cauhoi-toan}
 \setlength{\columnsep}{10pt}
\lamtieude
\indebai
\demexcel
\lapcauhoi{ma150}
\label{DebPage}
% \end{document}
\newpage
%%%In ra bản thống kê
\thispagestyle{empty}
\lamtieude
\dempa
\dapanexcel
\dapanma{ma150}

%%%%%%%%%%%%%%%%
\newpage
%%%Lời giải
\setcounter{page}{1}
\loaidethi{(Lời giải  \pageref{lgiPage} trang)}
\trangcuoi{lgiPage}
\lamtieude
\inloigiai
\addcontentsline{toc}{subsection}{Lời giải mã 150}
\begin{enumerate}[]
\foreachproblem[ma150]{\item\thisproblem\addcontentsline{toc}{subsubsection}{Câu \thesocauhoi.\thisproblemlabel}}
\end{enumerate}
\label{lgiPage}
\end{document}
% \newpage
% \lamtieude
% \indapanlietke
% \thispagestyle{empty}
% \begin{center}
% {\bf ĐÁP ÁN LIỆT KÊ THEO SỐ}
% \end{center}
% \begin{enumerate}[\socau]
% \begin{multicols}{5}
% \foreachproblem[ma100]{\item\thisproblem}
% \end{multicols}
% \end{enumerate}


% \newpage
% \setcounter{page}{1}
% \lamtieude
% \begin{center}
% {\bf LỜI GIẢI}
% \end{center}
% \inloigiai
% \begin{enumerate}[]
% \foreachproblem[ma100]{\item\thisproblem}
% \end{enumerate}




\newpage
\setcounter{page}{1}
\lamtieude
\indebailoigiai
\begin{center}
{\bf ĐỀ BÀI - ĐÁP ÁN - LỜI GIẢI}
\end{center}
\begin{enumerate}[]
\foreachproblem[ma100]{\item\causo\thisproblem}
\end{enumerate}

\newpage
\setcounter{page}{1}
\lamtieude
\indebaidapan
\begin{center}
{\bf ĐỀ BÀI VÀ ĐÁP ÁN ĐÁNH DẤU}
\end{center}
\begin{enumerate}[]
\foreachproblem[ma100]{\item\causo\thisproblem}
\end{enumerate}

\end{document}
https://drive.google.com/open?id=0B6t6keI_Xwz5UHNsTjY4NU1oMUU
