%Tệp mẫu làm đề thi trắc nghiệm phiên bản 3.0
%Tác giả Nguyễn Hữu Điển (ĐHKHTN, Hà Nội)
% Đề trắc nghiệm được thiết kế trên phông Unicode,
%Đã dùng lớp examdesign.cls có sửa đổi
%Cùng với gói lệnh dethi.sty tạo ra:
%Đề thi trắc nghiệm từ một bộ đề sinh ra các câu hởi được 
%sắp xếp ngẫu nhiên và các chi tiết của câu hỏi cũng được 
%xắp sếp ngẫu nhiên. Mỗi đề thi sinh ra đều có thể in ra đáp án riêng biệt.
%examdesign.cls đòi hỏi các gói lệnh enumerate, multicol, shortlst, keyval.
\documentclass[11pt]{article}
\usepackage{amsmath,amsxtra,latexsym, amssymb, amscd}
\usepackage[utf8]{vietnam}

\usepackage{color}
\usepackage{lastpage}
\usepackage{graphicx}
\usepackage{wrapfig}
\usepackage{mathptmx} 
% \usepackage{mathpazo} 
\usepackage{enumerate}
\usepackage{multicol}
\usepackage{shortlst}
\usepackage[baithi]{dethi} %Gói lệnh cho đề thi Việt Nam
% \usepackage{fancybox}
% \cornersize*{3.6mm}
\Fullpages %Định dạng trang đề thi
\ContinuousNumbering %Đánh số liên tục các bài thi
\NumberOfVersions{3} %10 là số bài thi khác nhau được in ra
\SectionPrefix{\relax }%\bf Phần \Roman{sectionindex}. \space}
\tieudetracnghiem
%\tieudethiviet
\tieudedapan
%\tieudetren
\tieudeduoi
\daungoac{}{.}                  %Dấu quanh phương án trả lời: {(}{)};{}{.};{}{)}
%\chuphuongan{\alph}    %Ký tự cho các phương án
%\chuphuongan{\arabic} %\Roman%\roman%kể cả số cho các phương án
\chucauhoi{Câu}                %Chữ trước các số câu hỏi
\mauchu{red}                     %Mầu số câu hỏi và phương án
\setlength{\baselineskip}{12truept}
\def\v#1{\overrightarrow{#1}} %Làm vectơ
\graphicspath{{hinh-cauhoi/}} %Đường dẫn của nơi để hình
\khoanh{\cbox}         %Khoanh các phương án: \cbox, \fbox
\hovaten{Họ và tên}         %Nếu không muốn có dòng này không gõ lệnh
% \tenlop{Tên lớp}         %Nếu không muốn có dòng này không gõ lệnh
\sobaodanh{Số báo danh}  %Nếu không muốn có dòng này không gõ lệnh
%\ketqua{}          %In ra phần Kết quả
%\giamkhao{}     %In ra phần chữ ký giám khảo ở phiếu thi
%\NoRearrange  %Lệnh không trộn đề
% \motphieuthi      %In ra một phiếu thi, Mặc định là không hiện ra phiếu thi
%\nhieuphieuthi   %In ra mỗi đề một phiếu thi
%\coloigiai           %In ra đáp án có lời giải
\ShortKey             %Lệnh hiện ra đáp án mỗi đề thi
%\OneKey            %Lệnh chỉ in ra 1 bản đáp án
%\NoKey               %Lệnh không in ra phần đáp án

\tentruong{BỘ GIÁO DỤC VÀ ĐÀO TẠO}
\tenkhoa{ĐỀ MINH HỌA}
\loaidethi{Đề gồm có 04 trang}%{ĐỀ THI LẠI}%%{ĐỀ CHÍNH THỨC}
\tenkythi{KÌ THI TRUNG HỌC PHỔ THÔNG QUỐC GIA NĂM 2017}
\tenmonhoc{Bài thi: Khoa học tự nhiên;  Môn: SINH HỌC}
\madethi{100}
\thoigian{\underline{Thời gian làm bài: 50 phút, không kể thời gian phát đề}}

\begin{document}

\setlength{\baselineskip}{12truept}
 \begin{vnmultiplechoice}[ rearrange=yes, keycolumns=3]%

\begin{question} %%01

\datcot
\bonpa
{\sai{}}
{\sai{}}
{\sai{}}
{\dung{}}
\end{question}

\begin{question} %%02

\datcot
\bonpa
{\sai{}}
{\sai{}}
{\sai{}}
{\dung{}}
\end{question}

\begin{question} %%03

\datcot
\bonpa
{\sai{}}
{\sai{}}
{\sai{}}
{\dung{}}
\end{question}

\begin{question} %%04

\datcot
\bonpa
{\sai{}}
{\sai{}}
{\sai{}}
{\dung{}}
\end{question}

\begin{question} %%05

\datcot
\bonpa
{\sai{}}
{\sai{}}
{\sai{}}
{\dung{}}
\end{question}

\begin{question} %%06

\datcot
\bonpa
{\sai{}}
{\sai{}}
{\sai{}}
{\dung{}}
\end{question}

\begin{question} %%07

\datcot
\bonpa
{\sai{}}
{\sai{}}
{\sai{}}
{\dung{}}
\end{question}

\begin{question} %%08

\datcot
\bonpa
{\sai{}}
{\sai{}}
{\sai{}}
{\dung{}}
\end{question}

\begin{question} %%09

\datcot
\bonpa
{\sai{}}
{\sai{}}
{\sai{}}
{\dung{}}
\end{question}

\begin{question} %%10

\datcot
\bonpa
{\sai{}}
{\sai{}}
{\sai{}}
{\dung{}}
\end{question}
%%%%%%%%%%%%%%%%%%%%%%
\begin{question} %%11

\datcot
\bonpa
{\sai{}}
{\sai{}}
{\sai{}}
{\dung{}}
\end{question}

\begin{question} %%12

\datcot
\bonpa
{\sai{}}
{\sai{}}
{\sai{}}
{\dung{}}
\end{question}

\begin{question} %%13

\datcot
\bonpa
{\sai{}}
{\sai{}}
{\sai{}}
{\dung{}}
\end{question}

\begin{question} %%14

\datcot
\bonpa
{\sai{}}
{\sai{}}
{\sai{}}
{\dung{}}
\end{question}

\begin{question} %%15

\datcot
\bonpa
{\sai{}}
{\sai{}}
{\sai{}}
{\dung{}}
\end{question}

\begin{question} %%16

\datcot
\bonpa
{\sai{}}
{\sai{}}
{\sai{}}
{\dung{}}
\end{question}

\begin{question} %%17

\datcot
\bonpa
{\sai{}}
{\sai{}}
{\sai{}}
{\dung{}}
\end{question}

\begin{question} %%18

\datcot
\bonpa
{\sai{}}
{\sai{}}
{\sai{}}
{\dung{}}
\end{question}

\begin{question} %%19

\datcot
\bonpa
{\sai{}}
{\sai{}}
{\sai{}}
{\dung{}}
\end{question}

\begin{question} %%20

\datcot
\bonpa
{\sai{}}
{\sai{}}
{\sai{}}
{\dung{}}
\end{question}
%%%%%%%%%%%%%%%%%%%%%%%%%
\begin{question} %%21

\datcot
\bonpa
{\sai{}}
{\sai{}}
{\sai{}}
{\dung{}}
\end{question}

\begin{question} %%22

\datcot
\bonpa
{\sai{}}
{\sai{}}
{\sai{}}
{\dung{}}
\end{question}

\begin{question} %%23

\datcot
\bonpa
{\sai{}}
{\sai{}}
{\sai{}}
{\dung{}}
\end{question}

\begin{question} %%24

\datcot
\bonpa
{\sai{}}
{\sai{}}
{\sai{}}
{\dung{}}
\end{question}

\begin{question} %%25

\datcot
\bonpa
{\sai{}}
{\sai{}}
{\sai{}}
{\dung{}}
\end{question}

\begin{question} %%26

\datcot
\bonpa
{\sai{}}
{\sai{}}
{\sai{}}
{\dung{}}
\end{question}

\begin{question} %%27

\datcot
\bonpa
{\sai{}}
{\sai{}}
{\sai{}}
{\dung{}}
\end{question}

\begin{question} %%28

\datcot
\bonpa
{\sai{}}
{\sai{}}
{\sai{}}
{\dung{}}
\end{question}

\begin{question} %%29

\datcot
\bonpa
{\sai{}}
{\sai{}}
{\sai{}}
{\dung{}}
\end{question}

\begin{question} %%30

\datcot
\bonpa
{\sai{}}
{\sai{}}
{\sai{}}
{\dung{}}
\end{question}
%%%%%%%%%%%%%%%%%%%%%%%%%%
\begin{question} %%31

\datcot
\bonpa
{\sai{}}
{\sai{}}
{\sai{}}
{\dung{}}
\end{question}

\begin{question} %%32

\datcot
\bonpa
{\sai{}}
{\sai{}}
{\sai{}}
{\dung{}}
\end{question}

\begin{question} %%33

\datcot
\bonpa
{\sai{}}
{\sai{}}
{\sai{}}
{\dung{}}
\end{question}

\begin{question} %%34

\datcot
\bonpa
{\sai{}}
{\sai{}}
{\sai{}}
{\dung{}}
\end{question}

\begin{question} %%35

\datcot
\bonpa
{\sai{}}
{\sai{}}
{\sai{}}
{\dung{}}
\end{question}

\begin{question} %%36

\datcot
\bonpa
{\sai{}}
{\sai{}}
{\sai{}}
{\dung{}}
\end{question}

\begin{question} %%37

\datcot
\bonpa
{\sai{}}
{\sai{}}
{\sai{}}
{\dung{}}
\end{question}

\begin{question} %%38

\datcot
\bonpa
{\sai{}}
{\sai{}}
{\sai{}}
{\dung{}}
\end{question}

\begin{question} %%39

\datcot
\bonpa
{\sai{}}
{\sai{}}
{\sai{}}
{\dung{}}
\end{question}

\begin{question} %%40

\datcot
\bonpa
{\sai{}}
{\sai{}}
{\sai{}}
{\dung{}}
\end{question}

\begin{examclosing}
\centerline{-- HẾT --}
\end{examclosing}
 \end{vnmultiplechoice}
\end{document}
[scribd id=327589426 key=key-oJpAkf8g7Gy0j9JNOeAs mode=scroll]
