%%21:17:45 21/10/2016 -VieTeX creates E:\tex\book-mau\mau-dethi30\vidu05-ftruefalse.tex
%Tệp mẫu làm đề thi trắc nghiệm phiên bản 3.0
%Tác giả Nguyễn Hữu Điển (ĐHKHTN, Hà Nội)
% Đề trắc nghiệm được thiết kế trên phông Unicode,
%Đã dùng lớp examdesign.cls có sửa đổi
%Cùng với gói lệnh dethi.sty tạo ra:
%Đề thi trắc nghiệm từ một bộ đề sinh ra các câu hởi được 
%sắp xếp ngẫu nhiên và các chi tiết của câu hỏi cũng được 
%xắp sếp ngẫu nhiên. Mỗi đề thi sinh ra đều có thể in ra đáp án riêng biệt.
%examdesign.cls đòi hỏi các gói lệnh enumerate, multicol, shortlst, keyval.
\documentclass[11pt]{article}
\usepackage{amsmath,amsxtra,latexsym, amssymb, amscd}
\usepackage[utf8]{vietnam}
% \usepackage{color}
\usepackage{graphicx}
% \usepackage{picinpar}
\usepackage{mathptmx} 
% \usepackage{mathpazo} 
% \usepackage{enumerate}
% \usepackage{multicol}
% \usepackage{shortlst}
\usepackage[baithi]{dethi} %Gói lệnh cho đề thi Việt Nam
% \usepackage{lastpage}
% \usepackage{fancybox}
% \cornersize*{3.6mm}
% \usepackage{tikz}
\Fullpages %Định dạng trang đề thi
\ContinuousNumbering %Đánh số liên tục các bài thi
\NumberOfVersions{1} %10 là số bài thi khác nhau được in ra
\SectionPrefix{\relax }%\bf Phần \Roman{sectionindex}. \space}
 \tieudetracnghiem
\tieudedapan
%\tieudetren
\tieudeduoi
\daungoac{\cboxx}{}                  %Dấu quanh phương án trả lời: {(}{)};{}{.};{}{)}
\chuphuongan{\bf\Alph}    %Ký tự cho các phương án
%\chuphuongan{\arabic} %\Roman%\roman%kể cả số cho các phương án
\chucauhoi{Câu}                %Chữ trước các số câu hỏi
\mauchu{blue}                     %Mầu số câu hỏi và phương án
\setlength{\baselineskip}{12truept}
\def\v#1{\overrightarrow{#1}} %Làm vectơ
\graphicspath{{hinh-cauhoi/}{hinh-vidu/}} %Đường dẫn của nơi để hình
\khoanh{\cboxx}         %Khoanh các phương án: \cbox, \fbox
\hovaten{Họ và tên}         %Nếu không muốn có dòng này không gõ lệnh
% \tenlop{Tên lớp}         %Nếu không muốn có dòng này không gõ lệnh
\sobaodanh{Số báo danh}  %Nếu không muốn có dòng này không gõ lệnh
%\ketqua{}          %In ra phần Kết quả
%\giamkhao{}     %In ra phần chữ ký giám khảo ở phiếu thi
%\NoRearrange  %Lệnh không trộn đề
%\socotdapan{2} %số cột đáp án và phiếu thi
%\motphieuthi      %In ra một phiếu thi, Mặc định là không hiện ra phiếu thi
%\nhieuphieuthi   %In ra mỗi đề một phiếu thi
% \coloigiai           %In ra đáp án có lời giải
\ShortKey             %Lệnh hiện ra đáp án mỗi đề thi
%\OneKey            %Lệnh chỉ in ra 1 bản đáp án
%\NoKey               %Lệnh không in ra phần đáp án

\tentruong{BỘ GIÁO DỤC VÀ ĐÀO TẠO}
\tenkhoa{ĐỀ MINH HỌA}
\loaidethi{Đề gồm có 06 trang}%{ĐỀ THI LẠI}%%{ĐỀ CHÍNH THỨC}
\tenkythi{KÌ THI TRUNG HỌC PHỔ THÔNG QUỐC GIA NĂM 2017}
\tenmonhoc{Môn: Toán}
\madethi{100}
\thoigian{\underline{Thời gian làm bài: 90 phút, không kể thời gian phát đề}}

\begin{document}

\setlength{\baselineskip}{12truept}

  \begin{multiplechoice}[title={\bf I. Các câu hỏi trắc nghiệm} , rearrange=yes, keycolumns=3]%
\begin{question}
Hệ số của số hạng không chứa $x$ trong khai triển $(\dfrac{1}{2}\sqrt x+\dfrac{2}{\sqrt[3]{x}})^{10}$ là:
\datcot
\bonpa
{\sai{Không có}}
{\sai{$210$}}
{\dung{$840$}}
{\sai{$120$}}
\end{question}
%%%%%%%%%%%%%
\begin{question}
Cho $A=\{ 0,1,2,3,4,5,6\}$. Số các số chẵn có 4 chữ số đôi một khác nhau được lập từ $A$ là:
\datcot
\bonpa
{\dung{$420$}}
{\sai{$360$}}
{\sai{$15$}}
{\sai{$400$}}
\end{question}

 \end{multiplechoice}
\end{document}
%%%%%%%%%%%%%%%%%%%%%%%%%%%%%%%%%%
\begin{shortanswer}[title={\textbf{II. Câu hỏi tự luận}}, rearrange=no ]
\begin{question}
  And as in uffish thought he stood, the Jabberwock, with eyes of flame,
  came whiffling through the tulgey wood, and burbled as it came!
  \begin{answer}
    Ask Lewis Carroll.
  \end{answer}
\end{question}

\begin{question}
  One, two! One, two! And through and through the vorpal blade went snicker-snack!
  He left it dead, and with its head he went galumphing back.
  \begin{answer}
    Ask Lewis Carroll.
  \end{answer}
\end{question}
\end{shortanswer}
%%%%%%%%%%%%%%%%%%%%%%%%%%%%%%%%%%%
\begin{fillin}[title={\bf III. Câu hỏi điền từ vào chỗ trống}, resetcounter=yes,keycolumns=2]

  \begin{question}
\examvspace*{0.7\baselineskip}
  Nhất trí How much \blank{wood} would a \blank{woodchuck} chuck, if a \blank{woodchuck}
  would \blank{chuck}, wood?
\end{question}
\begin{question}
\examvspace*{0.7\baselineskip}
  \blank{Wittgenstein}'s first work was the \textsl{Tractatus-\blank{Logico}
  Philosophicus}.
\end{question}
 \begin{block}
   I don't know why you'd need this, but here is a block of fill-in-the-blank
   questions. 
  \begin{question}
\examvspace*{0.7\baselineskip}
    \blank{Hobbes} thought that without a strong, \blank{centralized},
    effective government, chaos would reign in the state of nature.
  \end{question}
  \begin{question}
\examvspace*{0.7\baselineskip}
    One main component of Nietzche's moral philosophy is the \blank{will to
      power}.
  \end{question}
\end{block}

\begin{question}
\examvspace*{0.7\baselineskip}
  Mill's theory of morality is known as \blank{Utilitarianism}
\end{question}

\begin{question}
\examvspace*{0.7\baselineskip}
  According to Kant, we should always always always follow the
  \blank{categorical} imperative.
\end{question}
\end{fillin}
%%%%%%%%%%%%%%%%%%%%%%%%%%%
\begin{matching}[title={\bf IV. Câu hỏi ghép nối}]
  \pair{Elvis Costello}{Spike}
  \pair{Nirvana}{Nevermind}
  \pair{Love and Rockets}{Earth, Sun, Moon}
  \pair{The Jesus and Mary Chain}{Automatic}
  \pair{The Dave Matthews Band}{Under the Table and Dreaming}
\end{matching}
%%%%%%%%%%%%%%%%%%%%%%%%%%%%%
\begin{truefalse}[title={\bf  V. Câu hỏi đúng sai}]

\begin{question}
\examvspace*{0.7\baselineskip}
  \answer{Đúng} This sentence is notfalse.
\end{question}

\begin{block}
  I don't know why you'd need this, but here is a block of true/false
  questions. 
  \begin{question}
\examvspace*{-0.7\baselineskip}
    \answer{Đúng} `Roger \& Trường Me' chronicles one man's attempt to get into
    Disneyland so that he can visit Toontown.
  \end{question}
  
  \begin{question}
\examvspace*{-0.7\baselineskip}
    \answer{Sai} Laden swallows fly faster than unladen swallows, unless
    they carry coconuts.
  \end{question}
\end{block}

\begin{question}
\examvspace*{0.7\baselineskip}
  \answer{Đúng} `Monty Python and the Holy Grail' is a very funny movie.
\end{question}

\begin{question}
\examvspace*{0.7\baselineskip}
  \answer{Sai} All animals are created equal, but some animals are more
  equal than others.
\end{question}
\begin{examclosing}
\centerline{-- HẾT --}
\end{examclosing}
\end{truefalse}

\end{document}