% Tệp mẫu làm đề thi trắc nghiệm dựa vào gói lệnh lamdethi.sty
% Tác giả: Nguyên Hữu Điển
% Khoa Toán Cơ Tin học, ĐHKHTN HN, ĐHQGHN
% 334, Nguyễn Trãi, Thanh Xuân, Hà Nội
% huudien@vnu.edu.vn
% Ngày 26/12/2009
%%%%%%%%%%%%%%%%%%%%%%%%%%%%

\documentclass[11pt]{article}
\usepackage{amsmath,amsxtra,amssymb,latexsym, amscd,amsthm}
\usepackage[utf8]{vietnam}
\usepackage{longtable}%
\usepackage{mathptmx} 
\voffset=-3cm
% \hoffset=-2cm
\textheight 24truecm 
\textwidth 18.5truecm 
\usepackage[baitap]{dethi}
\tentruong{ĐẠI HỌC KHOA HỌC TỰ NHIÊN}
\tenkhoa{Khoa Toán - Cơ -Tin học}
\loaidethi{Đề gồm có \pageref{dbPage} trang}%
\tenkythi{ĐỀ THI GIỮA KỲ NĂM HỌC 2016-2017}
\tenmonhoc{Môn: Toán học tính toán}
\madethi{100}
\thoigian{\underline{Thời gian làm bài: 90 phút, không kể thời gian phát đề}}   
\hovaten{Họ và tên}         %Nếu không muốn có dòng này không gõ lệnh
\tenlop{Tên lớp}         %Nếu không muốn có dòng này không gõ lệnh
\sobaodanh{Số báo danh}  %Nếu không muốn có dòng này không gõ lệnh
\tieudeduoi
\khoanh{\cboxv}
\daungoac{\cboxx}{}
\chuphuongan{\small\bfseries\Alph}
\mauchu{blue}
\PSNrandseed{\time}
\usepackage{centerpage}
\graphicspath{{hinh-cauhoi/}} 
\parindent 10pt
\begin{document}
\setlength{\baselineskip}{12truept}
\def\v#1{\overrightarrow{#1}} %Làm vectơ
%%%%%%%%%%%%%
\lamtieude
\loadrandomproblems[bttracnghiem]{50}{2017-cauhoi-toan}
\trangcuoi{dbPage}
\indebai
\begin{enumerate}[]
\foreachproblem[bttracnghiem]{\item\causo\thisproblem}
\end{enumerate}
\label{dbPage}

\newpage
\setcounter{page}{1}
\loaidethi{Đề gồm có \pageref{dbdaPage} trang}%
\trangcuoi{dbdaPage}
\lamtieude
\indebaidapan
\begin{center}
{\bf ĐỀ BÀI VÀ ĐÁP ÁN ĐÁNH DẤU}
\end{center}
\begin{enumerate}[]
\foreachproblem[bttracnghiem]{\item\causo\thisproblem}
\end{enumerate}
\label{dbdaPage}

\newpage
\setcounter{page}{1}
\loaidethi{Đề gồm có \pageref{dblgPage} trang}%
\trangcuoi{dblgPage}
\lamtieude
\indebailoigiai
\begin{center}
{\bf ĐỀ BÀI - ĐÁP ÁN - LỜI GIẢI}
\end{center}
\begin{enumerate}[]
\foreachproblem[bttracnghiem]{\item\causo\thisproblem}
\end{enumerate}
\label{dblgPage}

\newpage
% \setcounter{page}{1}
\lamtieude
\indapanrutgon
\thispagestyle{empty}
\begin{center}
{\bf ĐÁP ÁN RÚT GỌN}
\end{center}
\begin{multicols}{3}
\begin{enumerate}[\causo]
\foreachproblem[bttracnghiem]{\item\thisproblem}
\end{enumerate}
\end{multicols}


\newpage
% \setcounter{page}{1}
\lamtieude
\indapanlietke
\thispagestyle{empty}
\begin{center}
{\bf ĐÁP ÁN LIỆT KÊ}
\end{center}

\begin{multicols}{5}
\begin{enumerate}[\causo]
\foreachproblem[bttracnghiem]{\item\thisproblem}
\end{enumerate}
\end{multicols}


\vspace*{1cm}
\lamtieude
\indapanso
\begin{center}
{\bf ĐÁP ÁN LIỆT KÊ THEO SỐ}
\end{center}
\begin{center}
\chucauhoi{ }
\begin{multicols}{10}
\begin{enumerate}[\causo]
\foreachproblem[bttracnghiem]{\item\thisproblem}
\end{enumerate}
\end{multicols}
\chucauhoi{Câu }
\end{center}




\newpage
\setcounter{page}{1}
\loaidethi{Đề gồm có \pageref{lgPage} trang}%
\trangcuoi{lgPage}
\lamtieude
\inloigiai
\begin{center}
{\bf LỜI GIẢI}
\end{center}
\begin{enumerate}[]
\foreachproblem[bttracnghiem]{\item\thisproblem}
\end{enumerate}
\label{lgPage}




\newpage
% \setcounter{page}{1}
\lamphieuthi
 \inphieuthi
\begin{multicols}{3}
\begin{enumerate}[\causo]
 \foreachproblem[bttracnghiem]{\item\thisproblem}
\end{enumerate}
\end{multicols}


% \addtocounter{page}{-2}


\end{document}
https://drive.google.com/open?id=0B6t6keI_Xwz5dmtjbHlIbjIyc0k
%%14:19:44 5/5/2017Last Modification of contents