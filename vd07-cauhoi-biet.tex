%-[câu hỏi 1]%%%%%%%%%%
\baitracnghiem{biet:b01}{%
Trong các mệnh đề sau, mệnh đề nào đúng?
}{
\datcot[4]
\bonpa
{\dung{Hình đa diện luôn có số cạnh lớn hơn hoặc bằng 6.}}
{\sai{Hình đa diện luôn có số cạnh lớn hơn 8.}}
{\sai{Hình đa diện luôn có số cạnh lớn hơn hoặc bằng 7.}}
{\sai{Hình đa diện luôn có số cạnh lớn hơn 6.}}
}
%-[câu hỏi 2]%%%%%%%%%%
\baitracnghiem{biet:b02}{%
Cho hình lăng trụ $ABCD.A'B'C'D'$ có hình chiếu của $A'$ lên $ABCD$ trùng với $M$ là trung  điểm $AC$, đường cao hình lăng trụ là
}{
\datcot
\bonpa
{\dung{$A'M$}}
{\sai{$AA'$}}
{\sai{$AM$}}
{\sai{$B'M$}}
}
%-[câu hỏi 3]%%%%%%%%%%
\baitracnghiem{biet:b03}{%
Cho hình chóp $S.ABCD$ có $(SAB)$ và $(SAD)$ cùng vuông góc $(ABCD)$ , đường cao hình chóp là:
}{
\datcot
\bonpa
{\dung{$SA$}}
{\sai{$SB$}}
{\sai{$SC$}}
{\sai{$SD$}}
}
%-[câu hỏi 4]%%%%%%%%%%
\baitracnghiem{biet:b04}{%
Cho hình chóp $S.ABCD$ đáy là hình vuông cạnh $a, M$ là trung điểm của $AB$, mặt $SAB$ là tam giác đều nằm trong mặt phẳng vuông góc với đáy. Trọng tâm tam giác $ABC$ là $G$.  Đường cao hình chóp là:
}{
\datcot
\bonpa
{\dung{$SM$}}
{\sai{$SC$}}
{\sai{$SG$}}
{\sai{$SA$}}
}
%-[câu hỏi 5]%%%%%%%%%%
\baitracnghiem{biet:b05}{%
Cho hình chóp $S.ABCD$ có $ABCD$ là hình vuông tâm $O$ và  $SA$ vuông góc $(ABCD)$ , góc giữa $SC$ và $(SAB)$ là:
}{
\datcot
\bonpa
{\dung{$\widehat{CSB}$}}
{\sai{$\widehat{SOC}$}}
{\sai{$\widehat{ASC}$}}
{\sai{$\widehat{SAC}$}}
}
%-[câu hỏi 6]%%%%%%%%%
\baitracnghiem{biet:b06}{%
Cho lăng trụ đứng $ABC.A'B'C'$,  tam giác $ABC$ vuông tại  $B$, góc giữa $(A'BC)$ và đáy là:
}{
\datcot
\bonpa
{\dung{$\widehat{A'BA}$}}
{\sai{$\widehat{A'AC}$}}
{\sai{$\widehat{A'AB}$}}
{\sai{$\widehat{A'CA}$}}
}
%-[câu hỏi 7]%%%%%%%%%%
\baitracnghiem{biet:b07}{%
Số cạnh của hình thập nhị diện đều là:
}{
\datcot
\bonpa
{\dung{Ba mươi}}
{\sai{Mười sáu}}
{\sai{Hai mươi}}
{\sai{Mười hai}}
}
%-[câu hỏi 8]%%%%%%%%%%
\baitracnghiem{biet:b08}{%
Số đỉnh của hình nhị thập diện đều là:
}{
\datcot
\bonpa
{\dung{Mười hai}}
{\sai{Mười sáu}}
{\sai{Hai mươi}}
{\sai{Ba mươi}}
}
%-[câu hỏi 9]%%%%%%%%%%
\baitracnghiem{biet:b09}{%
Số cạnh của một hình bát diện đều là: 
}{
\datcot
\bonpa
{\dung{Mười hai}}
{\sai{Mười sáu}}
{\sai{Mười}}
{\sai{Tám }}
}
%-[câu hỏi 10]%%%%%%%%%%
\baitracnghiem{biet:b10}{%
Số đỉnh của một hình thập nhị diện đều là: 
}{
\datcot
\bonpa
{\dung{Hai mươi}}
{\sai{Mười sáu}}
{\sai{Mười hai}}
{\sai{Ba mươi}}
}
%-[câu hỏi 11]%%%%%%%%%%
\baitracnghiem{biet:b11}{%
Hãy chọn cụm từ (hoặc từ) cho dưới đây để sau khi điền nó vào chổ trống mệnh đề sau trở thành mệnh đề đúng: ``Số cạnh của một hình đa diện luôn\ldots số mặt của hình đa diện ấy'' 
}{
\datcot
\bonpa
{\dung{lớn hơn}}
{\sai{nhỏ hơn hoặc bằng}}
{\sai{nhỏ hơn}}
{\sai{bằng}}
}
%-[câu hỏi 12]%%%%%%%%%
\baitracnghiem{biet:b12}{%
Số đỉnh của một hình nhị thập diện đều là: 
}{
\datcot
\bonpa
{\dung{Mười hai}}
{\sai{Mười sáu}}
{\sai{Hai mươi}}
{\sai{Bốn mươi}}
}
%-[câu hỏi 13]%%%%%%%%%%
\baitracnghiem{biet:b13}{%
Thể tích khối tứ diện đều có cạnh bằng $a$ là: 
}{
\datcot
\bonpa
{\dung{$\dfrac{a^3\sqrt{2}}{12}$}}
{\sai{$\dfrac{a^3\sqrt{2}}{6}$}}
{\sai{$\dfrac{a^3\sqrt{3}}{4}$}}
{\sai{$\dfrac{a^3\sqrt{2}}{4}$}}
}
%-[câu hỏi 14]%%%%%%%%%%
\baitracnghiem{biet:b14}{%
Thể tích của khối lăng trụ tam giác đều có tất cả các cạnh bằng $a$ là: 
}{
\datcot
\bonpa
{\dung{$\dfrac{{{a}^{3}}\sqrt{3}}{4}$}}
{\sai{$\dfrac{{{a}^{3}}\sqrt{2}}{3}$}}
{\sai{$\dfrac{{{a}^{3}}\sqrt{2}}{2}$}}
{\sai{$\dfrac{{{a}^{3}}}{2}$}}
}
%-[câu hỏi 15]%%%%%%%%%%
\baitracnghiem{biet:b15}{%
Thể tích khối bát diện đều có cạnh bằng $a$ là: 
}{
\datcot
\bonpa
{\dung{$\dfrac{a^3\sqrt{2}}{3}$}}
{\sai{$\dfrac{a^3\sqrt{2}}{6}$}}
{\sai{$\dfrac{a^3\sqrt{3}}{3}$}}
{\sai{$\dfrac{a^3\sqrt{2}}{4}$}}
}