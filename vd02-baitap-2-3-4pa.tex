% Tệp mẫu làm đề thi trắc nghiệm dựa vào gói lệnh lamdethi.sty
% Tác giả: Nguyên Hữu Điển
% Khoa Toán Cơ Tin học, ĐHKHTN HN, ĐHQGHN
% 334, Nguyễn Trãi, Thanh Xuân, Hà Nội
% huudien@vnu.edu.vn
% Ngày 26/12/2009
%%%%%%%%%%%%%%%%%%%%%%%%%%%%

\documentclass[11pt]{article}
\usepackage{amsmath,amsxtra,amssymb,latexsym, amscd,amsthm}
\usepackage{graphicx}
\usepackage{picinpar}
\usepackage[utf8]{vietnam}
\usepackage{longtable}%
\usepackage{multicol}%
\usepackage{tikz}
 \usepackage{shortlst}
\usepackage{enumerate}
\usepackage{color}
%%%%%%%%%%%%%%%%%%%%%
\usepackage{mathptmx} 
% \usepackage{mathpazo} 
\voffset=-3cm
% \hoffset=-2cm
\textheight 24truecm 
\textwidth 18truecm 
\usepackage[baitap]{dethi}
\tentruong{ĐẠI HỌC KHOA HỌC TỰ NHIÊN}
\tenkhoa{Khoa Toán - Cơ -Tin học}
\loaidethi{Đề gồm có \pageref{dbPage} trang}%{ĐỀ THI LẠI}%%{ĐỀ CHÍNH THỨC}
\tenkythi{ĐỀ THI GIỮA KỲ NĂM HỌC 2016-2017}
\tenmonhoc{Môn: Toán học tính toán}
\madethi{100}
\thoigian{\underline{Thời gian làm bài: 90 phút, không kể thời gian phát đề}}   
\hovaten{Họ và tên}         %Nếu không muốn có dòng này không gõ lệnh
\tenlop{Tên lớp}         %Nếu không muốn có dòng này không gõ lệnh
\sobaodanh{Số báo danh}  %Nếu không muốn có dòng này không gõ lệnh
\tieudeduoi
\usepackage{fancybox}
\cornersize*{3.6mm}
\daungoac{\Ovalbox}{}
\khoanh{\Ovalbox} %Khoanh các phương án: \cbox, \fbox
%\daungoac{(}{)}%%{[}{]}%Dấu quanh phương án trả lời: {(}{)};{}{.};{}{)}
\chuphuongan{\small\bfseries\Alph}
\mauchu{blue}
\PSNrandseed{\time}
% \coloigiai
\usepackage{centerpage}
\usepackage{lastpage}
\graphicspath{{hinh-cauhoi/}} 
\parindent 20pt
\begin{document}
% \hovaten{Họ và tên}
% \sobaodanh{Số báo danh}
\setlength{\baselineskip}{12truept}
\def\v#1{\overrightarrow{#1}} %Làm vectơ
%%%%%%%%%%%%%
% 


% \loadrandomproblems[dtracnghiem]{20}{cauhoi02-dtracnghiem}
% \loadrandomproblems[ktracnghiem]{3}{cauhoi02-tracnghiem-2-3-bonpat}
 \loadrandomproblems[tracnghiemba]{20}{vd02-cauhoi-tracnghiem-2-3-4pa}
\trangcuoi{dbPage}
\lamtieude
\indebai
\begin{enumerate}[]
% \foreachproblem[dtracnghiem]{\item\label{prob:\thisproblemlabel}\thisproblem}
\foreachproblem[tracnghiemba]{\item\causo\thisproblem}
\end{enumerate}
\label{dbPage}

\newpage
\setcounter{page}{1}
\trangcuoi{dbdaPage}
\lamtieude
\indebaidapan
\begin{center}
{\bf ĐỀ BÀI VÀ ĐÁP ÁN }
\end{center}
\begin{enumerate}[]
\foreachproblem[tracnghiemba]{\item\causo\thisproblem}
\end{enumerate}
\label{dbdaPage}

\newpage
% \setcounter{page}{1}
\lamtieude
\indapanrutgon
\thispagestyle{empty}
\begin{center}
{\bf ĐÁP ÁN RÚT GỌN}
\end{center}
\begin{center}
\begin{multicols}{3}
\begin{enumerate}[\causo]
\foreachproblem[tracnghiemba]{\item\thisproblem}
\end{enumerate}
\end{multicols}
\end{center}

\newpage
\lamphieuthi
\inphieuthi
\begin{center}
\begin{multicols}{3}
\begin{enumerate}[\causo]
 \foreachproblem[tracnghiemba]{\item\thisproblem}
\end{enumerate}
\end{multicols}
\end{center}
\addtocounter{page}{-2}
\end{document}
