% Tệp mẫu làm đề thi trắc nghiệm dựa vào gói lệnh dethi.sty 3.2
% Tác giả: Nguyên Hữu Điển
% Khoa Toán Cơ Tin học, ĐHKHTN HN, ĐHQGHN
% 334, Nguyễn Trãi, Thanh Xuân, Hà Nội
% huudien@vnu.edu.vn
% Ngày 26/12/2009
%%%%%%%%%%%%%%%%%%%%%%%%%%%%
\documentclass[12pt,openany]{article}
\usepackage{amsmath,amsxtra,amssymb,latexsym, amscd,amsthm}
\usepackage{graphicx}
\usepackage{picinpar}
\usepackage{tikz}
\usetikzlibrary{arrows}
\usepackage{tkz-tab}
\usepackage[utf8]{vietnam}
\usepackage{longtable}%
\usepackage{multicol}%
\usepackage{color}
 \usepackage{shortlst}
\usepackage{enumitem}
%%%%%%%%%%%%%%%%%%%%%
\usepackage{mathpazo} 
\voffset=-2cm
% \hoffset=-2cm
\textheight 24truecm 
\textwidth 18truecm 
\usepackage[baitap]{dethi}
\tentruong{ĐẠI HỌC KHOA HỌC TỰ NHIÊN}
\tenkhoa{Khoa Toán - Cơ -Tin học}
\loaidethi{Đề gồm có \pageref{LastPage} trang}%{ĐỀ THI LẠI}%%{ĐỀ CHÍNH THỨC}
\tenkythi{ĐỀ THI GIỮA KỲ NĂM HỌC 2016-2017}
\tenmonhoc{Môn: Toán học tính toán}
\madethi{100}
\thoigian{\underline{Thời gian làm bài: 90 phút, không kể thời gian phát đề}}   
\hovaten{Họ và tên}         %Nếu không muốn có dòng này không gõ lệnh
\tenlop{Tên lớp}         %Nếu không muốn có dòng này không gõ lệnh
\sobaodanh{Số báo danh}  %Nếu không muốn có dòng này không gõ lệnh
\setlength{\shortitemwidth}{0.12\textwidth}
 \usepackage{fancybox}
\cornersize*{5mm}
%\daungoac{\Ovalbox}{}
\daungoac{\cbox}{}
\khoanh{\cbox}
% \khoanh{\ovalbox}
\usepackage{tikz}
\khoanh{\cboxv}
\daungoac{\cboxx}{}%%{[}{]}%Dấu quanh phương án trả lời: {(}{)};{}{.};{}{)}
\chuphuongan{\small\bfseries\Alph}
\mauchu{blue}
\PSNrandseed{\time}
\usepackage{centerpage}
\usepackage{lastpage}
\graphicspath{{hinh-cauhoi/}} 
\parindent 20pt
\begin{document}
\soanthao
\soanquestion
\lamtieude
\indebailoigiai
%%%%%%%%%%%%%%%%%%%%%%%%%%%%%%%%%%%
\begin{question}
Tìm tất cả các giá trị thực của tham số $m$ sao cho hàm số $y=\dfrac{\tan x-2}{\tan x -m}$ đồng biến trên khoảng $\left(0;\dfrac{\pi}{4}\right)$.
\datcot[2]
\bonpa
{\dung{$m\le 0$ hoặc $1\le m <2$.}}
{\sai{$m\le 0$.}}
{\sai{$\le m < 2$.}}
{\sai {$m\ge 2$.}}
\loigiai{
$$y'=\dfrac{\dfrac{1}{\cos^2x}(\tan x-m)-\dfrac{1}{\cos^2x}(\tan x-2)}{(\tan x-m)^2}=\dfrac{2-m}{\cos^2x(\tan x-m)^2}.$$
Hàm số đồng biến trên $\left(0;\dfrac{\pi}{4}\right)$ khi và chỉ khi hàm số xác định trên $\left(0;\dfrac{\pi}{4}\right)$ và $y'\ge 0$ $\forall x\in \left(0;\dfrac{\pi}{4}\right)$\\
$\Leftrightarrow \begin{cases}
\tan x\ne m,&\forall x\in \left(0;\dfrac{\pi}{4}\right)\\
2-m\ge 0&
\end{cases}\Leftrightarrow \left[\begin{matrix}
m\le 0\\ 
1\le m \le 2.\\ 
\end{matrix}\right.$.
}
\end{question}

\begin{question}
Giải phương trình $\log_4(x-1)=3$.
\datcot
\bonpa
{\sai{$x=63$.}}
{\dung{$x=65$.}}
{\sai{$x=80$.}}
{\sai {$x=82$.}}
\loigiai{ 
Điện $x>1$.\\
Phương trình $\Leftrightarrow x-1=64\Leftrightarrow x=65$.
}
\end{question}

\begin{question}
Tính đạo hàm của hàm số $y=13^x$.
\datcot
\bonpa
{\sai{$y'=x.13^{x-1}$.}}
{\dung{$y'=13^{x}.\ln 13$.}}
{\sai{$y'=13^{x}$.}}
{\sai {$y'=\dfrac{13^{x}}{\ln 13}$.}}
\loigiai{
 $y'=13^x.\ln 13$.
}
\end{question}

\begin{question}
Giải bất phương trình $\log_2(3x-1)>3$.
\datcot
\bonpa
{\dung{$x>3$.}}
{\sai{$\dfrac{1}{3}<x<3$.}}
{\sai{$x<3$.}}
{\sai {$x>\dfrac{10}{3}$.}}
\loigiai{
Điều kiện: $x>\dfrac{1}{3}$. BPT $\Leftrightarrow 3x-1>8\Leftrightarrow x>3$.
Kết hợp điều kiện ta được $x > 3$.
}
\end{question}

\begin{question}
Tìm tập xác định $\mathcal{D}$ của hàm số $y=\log_2(x^2-2x-3)$.
\datcot[2]
\bonpa
{\sai{$\mathcal{D}=(-\infty;-1]\cup [3;+\infty)$.}}
{\sai{$\mathcal{D}=[-1;3]$.}}
{\dung{$\mathcal{D}=(-\infty;-1)\cup (3;+\infty)$.}}
{\sai {$\mathcal{D}=(-1;3)$.}}
\loigiai{
$x^2-2x-3>0\Leftrightarrow x\in (-\infty;-1)\cup (3;+\infty)$. 
}
\end{question}

\end{document}
